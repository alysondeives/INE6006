\documentclass[10pt,a4paper,oneside]{article}
\usepackage[utf8]{inputenc}
\usepackage{amsmath}
\usepackage{amsfonts}
\usepackage{amssymb}
\usepackage[left=3cm,right=3cm,top=2cm,bottom=2cm]{geometry}



\usepackage{booktabs}

\author{%
	Alexis A. Huf, %
	Alyson Pereira, %
	Bruno N. Oliveira,\\%
	Eliza Gomes, %
	Pedro H. Penna
	}

\title{Lista de Exercício I}


\begin{document}

\maketitle

\paragraph{Questão 01}

A Tabela \ref{table: dados perdidos} apresenta os dados perdidos e seus respectivos percentuais na base de dados em estudo. Para todos as variáveis o percentual observado de perda foi inferior a $5\%$, o que é aceitável.

\begin{table}[h]
\centering
\caption{Dados perdidos na base de dados.}
\label{table: dados perdidos}
\begin{tabular}{c c c c c c}
	\toprule
	\textbf{Região}         & \textbf{Área}          & \textbf{Pagamento}     & \textbf{Opinião}       & \textbf{Renda}         & \textbf{Idade} \\
	\midrule
	21 $(0.42\%)$  & 21 $(0.42\%)$ & 18 $(0.36\%)$ & 19 $(0.38\%)$ & 14 $(0.28\%)$ & 13 $(0.26\%)$ \\
	\bottomrule
\end{tabular}
\end{table}


\end{document}