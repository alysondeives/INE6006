\documentclass[10pt,a4paper,oneside]{article}
\usepackage[brazil]{babel}
\usepackage[utf8]{inputenc}
\usepackage{amsmath}
\usepackage{amsfonts}
\usepackage{amssymb}
\usepackage[left=3cm,right=3cm,top=2cm,bottom=2cm]{geometry}

\usepackage{xspace}

\newcommand{\arat}{Aratibutantã\xspace}
\newcommand{\baep}{Baependinha\xspace}
\newcommand{\itam}{Itamaracanã\xspace}
\newcommand{\jaqu}{Jaquereçaba\xspace}
\newcommand{\para}{Paranapitanga\xspace}

\newcommand{\adm}{Administração\xspace}
\newcommand{\comp}{Computação e Matemática\xspace}
\newcommand{\edu}{Educacional\xspace}
\newcommand{\eng}{Engenharia e Produção\xspace}
\newcommand{\hum}{Humanidades\xspace}
\newcommand{\jur}{Jurídica e Contábil\xspace}

% Tabelas.
\usepackage{booktabs}
\newcommand{\specialcell}[3][c]
{\begin{tabular}[#1]{@{}#2@{}}#3\end{tabular}}

\author{%
	Alexis A. Huf, %
	Alyson Pereira, %
	Bruno N. Oliveira,\\%
	Eliza Gomes, %
	Pedro H. Penna
	}

\title{Lista de Exercício I}


\begin{document}

\maketitle

\paragraph{Questão 01}

A Tabela \ref{table: dados perdidos} apresenta os dados perdidos e seus respectivos percentuais na base de dados em estudo. Para todos as variáveis o percentual observado de perda foi inferior a $5\%$, o que é aceitável.

\begin{table}[h]
\centering
\caption{Dados perdidos na base de dados.}
\label{table: dados perdidos}
\vspace{0.5em}
\begin{tabular}{c c c c c c}
	\toprule
	\textbf{Região} & \textbf{Área} & \textbf{Pagamento} & \textbf{Opinião} & \textbf{Renda} & \textbf{Idade} \\
	\midrule
	21 $(0.42\%)$   & 21 $(0.42\%)$ & 18 $(0.36\%)$      & 19 $(0.38\%)$    & 14 $(0.28\%)$  & 13 $(0.26\%)$ \\
	\bottomrule
\end{tabular}
\end{table}

%
% Dúvidas:
%   Devemos detalhar quis os erros (ex: Arábia, Araba)?
%
\paragraph{Questão 02}

A Tabela \ref{table: dados errados} resume os erros de coleta e seus respectivos percentuais na base de dados em estudo. Para as variáveis qualitativas \textit{Região}, \textit{Área}, \textit{Pagamento} e \textit{Opinião}, foram observados erros de ortografia e digitação. Para as variáveis quantitativas \textit{Renda} e \textit{Idade}, não foram observados erros na base de dados, isto é valores infactíveis (\textit{Renda} não positiva, \textit{Idade} inferior a 17 e superior a 100 anos).

Para evitar que erros de coleta sejam cometidos nas variáveis qualitativas, um formulário de múltipla escolha. Já para as variáveis quantitativas, no caso de um formulário eletrônico, limites inferiores e superiores de valores válidos poderiam ser checado previamente à submissão da observação.

\begin{table}[!h]
\centering
\caption{Erros na base de dados.}
\vspace{0.5em}
\label{table: dados errados}
\begin{tabular}{c c c c c c}
	\toprule
	\textbf{Região} & \textbf{Área}  & \textbf{Pagamento} & \textbf{Opinião} & \textbf{Renda} & \textbf{Idade} \\
	\midrule
	135 $(2.71\%)$  & 114 $(2.28\%)$ & 132 $(2.64\%)$     & 123 $(2.46\%)$   & 0 $(0.00\%)$   & 0 $(0.00\%)$ \\
	\bottomrule
\end{tabular}
\end{table}

\paragraph{Questão 03}

A tabela de frequências para a variável \textit{Região} está indicada na Tabela \ref{table: tabela frequencias regiao}. Por análise, pode-se observar que a \textit{Região} predominante na base dados é \textit{\baep}.

\begin{table}[!h]
\centering
\caption{Frequências para a variável \textit{Região}.}
\label{table: tabela frequencias regiao}
\vspace{0.5em}
\begin{tabular}{c c c c c}
	\toprule
	\textbf{\arat}    & \textbf{\baep}   & \textbf{\itam}  & \textbf{\jaqu}  & \textbf{\para} \\
	\midrule
	1185 $(23.80\%)$  & 2294 $(46.07\%)$ & 843 $(16.93\%)$ & 536 $(10.77\%)$ & 121 $(2.43\%)$ \\
	\bottomrule
\end{tabular}
\end{table}

\paragraph{Questão 04}

A tabela de frequências para a variável \textit{Área} está indicada na Tabela \ref{table: tabela frequencias area}. Por análise, pode-se observar que a \textit{Área} predominante da \texttt{TYU} modificou-se para \eng.


\begin{table}[h]
\centering
\caption{Frequências para a variável \textit{Área}.}
\label{table: tabela frequencias area}
\vspace{0.5em}
\begin{tabular}{c c c c c c}
	\toprule
	\textbf{\adm}   & \textbf{\comp} & \textbf{\edu} & \textbf{\eng}    & \textbf{\hum}   & \textbf{\jur} \\
	\midrule
	592 $(11.89\%)$ & 296 $(5.94\%)$ & 338 $(6.79\%)$ & 1741 $(34.97\%)$ & 503 $(10.10\%)$ & 1509 $(30.31\%)$ \\
	\bottomrule
\end{tabular}
\end{table}

\paragraph{Questão 04}

A tabela de frequências para a variável \textit{Área} está indicada na Tabela \ref{table:frequencias-area}. Por análise, pode-se observar que a área predominante na base dados é \textit{\edu}.

\begin{table}[!h]
\centering
\begin{tabular}{c c c c c c}
	\toprule
	\textbf{Adm.}    & \textbf{Comp. e Mat.}   & \textbf{Eng. e Prod.}  & \textbf{Educacional} & \textbf{Humanidades}  & \textbf{Jur. e Cont.} \\
	\midrule
	592 $(12\%)$  & 296 $(6\%)$ & 338 $(7\%)$ & 1741 $(35\%)$ & 503 $(10\%)$ & 1509 $(30\%)$ \\
	\bottomrule
\end{tabular}
\caption{Frequências para a variável \textit{Área}.}
\label{table:frequencias-area}
\end{table}


\paragraph{Questão 05}

A tabela de frequências para a variável \textit{Pagamento} está indicada na Tabela \ref{table:frequencias-pagamento}. Por análise, pode-se observar que a área predominante na base dados é \textit{Financiamento Bancário}.

\begin{table}[!h]
\centering
\begin{tabular}{c c c c c}
	\toprule
	\textbf{Aux. de Fam.}    & \textbf{Bolsas}   & \textbf{Financ. Banc.}  & \textbf{Inc. Federais} & \textbf{Rec. Próprios} \\
	\midrule
	257 $(5.16\%)$ & 328 $(6.58\%)$ & 2191 $(43.98\%)$ & 1448 $(29.06\%)$ & 758 $(15.21\%)$ \\
	\bottomrule
\end{tabular}
\caption{Frequências para a variável \textit{Pagamento}.}
\label{table:frequencias-pagamento}
\end{table}

\paragraph{Questão 06}

A tabela de frequências para a variável \textit{Opinião} está indicada na Tabela \ref{table:frequencias-opiniao}. Por análise, pode-se observar que a área predominante na base dados é \textit{Muito satisfeito}.

\begin{table}[!h]
\centering
\begin{tabular}{c c c c c}
	\toprule
	\textbf{Indiferente}    & \textbf{Insatisfeito}   & \textbf{Muito insatisfeito}  & \textbf{Muito satisfeito} & \textbf{Satisfeito} \\
	\midrule
	1006 $(20.20\%)$ & 749 $(15.04\%)$ & 472 $(9.48\%)$ & 1719 $(34.51\%)$ & 1035 $(20.78\%)$ \\
	\bottomrule
\end{tabular}
\caption{Frequências para a variável \textit{Opinião}.}
\label{table:frequencias-opiniao}
\end{table}

\end{document}
