\documentclass[10pt,a4paper,oneside]{article}
\usepackage[utf8]{inputenc}
\usepackage{amsmath}
\usepackage{amsfonts}
\usepackage{amssymb}
\usepackage[left=3cm,right=3cm,top=2cm,bottom=2cm]{geometry}


\usepackage{booktabs}

\author{%
	Alexis A. Huf, %
	Alyson Pereira, %
	Bruno N. Oliveira,\\%
	Eliza Gomes, %
	Pedro H. Penna
	}

\title{Lista de Exercício I}


\begin{document}

\maketitle

\paragraph{Questão 01}

A Tabela \ref{table: dados perdidos} apresenta os dados perdidos e seus respectivos percentuais na base de dados em estudo. Para todos as variáveis o percentual observado de perda foi inferior a $5\%$, o que é aceitável.

\begin{table}[h]
\centering
\begin{tabular}{c c c c c c}
	\toprule
	\textbf{Região} & \textbf{Área} & \textbf{Pagamento} & \textbf{Opinião} & \textbf{Renda} & \textbf{Idade} \\
	\midrule
	21 $(0.42\%)$   & 21 $(0.42\%)$ & 18 $(0.36\%)$      & 19 $(0.38\%)$    & 14 $(0.28\%)$  & 13 $(0.26\%)$ \\
	\bottomrule
\end{tabular}
\caption{Dados perdidos na base de dados.}
\label{table: dados perdidos}
\end{table}

%
% Dúvidas:
%   Devemos detalhar quis os erros (ex: Arábia, Araba)?
%
\paragraph{Questão 02}

A Tabela \ref{table: dados errados} resume os erros de coleta e seus respectivos percentuais na base de dados em estudo. Para as variáveis qualitativas \textit{Região}, \textit{Área}, \textit{Pagamento} e \textit{Opinião}, foram observados erros de ortografia e digitação. Para as variáveis quantitativas \textit{Renda} e \textit{Idade}, não foram observados erros na base de dados, isto é valores infactíveis (\textit{Renda} não positiva, \textit{Idade} inferior a 17 e superior a 100 anos).

Para evitar que erros de coleta sejam cometidos nas variáveis qualitativas, um formulário de múltipla escolha. Já para as variáveis quantitativas, no caso de um formulário eletrônico, limites inferiores e superiores de valores válidos poderiam ser checado previamente à submissão da observação.

\begin{table}[h]
\centering
\begin{tabular}{c c c c c c}
	\toprule
	\textbf{Região} & \textbf{Área}  & \textbf{Pagamento} & \textbf{Opinião} & \textbf{Renda} & \textbf{Idade} \\
	\midrule
	135 $(2.71\%)$  & 114 $(2.28\%)$ & 132 $(2.64\%)$     & 123 $(2.46\%)$   & 0 $(0.00\%)$   & 0 $(0.00\%)$ \\
	\bottomrule
\end{tabular}
\caption{Erros na base de dados.}
\label{table: dados errados}
\end{table}


\end{document}