\documentclass[10pt,a4paper,oneside]{article}
\usepackage[brazil]{babel}
\usepackage[utf8]{inputenc}
\usepackage{amsmath}
\usepackage{amsfonts}
\usepackage{amssymb}
\usepackage[output-decimal-marker={,}]{siunitx}
\usepackage[left=3cm,right=3cm,top=2cm,bottom=2cm]{geometry}
\usepackage{hyperref} %\autoref
\usepackage{todonotes} % todos
\usepackage{placeins} %\FloatBarrier
\usepackage{caption}
\usepackage{subcaption}
\usepackage{capt-of} %Workarounds for manually inserting captions
%\usepackage{subcaption} %subfigure environment
\usepackage{xspace}
\usepackage{graphicx}
\usepackage{multirow}
\usepackage{booktabs}
\usepackage{soulutf8}
%\usepackage{titlesec} %Customization of headings


%%%%%%%%%%%% Configuring packages

\newcommand{\specialcell}[3][c]
{\begin{tabular}[#1]{@{}#2@{}}#3\end{tabular}}

\renewcommand{\thesection}{\arabic{section}}
\renewcommand{\thesubsection}{\arabic{section}.\alph{subsection}}

%%%%%%%%%%%%% Macros
\newcommand{\arat}{Aratibutantã\xspace}
\newcommand{\baep}{Baependinha\xspace}
\newcommand{\itam}{Itamaracanã\xspace}
\newcommand{\jaqu}{Jaquereçaba\xspace}
\newcommand{\para}{Paranapitanga\xspace}

\newcommand{\adm}{Administração\xspace}
\newcommand{\comp}{Computação e Matemática\xspace}
\newcommand{\edu}{Educacional\xspace}
\newcommand{\eng}{Engenharia e Produção\xspace}
\newcommand{\hum}{Humanidades\xspace}
\newcommand{\jur}{Jurídica e Contábil\xspace}

%%%%%%%%%%%%% Autor e Título

\author{%
	Alexis A. Huf, %
	Alyson D. Pereira, %
	Bruno C. N. Oliveira,\\%
	Eliza Gomes, %
	Pedro H. Penna
	}

\title{Lista de Exercícios II - Grupo 6}


\begin{document}

\maketitle

\begin{center}
	\section*{Parte 1: Distribuição Amostral da média}
\end{center}

\section{Variável Renda}
\label{questao:1}
\newcommand{\TRESicmin}{1.85\xspace}
\newcommand{\TRESicmax}{3.10\xspace}
\newcommand{\TRESicNoveNoveMin}{1.62\xspace}
\newcommand{\TRESicNoveNoveMax}{3.33\xspace}
\newcommand{\TRESnZero}{26\xspace}


Foram retiradas 200 amostras, sem reposição, com tamanhos 4, 8, 16, 30 e 100, 
dentre os 4986 alunos que possuíam um valor definido para a variável Renda.
A população possuí média de ${\UMx}$ para a variável Renda, com desvio padrão de
${\UMsd}$. A \autoref{tab:q1} apresentam os valores esperados da média amostral
e seu respectivos desvios padrões.

\begin{table}[h]
\centering
\caption{Valores das médias amostrais para a variável \textit{Renda}}
\label{tab:q1}
\vspace{0.5em}
\begin{tabular}{l r r r}
	\toprule
	\textbf{n} & \textbf{Média} & \textbf{Desvio Padrão} & \textbf{$\frac{\sigma}{\sqrt{n}}$}\\
	\midrule
	$4$       & ${\UMxQuatro}$   & ${\UMsdQuatro}$   & ${\UMsdeQuatro}$   \\
	$8$       & ${\UMxOito}$   & ${\UMsdOito}$   & ${\UMsdeOito}$   \\
	$16$      & ${\UMxDezesseis}$  & ${\UMsdDezesseis}$  & ${\UMsdeDezesseis}$  \\
	$30$      & ${\UMxTrinta}$  & ${\UMsdTrinta}$  & ${\UMsdeTrinta}$  \\
	$100$     & ${\UMxCem}$ & ${\UMsdCem}$ & ${\UMsdeCem}$ \\
	\bottomrule
\end{tabular}
\end{table}

\subsection{(item a)}
A medida que se aumentou o tamanho da amostra, de 4 para 16, o resultado se aproximou à media populacional.
Porém para amostra de tamanho 30 e 100, a média amostral se distanciou da média populacional.


\subsection{(item b)}
Sim, as amostras retiradas confirmam a afirmação. Conforme indicado na \autoref{tab:q1}, os valores 
de desvio padrão das amostras são próximos aos valores de $\frac{\sigma}{sqrt{n}}$ e para a amostra de tamanho 100, 
são exatamente iguais.

\subsection{(item c)}
Como gerar o gráfico? Escolher uma amostra dentre as 200 para algum tamanho? Qual seria o valor da distribuição?



\begin{center}
	\section*{Parte 2: Intervalos de Confiança para Média}
\end{center}

\section{Variável Idade}
\label{questao:2}
\subsection{Intervalo de 95\% de confiança para a média populacional da Idade dos alunos}
\label{sub:1a}
	
	Após retirar os dados perdidos da variável Idade, foi retirada uma
	amostra aleatória de 20 idades, por meio de sorteio. Com a amostra
	adquirida, tem-se média $\bar{x}$ = 31,85 e desvio padrão s = 6,11534.
	Com o desvio padrão da população desconhecido, utiliza-se o desvio
	padrão da amostra (s = 6,11534).  O tamanho da população é conhecido (N
	= 4987) e o da amostra é pequeno, $n < 30$, então, nesse caso utiliza-se
	a \textit{distribuição t de Student}.

	Para obter o intervalo de confiança para a média em uma população
	conhecida, é possível utilizar a seguinte equação:

	
	\begin{align}
		\label{eq:dois-a-expr}
		IC (\mu, \gamma) = \bar{x} \pm t_\gamma \frac{s}{\sqrt{n}} \sqrt{\frac{N-n}{N-1}}
	\end{align}

	Para um intervalo de confiança de 95\%, obtém-se o valor da cauda
	superior 0,025. Diante disso, procura-se na tabela \textit{distribuição
	t de student} o grau de liberdade 19, pois gl = n-1, e o valor 0,025,
	resultando em \textit{t = 2,093}. Inserindo os valores na equação
	tem-se:

	\begin{align*}
		IC (\mu, 95\%) &= 31,85 \pm 2,093 \frac{6,11534}{\sqrt{20}} \sqrt{\frac{4987 - 20}{4987 - 1}} \\
		IC (\mu, 95\%) &= 31,85 \pm 2,8565
	\end{align*}

	A média de idades de amostra é 31,85 anos, com o nível de confiança de
	95\%, a margem de erro é de 2,8565 anos para mais ou para menos. Diante
	disso, o intervalo de confiança é de [28,9935; 34,7065].

\subsection{Intervalo de 99\% de confiança para a média populacional da Idade dos alunos, com uma precisão de 2 anos}

	Com o desvio padrão da população desconhecido, utiliza-se como amostra
	piloto n = 20 e desvio padrão s = 6,11534, obtidos anteriormente. Como o
	tamanho da amostra é pequeno ($n < 30$) utiliza-se a
	\textit{distribuição t student}.

	Para saber o tamanho da amostra necessário para estimar a média
	populacional da idade dos alunos, com erro amostral máximo tolerado de 2
	anos, utiliza-se a seguinte equação:

	\begin{align}
		\label{eq:dois-b-expr1}
		 n_0 = \left (\frac{t_\gamma s}{E_0} \right)^2
	\end{align}

	Sabendo-se que o grau de liberdade é 19, pois gl = n - 1, e o nível de
	confiança é de 99\%, o valor da cauda superior será de 0,005. Obtendo os
	valores da tabela de \textit{distribuição t student}, t = 2,861.
	Inserindo os valores na equação, tem-se:

	\begin{align*}
		n_0 = \left (\frac{2,861 \times 6,11534}{2} \right)^2 = 76,52 \cong 77
	\end{align*}

	Como o tamanho da população não é grande e é conhecido, adicionalmente,
	utiliza-se a seguinte equação para calcular o tamanho da amostra.

	\begin{align}
		\label{eq:dois-b-expr2}
		n = \frac{N . n_0}{N + n_0}
	\end{align}

	Inserindo os valores na equação, tem-se:

	\begin{align*}
		n &= \lceil \frac{4987 \times 76,52}{4987 + 76,52} \rceil \\
		n &= \lceil 75,3636 \rceil \\
		n &= 76
	\end{align*}

	Diante dos cálculos apresentados acima, pode-se dizer que uma amostra
	com 20 elementos não é suficiente para encontrar um intervalo de 99\% de
	confiança para a média populacional da idade dos alunos com um erro
	máximo tolerado de 2 anos. Seria necessário, no mínimo $76$ elementos da
	amostra para garantir certa precisão.

\subsection{Você concorda com o plano de amostragem usado, que considerou a população homogênea}

	Com o objetivo de caracterizar melhor o perfil dos alunos através da
	variável idade, o plano de amostragem utilizado, considerando a
	população homogênea, não é o ideal, uma vez que a variável idade está
	associada a pelo menos uma outra variável. Por exemplo, existe uma
	associação entre a variável Idade e a variável Opinião, mostrando que os
	alunos com idades inferiores a 30 anos possuem maiores percentuais
	(72,57\%) para a opinião \textquotedblleft Muito
	Satisfeitos\textquotedblright, enquanto os alunos com idades superiores
	a 30 anos possuem maiores percentuais (25,62\%) para a opinião
	\textquotedblleft Indiferente\textquotedblright, não obtendo flutuação
	maior do que 5\% para as opiniões \textquotedblleft
	Satisfeitos\textquotedblright (21,94\%) e \textquotedblleft
	Insatisfeitos\textquotedblright (21,28\%). Além disso, o perfil dos
	alunos pode ser analisado através da associações entre outras variáveis,
	como por exemplo, Opinião e Pagamento e Pagamento e Região, conforme
	apresentado na Lista 1. 

	Como resultado da associação entre Opinião e Pagamento, observou-se que
	os alunos que pagam sua mensalidade com Auxílio de Familiares são mais
	\textquotedblleft Indiferentes\textquotedblright (36,96\%), com Bolsas
	de Estudos são mais \textquotedblleft Indiferentes\textquotedblright
	(38,72\%), com Financiamento Bancário são mais \textquotedblleft Muito
	Satisfeitos\textquotedblright (66,36\%), com Incentivos Federal são mais
	\textquotedblleft Insatisfeitos\textquotedblright (35,30\%) e com
	Recursos Próprios são mais \textquotedblleft
	Satisfeitos\textquotedblright (34,48\%).

	Por outro lado, a associação entre as variáveis Pagamento e Região
	apresentou o seguinte resultado: os alunos de Aratibutantã utilizam,
	maioritariamente, a forma de pagamento por \textquotedblleft Incentivos
	Federais\textquotedblright (47,51\%), os de Baependinha utilizam o
	\textquotedblleft Financiamento Bancário\textquotedblright (53,01\%), os
	de Itamaracanã utilizam o \textquotedblleft Financiamento Bancário
	\textquotedblright (90,98\%), os de Jaquereçaba utilizam os
	\textquotedblleft Incentivos Federais\textquotedblright (76,68\%) e por
	fim, os alunos de Paranapitanga utilizam, maioritariamente, a forma de
	pagamento \textquotedblleft Incentivos Federais\textquotedblright
	(95,87\%). 

	Para caracterizar melhor o perfil dos alunos, poderia ser realizado um
	plano de amostragem estratificada proporcional, uma vez que se conhece,
	neste caso, as características da população (quantos e quais são os
	tamanhos dos estratos). 


\section{Variável Renda}
\label{questao:3}
\newcommand{\TRESicmin}{1.85\xspace}
\newcommand{\TRESicmax}{3.10\xspace}
\newcommand{\TRESicNoveNoveMin}{1.62\xspace}
\newcommand{\TRESicNoveNoveMax}{3.33\xspace}
\newcommand{\TRESnZero}{26\xspace}


\subsection{Intervalo de confiança de 95\% para Renda}
\todo[inline]{Desenvolvimento?}
O intervalo encontrado com grau de confiança de 95\% é de \TRESicmin à \TRESicmax.
\todo[inline]{Interpretação do resultado?}

\subsection{Intervalo de Confiança de 99\% para Renda}
\todo[inline]{Desenvolvimento?}
O intervalo encontrado com grau de confiança de 99\% é de \TRESicNoveNoveMin à \TRESicNoveNoveMax.
O intervalo ficou maior em relação ao intervalo com grau de confiança de 95\%.{}
Para diminuir esse intervalo seria necessário aumentar o tamanho da amostra.

\subsection{Tamanho mínimo de amostra}
\todo[inline]{Desenvolvimento? No mínimo citar as equações usadas (o professor não tem acesso ao script)}
\todo[inline]{Erro: Aqui está sendo calculado $n_0$. Como $N$ (tamanho da população) é conhecido, \\ $n=\frac{N \cdot n_0}{N + n_0 - 1}$}
\todo[inline]{Mencionar que a variância populacional foi estimada a partir da amostra já usada nessa questão, considerando a distribuição $t$ com $gl=19$.}
Para se obter um intervalo de confiança de 99\% com precisão de \num{1.5} salários
minimos, seria necessário uma amostra de tamanho \TRESnZero.

Obs: precisao de \num{1.5} quer dizer um intervalo de \num{1.5}, entao E = \num{1.5}/2?
\todo[inline]{Precisão de \num{1.5} quer dizer $E_0=\num{1.5}$}


\newpage
\begin{center}
	\section*{Parte 3: Intervalos de Confiança para Proporção}
\end{center}

	Nas questões que se seguem, são estudados os intervalos de confiança para a
	proporção de algumas variáveis da base. Para fins de sintése, considere as
	seguintes equações.
	
	A estimação do parâmetro proporção da população foi feita com base na
	Equação (??).

		\begin{equation}
			IC(p, \gamma) = \hat{p} \pm z_\gamma \sigma_{\hat{p}}
			\label{equation: intervalo de confianca para proporcao 1}
		\end{equation}
	
	Alternativamente, quando $\sigma_{\hat{p}}$ não é conhecido, e $n$ é
	suficientemente grande ($n \geq 50$), $\sigma_{\hat{p}}$ pode
	ser aproximado por $s_{\hat{p}}$, o desvio padrão da amostra tomada, e o
	parâmetro proporção pode ser estimado de acordo com a Equação ??.

		\begin{equation}
			IC(p, \gamma) = \hat{p} \pm z_\gamma \sqrt{\frac{\hat{p}(1-\hat{p})}{n}} \\
			\label{equation: intervalo de confianca para proporcao 2}
		\end{equation}

\section{Variável Pagamento}
\label{questao:4}




<!DOCTYPE html>
<html lang="en" class=" is-copy-enabled is-u2f-enabled">
  <head prefix="og: http://ogp.me/ns# fb: http://ogp.me/ns/fb# object: http://ogp.me/ns/object# article: http://ogp.me/ns/article# profile: http://ogp.me/ns/profile#">
    <meta charset='utf-8'>

    <link crossorigin="anonymous" href="https://assets-cdn.github.com/assets/frameworks-3c1694fab1568340f2e75b26efa1f55d97c5153364a7357e9e1104c718ff1a2f.css" integrity="sha256-PBaU+rFWg0Dy51sm76H1XZfFFTNkpzV+nhEExxj/Gi8=" media="all" rel="stylesheet" />
    <link crossorigin="anonymous" href="https://assets-cdn.github.com/assets/github-3e95d73eb454e0099947df00d91ab0dbfc6b10be69dd5daf5de7aeb676580d20.css" integrity="sha256-PpXXPrRU4AmZR98A2Rqw2/xrEL5p3V2vXeeutnZYDSA=" media="all" rel="stylesheet" />
    
    
    
    

    <link as="script" href="https://assets-cdn.github.com/assets/frameworks-f8175c23360b42a4eb18b2319fefeae252cfeea482fb804356f4136a52bfddb3.js" rel="preload" />
    
    <link as="script" href="https://assets-cdn.github.com/assets/github-723b665a2ff0acc9038f853f36908a2ab523595d75d4e0ded1d9138b5a7f5526.js" rel="preload" />

    <meta http-equiv="X-UA-Compatible" content="IE=edge">
    <meta http-equiv="Content-Language" content="en">
    <meta name="viewport" content="width=1020">
    
    
    <title>INE6006/questao4.tex at master · alysondp/INE6006</title>
    <link rel="search" type="application/opensearchdescription+xml" href="/opensearch.xml" title="GitHub">
    <link rel="fluid-icon" href="https://github.com/fluidicon.png" title="GitHub">
    <link rel="apple-touch-icon" href="/apple-touch-icon.png">
    <link rel="apple-touch-icon" sizes="57x57" href="/apple-touch-icon-57x57.png">
    <link rel="apple-touch-icon" sizes="60x60" href="/apple-touch-icon-60x60.png">
    <link rel="apple-touch-icon" sizes="72x72" href="/apple-touch-icon-72x72.png">
    <link rel="apple-touch-icon" sizes="76x76" href="/apple-touch-icon-76x76.png">
    <link rel="apple-touch-icon" sizes="114x114" href="/apple-touch-icon-114x114.png">
    <link rel="apple-touch-icon" sizes="120x120" href="/apple-touch-icon-120x120.png">
    <link rel="apple-touch-icon" sizes="144x144" href="/apple-touch-icon-144x144.png">
    <link rel="apple-touch-icon" sizes="152x152" href="/apple-touch-icon-152x152.png">
    <link rel="apple-touch-icon" sizes="180x180" href="/apple-touch-icon-180x180.png">
    <meta property="fb:app_id" content="1401488693436528">

      <meta content="https://avatars2.githubusercontent.com/u/4824231?v=3&amp;s=400" name="twitter:image:src" /><meta content="@github" name="twitter:site" /><meta content="summary" name="twitter:card" /><meta content="alysondp/INE6006" name="twitter:title" /><meta content="INE6006 - Trabalho Prático de Estátistica" name="twitter:description" />
      <meta content="https://avatars2.githubusercontent.com/u/4824231?v=3&amp;s=400" property="og:image" /><meta content="GitHub" property="og:site_name" /><meta content="object" property="og:type" /><meta content="alysondp/INE6006" property="og:title" /><meta content="https://github.com/alysondp/INE6006" property="og:url" /><meta content="INE6006 - Trabalho Prático de Estátistica" property="og:description" />
      <meta name="browser-stats-url" content="https://api.github.com/_private/browser/stats">
    <meta name="browser-errors-url" content="https://api.github.com/_private/browser/errors">
    <link rel="assets" href="https://assets-cdn.github.com/">
    <link rel="web-socket" href="wss://live.github.com/_sockets/NDgyNDIzMTpmNzc3ZjkwYjJkMWJmNGI5ODRhYjc4ZjRjOGUzZTRkMjo5NDliMDVmMTA2MGY5OTAzOTY3ZWY5YjNkNmIyZDMxNzRlNzQ4OTAwMDI5NTQ0ZTkzMzA5YzBiNzE4NmQ3MDk1--6a05d49f3302aa6ae98e995fc9a903892cc7e48b">
    <meta name="pjax-timeout" content="1000">
    <link rel="sudo-modal" href="/sessions/sudo_modal">

    <meta name="msapplication-TileImage" content="/windows-tile.png">
    <meta name="msapplication-TileColor" content="#ffffff">
    <meta name="selected-link" value="repo_source" data-pjax-transient>

    <meta name="google-site-verification" content="KT5gs8h0wvaagLKAVWq8bbeNwnZZK1r1XQysX3xurLU">
<meta name="google-site-verification" content="ZzhVyEFwb7w3e0-uOTltm8Jsck2F5StVihD0exw2fsA">
    <meta name="google-analytics" content="UA-3769691-2">

<meta content="collector.githubapp.com" name="octolytics-host" /><meta content="github" name="octolytics-app-id" /><meta content="BAD89752:3B27:33B880:57353AAB" name="octolytics-dimension-request_id" /><meta content="4824231" name="octolytics-actor-id" /><meta content="alysondp" name="octolytics-actor-login" /><meta content="4bdd38763ddf82bcbf65e5c654a3119be267206b26bd08078a0fed6115083729" name="octolytics-actor-hash" />
<meta content="/&lt;user-name&gt;/&lt;repo-name&gt;/blob/show" data-pjax-transient="true" name="analytics-location" />



  <meta class="js-ga-set" name="dimension1" content="Logged In">



        <meta name="hostname" content="github.com">
    <meta name="user-login" content="alysondp">

        <meta name="expected-hostname" content="github.com">
      <meta name="js-proxy-site-detection-payload" content="NDhhOTNmN2RmN2M1MWM4NzJlZTQ3MjhlMjRiOTIzNTdjMTBlNzdiNjAwZWJjMDg5MGIyZTI3ZDY1ODg2MTAxMHx7InJlbW90ZV9hZGRyZXNzIjoiMTg2LjIxNi4xNTEuODIiLCJyZXF1ZXN0X2lkIjoiQkFEODk3NTI6M0IyNzozM0I4ODA6NTczNTNBQUIiLCJ0aW1lc3RhbXAiOjE0NjMxMDYyMTl9">


      <link rel="mask-icon" href="https://assets-cdn.github.com/pinned-octocat.svg" color="#4078c0">
      <link rel="icon" type="image/x-icon" href="https://assets-cdn.github.com/favicon.ico">

    <meta content="91c711a2eaac4347558f9765b468eb891648565f" name="form-nonce" />

    <meta http-equiv="x-pjax-version" content="5be251f42ac2c9b70d2d3f16b297590b">
    

      
  <meta name="description" content="INE6006 - Trabalho Prático de Estátistica">
  <meta name="go-import" content="github.com/alysondp/INE6006 git https://github.com/alysondp/INE6006.git">

  <meta content="4824231" name="octolytics-dimension-user_id" /><meta content="alysondp" name="octolytics-dimension-user_login" /><meta content="55005051" name="octolytics-dimension-repository_id" /><meta content="alysondp/INE6006" name="octolytics-dimension-repository_nwo" /><meta content="true" name="octolytics-dimension-repository_public" /><meta content="false" name="octolytics-dimension-repository_is_fork" /><meta content="55005051" name="octolytics-dimension-repository_network_root_id" /><meta content="alysondp/INE6006" name="octolytics-dimension-repository_network_root_nwo" />
  <link href="https://github.com/alysondp/INE6006/commits/master.atom" rel="alternate" title="Recent Commits to INE6006:master" type="application/atom+xml">


      <link rel="canonical" href="https://github.com/alysondp/INE6006/blob/master/lista2/questao4/questao4.tex" data-pjax-transient>
  </head>


  <body class="logged-in   env-production linux vis-public page-blob">
    <div id="js-pjax-loader-bar" class="pjax-loader-bar"></div>
    <a href="#start-of-content" tabindex="1" class="accessibility-aid js-skip-to-content">Skip to content</a>

    
    
    



        <div class="header header-logged-in true" role="banner">
  <div class="container clearfix">

    <a class="header-logo-invertocat" href="https://github.com/" data-hotkey="g d" aria-label="Homepage" data-ga-click="Header, go to dashboard, icon:logo">
  <svg aria-hidden="true" class="octicon octicon-mark-github" height="28" version="1.1" viewBox="0 0 16 16" width="28"><path d="M8 0C3.58 0 0 3.58 0 8c0 3.54 2.29 6.53 5.47 7.59 0.4 0.07 0.55-0.17 0.55-0.38 0-0.19-0.01-0.82-0.01-1.49-2.01 0.37-2.53-0.49-2.69-0.94-0.09-0.23-0.48-0.94-0.82-1.13-0.28-0.15-0.68-0.52-0.01-0.53 0.63-0.01 1.08 0.58 1.23 0.82 0.72 1.21 1.87 0.87 2.33 0.66 0.07-0.52 0.28-0.87 0.51-1.07-1.78-0.2-3.64-0.89-3.64-3.95 0-0.87 0.31-1.59 0.82-2.15-0.08-0.2-0.36-1.02 0.08-2.12 0 0 0.67-0.21 2.2 0.82 0.64-0.18 1.32-0.27 2-0.27 0.68 0 1.36 0.09 2 0.27 1.53-1.04 2.2-0.82 2.2-0.82 0.44 1.1 0.16 1.92 0.08 2.12 0.51 0.56 0.82 1.27 0.82 2.15 0 3.07-1.87 3.75-3.65 3.95 0.29 0.25 0.54 0.73 0.54 1.48 0 1.07-0.01 1.93-0.01 2.2 0 0.21 0.15 0.46 0.55 0.38C13.71 14.53 16 11.53 16 8 16 3.58 12.42 0 8 0z"></path></svg>
</a>


        <div class="header-search scoped-search site-scoped-search js-site-search" role="search">
  <!-- </textarea> --><!-- '"` --><form accept-charset="UTF-8" action="/alysondp/INE6006/search" class="js-site-search-form" data-scoped-search-url="/alysondp/INE6006/search" data-unscoped-search-url="/search" method="get"><div style="margin:0;padding:0;display:inline"><input name="utf8" type="hidden" value="&#x2713;" /></div>
    <label class="form-control header-search-wrapper js-chromeless-input-container">
      <div class="header-search-scope">This repository</div>
      <input type="text"
        class="form-control header-search-input js-site-search-focus js-site-search-field is-clearable"
        data-hotkey="s"
        name="q"
        placeholder="Search"
        aria-label="Search this repository"
        data-unscoped-placeholder="Search GitHub"
        data-scoped-placeholder="Search"
        tabindex="1"
        autocapitalize="off">
    </label>
</form></div>


      <ul class="header-nav left" role="navigation">
        <li class="header-nav-item">
          <a href="/pulls" class="js-selected-navigation-item header-nav-link" data-ga-click="Header, click, Nav menu - item:pulls context:user" data-hotkey="g p" data-selected-links="/pulls /pulls/assigned /pulls/mentioned /pulls">
            Pull requests
</a>        </li>
        <li class="header-nav-item">
          <a href="/issues" class="js-selected-navigation-item header-nav-link" data-ga-click="Header, click, Nav menu - item:issues context:user" data-hotkey="g i" data-selected-links="/issues /issues/assigned /issues/mentioned /issues">
            Issues
</a>        </li>
          <li class="header-nav-item">
            <a class="header-nav-link" href="https://gist.github.com/" data-ga-click="Header, go to gist, text:gist">Gist</a>
          </li>
      </ul>

    
<ul class="header-nav user-nav right" id="user-links">
  <li class="header-nav-item">
    
    <a href="/notifications" aria-label="You have unread notifications" class="header-nav-link notification-indicator tooltipped tooltipped-s js-socket-channel js-notification-indicator" data-channel="notification-changed-v2:4824231" data-ga-click="Header, go to notifications, icon:unread" data-hotkey="g n">
        <span class="mail-status unread"></span>
        <svg aria-hidden="true" class="octicon octicon-bell" height="16" version="1.1" viewBox="0 0 14 16" width="14"><path d="M14 12v1H0v-1l0.73-0.58c0.77-0.77 0.81-2.55 1.19-4.42 0.77-3.77 4.08-5 4.08-5 0-0.55 0.45-1 1-1s1 0.45 1 1c0 0 3.39 1.23 4.16 5 0.38 1.88 0.42 3.66 1.19 4.42l0.66 0.58z m-7 4c1.11 0 2-0.89 2-2H5c0 1.11 0.89 2 2 2z"></path></svg>
</a>
  </li>

  <li class="header-nav-item dropdown js-menu-container">
    <a class="header-nav-link tooltipped tooltipped-s js-menu-target" href="/new"
       aria-label="Create new…"
       data-ga-click="Header, create new, icon:add">
      <svg aria-hidden="true" class="octicon octicon-plus left" height="16" version="1.1" viewBox="0 0 12 16" width="12"><path d="M12 9H7v5H5V9H0V7h5V2h2v5h5v2z"></path></svg>
      <span class="dropdown-caret"></span>
    </a>

    <div class="dropdown-menu-content js-menu-content">
      <ul class="dropdown-menu dropdown-menu-sw">
        
<a class="dropdown-item" href="/new" data-ga-click="Header, create new repository">
  New repository
</a>

  <a class="dropdown-item" href="/new/import" data-ga-click="Header, import a repository">
    Import repository
  </a>


  <a class="dropdown-item" href="/organizations/new" data-ga-click="Header, create new organization">
    New organization
  </a>



  <div class="dropdown-divider"></div>
  <div class="dropdown-header">
    <span title="alysondp/INE6006">This repository</span>
  </div>
    <a class="dropdown-item" href="/alysondp/INE6006/issues/new" data-ga-click="Header, create new issue">
      New issue
    </a>
    <a class="dropdown-item" href="/alysondp/INE6006/settings/collaboration" data-ga-click="Header, create new collaborator">
      New collaborator
    </a>

      </ul>
    </div>
  </li>

  <li class="header-nav-item dropdown js-menu-container">
    <a class="header-nav-link name tooltipped tooltipped-sw js-menu-target" href="/alysondp"
       aria-label="View profile and more"
       data-ga-click="Header, show menu, icon:avatar">
      <img alt="@alysondp" class="avatar" height="20" src="https://avatars0.githubusercontent.com/u/4824231?v=3&amp;s=40" width="20" />
      <span class="dropdown-caret"></span>
    </a>

    <div class="dropdown-menu-content js-menu-content">
      <div class="dropdown-menu  dropdown-menu-sw">
        <div class=" dropdown-header header-nav-current-user css-truncate">
            Signed in as <strong class="css-truncate-target">alysondp</strong>

        </div>


        <div class="dropdown-divider"></div>

          <a class="dropdown-item" href="/alysondp" data-ga-click="Header, go to profile, text:your profile">
            Your profile
          </a>
        <a class="dropdown-item" href="/stars" data-ga-click="Header, go to starred repos, text:your stars">
          Your stars
        </a>
          <a class="dropdown-item" href="/explore" data-ga-click="Header, go to explore, text:explore">
            Explore
          </a>
          <a class="dropdown-item" href="/integrations" data-ga-click="Header, go to integrations, text:integrations">
            Integrations
          </a>
        <a class="dropdown-item" href="https://help.github.com" data-ga-click="Header, go to help, text:help">
          Help
        </a>


          <div class="dropdown-divider"></div>

          <a class="dropdown-item" href="/settings/profile" data-ga-click="Header, go to settings, icon:settings">
            Settings
          </a>

          <!-- </textarea> --><!-- '"` --><form accept-charset="UTF-8" action="/logout" class="logout-form" data-form-nonce="91c711a2eaac4347558f9765b468eb891648565f" method="post"><div style="margin:0;padding:0;display:inline"><input name="utf8" type="hidden" value="&#x2713;" /><input name="authenticity_token" type="hidden" value="jIAVoWYORXFIH9YPFeH3P5Gu9hzPbOF7xBpzfEf4CmZ6o2WMklD0zKGsiTypoHVAcyXqOc0CprOOIc8y+byA6w==" /></div>
            <button class="dropdown-item dropdown-signout" data-ga-click="Header, sign out, icon:logout">
              Sign out
            </button>
</form>
      </div>
    </div>
  </li>
</ul>


    
  </div>
</div>


      


    <div id="start-of-content" class="accessibility-aid"></div>

      <div id="js-flash-container">
</div>


    <div role="main" class="main-content">
        <div itemscope itemtype="http://schema.org/SoftwareSourceCode">
    <div id="js-repo-pjax-container" data-pjax-container>
      
<div class="pagehead repohead instapaper_ignore readability-menu experiment-repo-nav">
  <div class="container repohead-details-container">

    

<ul class="pagehead-actions">

  <li>
        <!-- </textarea> --><!-- '"` --><form accept-charset="UTF-8" action="/notifications/subscribe" class="js-social-container" data-autosubmit="true" data-form-nonce="91c711a2eaac4347558f9765b468eb891648565f" data-remote="true" method="post"><div style="margin:0;padding:0;display:inline"><input name="utf8" type="hidden" value="&#x2713;" /><input name="authenticity_token" type="hidden" value="EX84Pi95qNfXhN2c2D1OOhX/ioLEhUMI/v7HHD557g8F7TmFz8GNnKtPdCisUOyWy7DZLGERo0i/SlxdTyBO3Q==" /></div>      <input class="form-control" id="repository_id" name="repository_id" type="hidden" value="55005051" />

        <div class="select-menu js-menu-container js-select-menu">
          <a href="/alysondp/INE6006/subscription"
            class="btn btn-sm btn-with-count select-menu-button js-menu-target" role="button" tabindex="0" aria-haspopup="true"
            data-ga-click="Repository, click Watch settings, action:blob#show">
            <span class="js-select-button">
              <svg aria-hidden="true" class="octicon octicon-eye" height="16" version="1.1" viewBox="0 0 16 16" width="16"><path d="M8.06 2C3 2 0 8 0 8s3 6 8.06 6c4.94 0 7.94-6 7.94-6S13 2 8.06 2z m-0.06 10c-2.2 0-4-1.78-4-4 0-2.2 1.8-4 4-4 2.22 0 4 1.8 4 4 0 2.22-1.78 4-4 4z m2-4c0 1.11-0.89 2-2 2s-2-0.89-2-2 0.89-2 2-2 2 0.89 2 2z"></path></svg>
              Unwatch
            </span>
          </a>
          <a class="social-count js-social-count" href="/alysondp/INE6006/watchers">
            4
          </a>

        <div class="select-menu-modal-holder">
          <div class="select-menu-modal subscription-menu-modal js-menu-content" aria-hidden="true">
            <div class="select-menu-header js-navigation-enable" tabindex="-1">
              <svg aria-label="Close" class="octicon octicon-x js-menu-close" height="16" role="img" version="1.1" viewBox="0 0 12 16" width="12"><path d="M7.48 8l3.75 3.75-1.48 1.48-3.75-3.75-3.75 3.75-1.48-1.48 3.75-3.75L0.77 4.25l1.48-1.48 3.75 3.75 3.75-3.75 1.48 1.48-3.75 3.75z"></path></svg>
              <span class="select-menu-title">Notifications</span>
            </div>

              <div class="select-menu-list js-navigation-container" role="menu">

                <div class="select-menu-item js-navigation-item " role="menuitem" tabindex="0">
                  <svg aria-hidden="true" class="octicon octicon-check select-menu-item-icon" height="16" version="1.1" viewBox="0 0 12 16" width="12"><path d="M12 5L4 13 0 9l1.5-1.5 2.5 2.5 6.5-6.5 1.5 1.5z"></path></svg>
                  <div class="select-menu-item-text">
                    <input id="do_included" name="do" type="radio" value="included" />
                    <span class="select-menu-item-heading">Not watching</span>
                    <span class="description">Be notified when participating or @mentioned.</span>
                    <span class="js-select-button-text hidden-select-button-text">
                      <svg aria-hidden="true" class="octicon octicon-eye" height="16" version="1.1" viewBox="0 0 16 16" width="16"><path d="M8.06 2C3 2 0 8 0 8s3 6 8.06 6c4.94 0 7.94-6 7.94-6S13 2 8.06 2z m-0.06 10c-2.2 0-4-1.78-4-4 0-2.2 1.8-4 4-4 2.22 0 4 1.8 4 4 0 2.22-1.78 4-4 4z m2-4c0 1.11-0.89 2-2 2s-2-0.89-2-2 0.89-2 2-2 2 0.89 2 2z"></path></svg>
                      Watch
                    </span>
                  </div>
                </div>

                <div class="select-menu-item js-navigation-item selected" role="menuitem" tabindex="0">
                  <svg aria-hidden="true" class="octicon octicon-check select-menu-item-icon" height="16" version="1.1" viewBox="0 0 12 16" width="12"><path d="M12 5L4 13 0 9l1.5-1.5 2.5 2.5 6.5-6.5 1.5 1.5z"></path></svg>
                  <div class="select-menu-item-text">
                    <input checked="checked" id="do_subscribed" name="do" type="radio" value="subscribed" />
                    <span class="select-menu-item-heading">Watching</span>
                    <span class="description">Be notified of all conversations.</span>
                    <span class="js-select-button-text hidden-select-button-text">
                      <svg aria-hidden="true" class="octicon octicon-eye" height="16" version="1.1" viewBox="0 0 16 16" width="16"><path d="M8.06 2C3 2 0 8 0 8s3 6 8.06 6c4.94 0 7.94-6 7.94-6S13 2 8.06 2z m-0.06 10c-2.2 0-4-1.78-4-4 0-2.2 1.8-4 4-4 2.22 0 4 1.8 4 4 0 2.22-1.78 4-4 4z m2-4c0 1.11-0.89 2-2 2s-2-0.89-2-2 0.89-2 2-2 2 0.89 2 2z"></path></svg>
                      Unwatch
                    </span>
                  </div>
                </div>

                <div class="select-menu-item js-navigation-item " role="menuitem" tabindex="0">
                  <svg aria-hidden="true" class="octicon octicon-check select-menu-item-icon" height="16" version="1.1" viewBox="0 0 12 16" width="12"><path d="M12 5L4 13 0 9l1.5-1.5 2.5 2.5 6.5-6.5 1.5 1.5z"></path></svg>
                  <div class="select-menu-item-text">
                    <input id="do_ignore" name="do" type="radio" value="ignore" />
                    <span class="select-menu-item-heading">Ignoring</span>
                    <span class="description">Never be notified.</span>
                    <span class="js-select-button-text hidden-select-button-text">
                      <svg aria-hidden="true" class="octicon octicon-mute" height="16" version="1.1" viewBox="0 0 16 16" width="16"><path d="M8 2.81v10.38c0 0.67-0.81 1-1.28 0.53L3 10H1c-0.55 0-1-0.45-1-1V7c0-0.55 0.45-1 1-1h2l3.72-3.72c0.47-0.47 1.28-0.14 1.28 0.53z m7.53 3.22l-1.06-1.06-1.97 1.97-1.97-1.97-1.06 1.06 1.97 1.97-1.97 1.97 1.06 1.06 1.97-1.97 1.97 1.97 1.06-1.06-1.97-1.97 1.97-1.97z"></path></svg>
                      Stop ignoring
                    </span>
                  </div>
                </div>

              </div>

            </div>
          </div>
        </div>
</form>
  </li>

  <li>
    
  <div class="js-toggler-container js-social-container starring-container ">

    <!-- </textarea> --><!-- '"` --><form accept-charset="UTF-8" action="/alysondp/INE6006/unstar" class="starred" data-form-nonce="91c711a2eaac4347558f9765b468eb891648565f" data-remote="true" method="post"><div style="margin:0;padding:0;display:inline"><input name="utf8" type="hidden" value="&#x2713;" /><input name="authenticity_token" type="hidden" value="K9RhkzX3z9L1Wn++S3fZQPX44wgxG1HYV8bFNoyrp0zZkHc5yhPNu4/rbCn3Ipmthu7s7b2Ky3mCSf1gney3UQ==" /></div>
      <button
        class="btn btn-sm btn-with-count js-toggler-target"
        aria-label="Unstar this repository" title="Unstar alysondp/INE6006"
        data-ga-click="Repository, click unstar button, action:blob#show; text:Unstar">
        <svg aria-hidden="true" class="octicon octicon-star" height="16" version="1.1" viewBox="0 0 14 16" width="14"><path d="M14 6l-4.9-0.64L7 1 4.9 5.36 0 6l3.6 3.26L2.67 14l4.33-2.33 4.33 2.33L10.4 9.26 14 6z"></path></svg>
        Unstar
      </button>
        <a class="social-count js-social-count" href="/alysondp/INE6006/stargazers">
          0
        </a>
</form>
    <!-- </textarea> --><!-- '"` --><form accept-charset="UTF-8" action="/alysondp/INE6006/star" class="unstarred" data-form-nonce="91c711a2eaac4347558f9765b468eb891648565f" data-remote="true" method="post"><div style="margin:0;padding:0;display:inline"><input name="utf8" type="hidden" value="&#x2713;" /><input name="authenticity_token" type="hidden" value="FtBTtL9YhrmhQKBAR+HlfkHd1efdszNu9kwUuolExRTOQ8nAro5Klxw3AUXn5YXQiKNT853mHJCgeef0WT9E7A==" /></div>
      <button
        class="btn btn-sm btn-with-count js-toggler-target"
        aria-label="Star this repository" title="Star alysondp/INE6006"
        data-ga-click="Repository, click star button, action:blob#show; text:Star">
        <svg aria-hidden="true" class="octicon octicon-star" height="16" version="1.1" viewBox="0 0 14 16" width="14"><path d="M14 6l-4.9-0.64L7 1 4.9 5.36 0 6l3.6 3.26L2.67 14l4.33-2.33 4.33 2.33L10.4 9.26 14 6z"></path></svg>
        Star
      </button>
        <a class="social-count js-social-count" href="/alysondp/INE6006/stargazers">
          0
        </a>
</form>  </div>

  </li>

  <li>
          <a href="#fork-destination-box" class="btn btn-sm btn-with-count"
              title="Fork your own copy of alysondp/INE6006 to your account"
              aria-label="Fork your own copy of alysondp/INE6006 to your account"
              rel="facebox"
              data-ga-click="Repository, show fork modal, action:blob#show; text:Fork">
              <svg aria-hidden="true" class="octicon octicon-repo-forked" height="16" version="1.1" viewBox="0 0 10 16" width="10"><path d="M8 1c-1.11 0-2 0.89-2 2 0 0.73 0.41 1.38 1 1.72v1.28L5 8 3 6v-1.28c0.59-0.34 1-0.98 1-1.72 0-1.11-0.89-2-2-2S0 1.89 0 3c0 0.73 0.41 1.38 1 1.72v1.78l3 3v1.78c-0.59 0.34-1 0.98-1 1.72 0 1.11 0.89 2 2 2s2-0.89 2-2c0-0.73-0.41-1.38-1-1.72V9.5l3-3V4.72c0.59-0.34 1-0.98 1-1.72 0-1.11-0.89-2-2-2zM2 4.2c-0.66 0-1.2-0.55-1.2-1.2s0.55-1.2 1.2-1.2 1.2 0.55 1.2 1.2-0.55 1.2-1.2 1.2z m3 10c-0.66 0-1.2-0.55-1.2-1.2s0.55-1.2 1.2-1.2 1.2 0.55 1.2 1.2-0.55 1.2-1.2 1.2z m3-10c-0.66 0-1.2-0.55-1.2-1.2s0.55-1.2 1.2-1.2 1.2 0.55 1.2 1.2-0.55 1.2-1.2 1.2z"></path></svg>
            Fork
          </a>

          <div id="fork-destination-box" style="display: none;">
            <h2 class="facebox-header" data-facebox-id="facebox-header">Where should we fork this repository?</h2>
            <include-fragment src=""
                class="js-fork-select-fragment fork-select-fragment"
                data-url="/alysondp/INE6006/fork?fragment=1">
              <img alt="Loading" height="64" src="https://assets-cdn.github.com/images/spinners/octocat-spinner-128.gif" width="64" />
            </include-fragment>
          </div>

    <a href="/alysondp/INE6006/network" class="social-count">
      0
    </a>
  </li>
</ul>

    <h1 class="entry-title public ">
  <svg aria-hidden="true" class="octicon octicon-repo" height="16" version="1.1" viewBox="0 0 12 16" width="12"><path d="M4 9h-1v-1h1v1z m0-3h-1v1h1v-1z m0-2h-1v1h1v-1z m0-2h-1v1h1v-1z m8-1v12c0 0.55-0.45 1-1 1H6v2l-1.5-1.5-1.5 1.5V14H1c-0.55 0-1-0.45-1-1V1C0 0.45 0.45 0 1 0h10c0.55 0 1 0.45 1 1z m-1 10H1v2h2v-1h3v1h5V11z m0-10H2v9h9V1z"></path></svg>
  <span class="author" itemprop="author"><a href="/alysondp" class="url fn" rel="author">alysondp</a></span><!--
--><span class="path-divider">/</span><!--
--><strong itemprop="name"><a href="/alysondp/INE6006" data-pjax="#js-repo-pjax-container">INE6006</a></strong>

</h1>

  </div>
  <div class="container">
    
<nav class="reponav js-repo-nav js-sidenav-container-pjax"
     itemscope
     itemtype="http://schema.org/BreadcrumbList"
     role="navigation"
     data-pjax="#js-repo-pjax-container">

  <span itemscope itemtype="http://schema.org/ListItem" itemprop="itemListElement">
    <a href="/alysondp/INE6006" aria-selected="true" class="js-selected-navigation-item selected reponav-item" data-hotkey="g c" data-selected-links="repo_source repo_downloads repo_commits repo_releases repo_tags repo_branches /alysondp/INE6006" itemprop="url">
      <svg aria-hidden="true" class="octicon octicon-code" height="16" version="1.1" viewBox="0 0 14 16" width="14"><path d="M9.5 3l-1.5 1.5 3.5 3.5L8 11.5l1.5 1.5 4.5-5L9.5 3zM4.5 3L0 8l4.5 5 1.5-1.5L2.5 8l3.5-3.5L4.5 3z"></path></svg>
      <span itemprop="name">Code</span>
      <meta itemprop="position" content="1">
</a>  </span>

    <span itemscope itemtype="http://schema.org/ListItem" itemprop="itemListElement">
      <a href="/alysondp/INE6006/issues" class="js-selected-navigation-item reponav-item" data-hotkey="g i" data-selected-links="repo_issues repo_labels repo_milestones /alysondp/INE6006/issues" itemprop="url">
        <svg aria-hidden="true" class="octicon octicon-issue-opened" height="16" version="1.1" viewBox="0 0 14 16" width="14"><path d="M7 2.3c3.14 0 5.7 2.56 5.7 5.7S10.14 13.7 7 13.7 1.3 11.14 1.3 8s2.56-5.7 5.7-5.7m0-1.3C3.14 1 0 4.14 0 8s3.14 7 7 7 7-3.14 7-7S10.86 1 7 1z m1 3H6v5h2V4z m0 6H6v2h2V10z"></path></svg>
        <span itemprop="name">Issues</span>
        <span class="counter">0</span>
        <meta itemprop="position" content="2">
</a>    </span>

  <span itemscope itemtype="http://schema.org/ListItem" itemprop="itemListElement">
    <a href="/alysondp/INE6006/pulls" class="js-selected-navigation-item reponav-item" data-hotkey="g p" data-selected-links="repo_pulls /alysondp/INE6006/pulls" itemprop="url">
      <svg aria-hidden="true" class="octicon octicon-git-pull-request" height="16" version="1.1" viewBox="0 0 12 16" width="12"><path d="M11 11.28c0-1.73 0-6.28 0-6.28-0.03-0.78-0.34-1.47-0.94-2.06s-1.28-0.91-2.06-0.94c0 0-1.02 0-1 0V0L4 3l3 3V4h1c0.27 0.02 0.48 0.11 0.69 0.31s0.3 0.42 0.31 0.69v6.28c-0.59 0.34-1 0.98-1 1.72 0 1.11 0.89 2 2 2s2-0.89 2-2c0-0.73-0.41-1.38-1-1.72z m-1 2.92c-0.66 0-1.2-0.55-1.2-1.2s0.55-1.2 1.2-1.2 1.2 0.55 1.2 1.2-0.55 1.2-1.2 1.2zM4 3c0-1.11-0.89-2-2-2S0 1.89 0 3c0 0.73 0.41 1.38 1 1.72 0 1.55 0 5.56 0 6.56-0.59 0.34-1 0.98-1 1.72 0 1.11 0.89 2 2 2s2-0.89 2-2c0-0.73-0.41-1.38-1-1.72V4.72c0.59-0.34 1-0.98 1-1.72z m-0.8 10c0 0.66-0.55 1.2-1.2 1.2s-1.2-0.55-1.2-1.2 0.55-1.2 1.2-1.2 1.2 0.55 1.2 1.2z m-1.2-8.8c-0.66 0-1.2-0.55-1.2-1.2s0.55-1.2 1.2-1.2 1.2 0.55 1.2 1.2-0.55 1.2-1.2 1.2z"></path></svg>
      <span itemprop="name">Pull requests</span>
      <span class="counter">0</span>
      <meta itemprop="position" content="3">
</a>  </span>

    <a href="/alysondp/INE6006/wiki" class="js-selected-navigation-item reponav-item" data-hotkey="g w" data-selected-links="repo_wiki /alysondp/INE6006/wiki">
      <svg aria-hidden="true" class="octicon octicon-book" height="16" version="1.1" viewBox="0 0 16 16" width="16"><path d="M2 5h4v1H2v-1z m0 3h4v-1H2v1z m0 2h4v-1H2v1z m11-5H9v1h4v-1z m0 2H9v1h4v-1z m0 2H9v1h4v-1z m2-6v9c0 0.55-0.45 1-1 1H8.5l-1 1-1-1H1c-0.55 0-1-0.45-1-1V3c0-0.55 0.45-1 1-1h5.5l1 1 1-1h5.5c0.55 0 1 0.45 1 1z m-8 0.5l-0.5-0.5H1v9h6V3.5z m7-0.5H8.5l-0.5 0.5v8.5h6V3z"></path></svg>
      Wiki
</a>

  <a href="/alysondp/INE6006/pulse" class="js-selected-navigation-item reponav-item" data-selected-links="pulse /alysondp/INE6006/pulse">
    <svg aria-hidden="true" class="octicon octicon-pulse" height="16" version="1.1" viewBox="0 0 14 16" width="14"><path d="M11.5 8L8.8 5.4 6.6 8.5 5.5 1.6 2.38 8H0V10h3.6L4.5 8.2l0.9 5.4L9 8.5l1.6 1.5H14V8H11.5z"></path></svg>
    Pulse
</a>
  <a href="/alysondp/INE6006/graphs" class="js-selected-navigation-item reponav-item" data-selected-links="repo_graphs repo_contributors /alysondp/INE6006/graphs">
    <svg aria-hidden="true" class="octicon octicon-graph" height="16" version="1.1" viewBox="0 0 16 16" width="16"><path d="M16 14v1H0V0h1v14h15z m-11-1H3V8h2v5z m4 0H7V3h2v10z m4 0H11V6h2v7z"></path></svg>
    Graphs
</a>
    <a href="/alysondp/INE6006/settings" class="js-selected-navigation-item reponav-item" data-selected-links="repo_settings repo_branch_settings hooks /alysondp/INE6006/settings">
      <svg aria-hidden="true" class="octicon octicon-gear" height="16" version="1.1" viewBox="0 0 14 16" width="14"><path d="M14 8.77V7.17l-1.94-0.64-0.45-1.09 0.88-1.84-1.13-1.13-1.81 0.91-1.09-0.45-0.69-1.92H6.17l-0.63 1.94-1.11 0.45-1.84-0.88-1.13 1.13 0.91 1.81-0.45 1.09L0 7.23v1.59l1.94 0.64 0.45 1.09-0.88 1.84 1.13 1.13 1.81-0.91 1.09 0.45 0.69 1.92h1.59l0.63-1.94 1.11-0.45 1.84 0.88 1.13-1.13-0.92-1.81 0.47-1.09 1.92-0.69zM7 11c-1.66 0-3-1.34-3-3s1.34-3 3-3 3 1.34 3 3-1.34 3-3 3z"></path></svg>
      Settings
</a>
</nav>

  </div>
</div>

<div class="container new-discussion-timeline experiment-repo-nav">
  <div class="repository-content">

    

<a href="/alysondp/INE6006/blob/2459c729da16eb34aa4fa8d215fbdfd667769bf4/lista2/questao4/questao4.tex" class="hidden js-permalink-shortcut" data-hotkey="y">Permalink</a>

<!-- blob contrib key: blob_contributors:v21:ecdc39db196c881b4bcb096d29bf1cec -->

<div class="file-navigation js-zeroclipboard-container">
  
<div class="select-menu branch-select-menu js-menu-container js-select-menu left">
  <button class="btn btn-sm select-menu-button js-menu-target css-truncate" data-hotkey="w"
    title="master"
    type="button" aria-label="Switch branches or tags" tabindex="0" aria-haspopup="true">
    <i>Branch:</i>
    <span class="js-select-button css-truncate-target">master</span>
  </button>

  <div class="select-menu-modal-holder js-menu-content js-navigation-container" data-pjax aria-hidden="true">

    <div class="select-menu-modal">
      <div class="select-menu-header">
        <svg aria-label="Close" class="octicon octicon-x js-menu-close" height="16" role="img" version="1.1" viewBox="0 0 12 16" width="12"><path d="M7.48 8l3.75 3.75-1.48 1.48-3.75-3.75-3.75 3.75-1.48-1.48 3.75-3.75L0.77 4.25l1.48-1.48 3.75 3.75 3.75-3.75 1.48 1.48-3.75 3.75z"></path></svg>
        <span class="select-menu-title">Switch branches/tags</span>
      </div>

      <div class="select-menu-filters">
        <div class="select-menu-text-filter">
          <input type="text" aria-label="Find or create a branch…" id="context-commitish-filter-field" class="form-control js-filterable-field js-navigation-enable" placeholder="Find or create a branch…">
        </div>
        <div class="select-menu-tabs">
          <ul>
            <li class="select-menu-tab">
              <a href="#" data-tab-filter="branches" data-filter-placeholder="Find or create a branch…" class="js-select-menu-tab" role="tab">Branches</a>
            </li>
            <li class="select-menu-tab">
              <a href="#" data-tab-filter="tags" data-filter-placeholder="Find a tag…" class="js-select-menu-tab" role="tab">Tags</a>
            </li>
          </ul>
        </div>
      </div>

      <div class="select-menu-list select-menu-tab-bucket js-select-menu-tab-bucket" data-tab-filter="branches" role="menu">

        <div data-filterable-for="context-commitish-filter-field" data-filterable-type="substring">


            <a class="select-menu-item js-navigation-item js-navigation-open selected"
               href="/alysondp/INE6006/blob/master/lista2/questao4/questao4.tex"
               data-name="master"
               data-skip-pjax="true"
               rel="nofollow">
              <svg aria-hidden="true" class="octicon octicon-check select-menu-item-icon" height="16" version="1.1" viewBox="0 0 12 16" width="12"><path d="M12 5L4 13 0 9l1.5-1.5 2.5 2.5 6.5-6.5 1.5 1.5z"></path></svg>
              <span class="select-menu-item-text css-truncate-target js-select-menu-filter-text" title="master">
                master
              </span>
            </a>
        </div>

          <!-- </textarea> --><!-- '"` --><form accept-charset="UTF-8" action="/alysondp/INE6006/branches" class="js-create-branch select-menu-item select-menu-new-item-form js-navigation-item js-new-item-form" data-form-nonce="91c711a2eaac4347558f9765b468eb891648565f" method="post"><div style="margin:0;padding:0;display:inline"><input name="utf8" type="hidden" value="&#x2713;" /><input name="authenticity_token" type="hidden" value="1ry3wNGKg9QSk4MwrKN6oUffHhMNgrFKO7VI57uKPApyvzkt7VgSW6LyokECWFyZ2w+6bsRNrlhZ5fm9MSbxig==" /></div>
          <svg aria-hidden="true" class="octicon octicon-git-branch select-menu-item-icon" height="16" version="1.1" viewBox="0 0 10 16" width="10"><path d="M10 5c0-1.11-0.89-2-2-2s-2 0.89-2 2c0 0.73 0.41 1.38 1 1.72v0.3c-0.02 0.52-0.23 0.98-0.63 1.38s-0.86 0.61-1.38 0.63c-0.83 0.02-1.48 0.16-2 0.45V4.72c0.59-0.34 1-0.98 1-1.72 0-1.11-0.89-2-2-2S0 1.89 0 3c0 0.73 0.41 1.38 1 1.72v6.56C0.41 11.63 0 12.27 0 13c0 1.11 0.89 2 2 2s2-0.89 2-2c0-0.53-0.2-1-0.53-1.36 0.09-0.06 0.48-0.41 0.59-0.47 0.25-0.11 0.56-0.17 0.94-0.17 1.05-0.05 1.95-0.45 2.75-1.25s1.2-1.98 1.25-3.02h-0.02c0.61-0.36 1.02-1 1.02-1.73zM2 1.8c0.66 0 1.2 0.55 1.2 1.2s-0.55 1.2-1.2 1.2-1.2-0.55-1.2-1.2 0.55-1.2 1.2-1.2z m0 12.41c-0.66 0-1.2-0.55-1.2-1.2s0.55-1.2 1.2-1.2 1.2 0.55 1.2 1.2-0.55 1.2-1.2 1.2z m6-8c-0.66 0-1.2-0.55-1.2-1.2s0.55-1.2 1.2-1.2 1.2 0.55 1.2 1.2-0.55 1.2-1.2 1.2z"></path></svg>
            <div class="select-menu-item-text">
              <span class="select-menu-item-heading">Create branch: <span class="js-new-item-name"></span></span>
              <span class="description">from ‘master’</span>
            </div>
            <input type="hidden" name="name" id="name" class="js-new-item-value">
            <input type="hidden" name="branch" id="branch" value="master">
            <input type="hidden" name="path" id="path" value="lista2/questao4/questao4.tex">
</form>
      </div>

      <div class="select-menu-list select-menu-tab-bucket js-select-menu-tab-bucket" data-tab-filter="tags">
        <div data-filterable-for="context-commitish-filter-field" data-filterable-type="substring">


        </div>

        <div class="select-menu-no-results">Nothing to show</div>
      </div>

    </div>
  </div>
</div>

  <div class="btn-group right">
    <a href="/alysondp/INE6006/find/master"
          class="js-pjax-capture-input btn btn-sm"
          data-pjax
          data-hotkey="t">
      Find file
    </a>
    <button aria-label="Copy file path to clipboard" class="js-zeroclipboard btn btn-sm zeroclipboard-button tooltipped tooltipped-s" data-copied-hint="Copied!" type="button">Copy path</button>
  </div>
  <div class="breadcrumb js-zeroclipboard-target">
    <span class="repo-root js-repo-root"><span class="js-path-segment"><a href="/alysondp/INE6006"><span>INE6006</span></a></span></span><span class="separator">/</span><span class="js-path-segment"><a href="/alysondp/INE6006/tree/master/lista2"><span>lista2</span></a></span><span class="separator">/</span><span class="js-path-segment"><a href="/alysondp/INE6006/tree/master/lista2/questao4"><span>questao4</span></a></span><span class="separator">/</span><strong class="final-path">questao4.tex</strong>
  </div>
</div>


  <div class="commit-tease">
      <span class="right">
        <a class="commit-tease-sha" href="/alysondp/INE6006/commit/86145ac2b873200b32c74b98af5a57e31619f559" data-pjax>
          86145ac
        </a>
        <relative-time datetime="2016-05-10T15:08:55Z">May 10, 2016</relative-time>
      </span>
      <div>
        <img alt="@alexishuf" class="avatar" height="20" src="https://avatars3.githubusercontent.com/u/11590809?v=3&amp;s=40" width="20" />
        <a href="/alexishuf" class="user-mention" rel="contributor">alexishuf</a>
          <a href="/alysondp/INE6006/commit/86145ac2b873200b32c74b98af5a57e31619f559" class="message" data-pjax="true" title="questão 4.c clarificada">questão 4.c clarificada</a>
      </div>

    <div class="commit-tease-contributors">
      <button type="button" class="btn-link muted-link contributors-toggle" data-facebox="#blob_contributors_box">
        <strong>1</strong>
         contributor
      </button>
      
    </div>

    <div id="blob_contributors_box" style="display:none">
      <h2 class="facebox-header" data-facebox-id="facebox-header">Users who have contributed to this file</h2>
      <ul class="facebox-user-list" data-facebox-id="facebox-description">
          <li class="facebox-user-list-item">
            <img alt="@alexishuf" height="24" src="https://avatars1.githubusercontent.com/u/11590809?v=3&amp;s=48" width="24" />
            <a href="/alexishuf">alexishuf</a>
          </li>
      </ul>
    </div>
  </div>

<div class="file">
  <div class="file-header">
  <div class="file-actions">

    <div class="btn-group">
      <a href="/alysondp/INE6006/raw/master/lista2/questao4/questao4.tex" class="btn btn-sm " id="raw-url">Raw</a>
        <a href="/alysondp/INE6006/blame/master/lista2/questao4/questao4.tex" class="btn btn-sm js-update-url-with-hash">Blame</a>
      <a href="/alysondp/INE6006/commits/master/lista2/questao4/questao4.tex" class="btn btn-sm " rel="nofollow">History</a>
    </div>


        <!-- </textarea> --><!-- '"` --><form accept-charset="UTF-8" action="/alysondp/INE6006/edit/master/lista2/questao4/questao4.tex" class="inline-form js-update-url-with-hash" data-form-nonce="91c711a2eaac4347558f9765b468eb891648565f" method="post"><div style="margin:0;padding:0;display:inline"><input name="utf8" type="hidden" value="&#x2713;" /><input name="authenticity_token" type="hidden" value="R/1NQ97TDKeFeI9uCRDOreBFHM/Fv+X2+/rN64w9WEz79LcOP66Pn4MzWxTdkYUc4ZwBJGfEfuQxKlWY3DZIFg==" /></div>
          <button class="btn-octicon tooltipped tooltipped-nw" type="submit"
            aria-label="Edit this file" data-hotkey="e" data-disable-with>
            <svg aria-hidden="true" class="octicon octicon-pencil" height="16" version="1.1" viewBox="0 0 14 16" width="14"><path d="M0 12v3h3l8-8-3-3L0 12z m3 2H1V12h1v1h1v1z m10.3-9.3l-1.3 1.3-3-3 1.3-1.3c0.39-0.39 1.02-0.39 1.41 0l1.59 1.59c0.39 0.39 0.39 1.02 0 1.41z"></path></svg>
          </button>
</form>        <!-- </textarea> --><!-- '"` --><form accept-charset="UTF-8" action="/alysondp/INE6006/delete/master/lista2/questao4/questao4.tex" class="inline-form" data-form-nonce="91c711a2eaac4347558f9765b468eb891648565f" method="post"><div style="margin:0;padding:0;display:inline"><input name="utf8" type="hidden" value="&#x2713;" /><input name="authenticity_token" type="hidden" value="VZVVk9vPymb0dJ1uaWe4EsR7UwJ+L20fTIFuBq61YTDqCBGhHqb/IvSvsbR0IItEtG7DRwqauVjQ++Ftvfo44g==" /></div>
          <button class="btn-octicon btn-octicon-danger tooltipped tooltipped-nw" type="submit"
            aria-label="Delete this file" data-disable-with>
            <svg aria-hidden="true" class="octicon octicon-trashcan" height="16" version="1.1" viewBox="0 0 12 16" width="12"><path d="M10 2H8c0-0.55-0.45-1-1-1H4c-0.55 0-1 0.45-1 1H1c-0.55 0-1 0.45-1 1v1c0 0.55 0.45 1 1 1v9c0 0.55 0.45 1 1 1h7c0.55 0 1-0.45 1-1V5c0.55 0 1-0.45 1-1v-1c0-0.55-0.45-1-1-1z m-1 12H2V5h1v8h1V5h1v8h1V5h1v8h1V5h1v9z m1-10H1v-1h9v1z"></path></svg>
          </button>
</form>  </div>

  <div class="file-info">
      93 lines (71 sloc)
      <span class="file-info-divider"></span>
    5.09 KB
  </div>
</div>

  

  <div itemprop="text" class="blob-wrapper data type-tex">
      <table class="highlight tab-size js-file-line-container" data-tab-size="8">
      <tr>
        <td id="L1" class="blob-num js-line-number" data-line-number="1"></td>
        <td id="LC1" class="blob-code blob-code-inner js-file-line"><span class="pl-c1">\input</span>{questao4/vars}</td>
      </tr>
      <tr>
        <td id="L2" class="blob-num js-line-number" data-line-number="2"></td>
        <td id="LC2" class="blob-code blob-code-inner js-file-line">
</td>
      </tr>
      <tr>
        <td id="L3" class="blob-num js-line-number" data-line-number="3"></td>
        <td id="LC3" class="blob-code blob-code-inner js-file-line">Foi retirada uma amostra aleatória simples, sem reposição com <span class="pl-c1">\QUATROn</span> elementos, dentre os <span class="pl-c1">\QUATRON</span> alunos que possuíam um valor definido para a variável Pagamento. A proporção amostral <span class="pl-s"><span class="pl-pds">$</span><span class="pl-c1">\hat</span>{p}<span class="pl-pds">$</span></span> de alunos cuja fonte de Pagamento são ``Incentivos Federais&#39;&#39; foi de <span class="pl-c1">\QUATROpAmostral</span>.</td>
      </tr>
      <tr>
        <td id="L4" class="blob-num js-line-number" data-line-number="4"></td>
        <td id="LC4" class="blob-code blob-code-inner js-file-line">
</td>
      </tr>
      <tr>
        <td id="L5" class="blob-num js-line-number" data-line-number="5"></td>
        <td id="LC5" class="blob-code blob-code-inner js-file-line"><span class="pl-c1">\subsection</span>{Intervalo de confiança}</td>
      </tr>
      <tr>
        <td id="L6" class="blob-num js-line-number" data-line-number="6"></td>
        <td id="LC6" class="blob-code blob-code-inner js-file-line">O intervalo de confiança para a a estatística <span class="pl-s"><span class="pl-pds">$</span><span class="pl-c1">\hat</span>{p}<span class="pl-pds">$</span></span> é dado pela <span class="pl-c1">\autoref</span>{eq:quatro-a-expr}.</td>
      </tr>
      <tr>
        <td id="L7" class="blob-num js-line-number" data-line-number="7"></td>
        <td id="LC7" class="blob-code blob-code-inner js-file-line">
</td>
      </tr>
      <tr>
        <td id="L8" class="blob-num js-line-number" data-line-number="8"></td>
        <td id="LC8" class="blob-code blob-code-inner js-file-line"><span class="pl-c1">\begin</span>{align} </td>
      </tr>
      <tr>
        <td id="L9" class="blob-num js-line-number" data-line-number="9"></td>
        <td id="LC9" class="blob-code blob-code-inner js-file-line">	IC(p, 95<span class="pl-cce">\%</span>) <span class="pl-c1">\label</span>{eq:quatro-a-expr}</td>
      </tr>
      <tr>
        <td id="L10" class="blob-num js-line-number" data-line-number="10"></td>
        <td id="LC10" class="blob-code blob-code-inner js-file-line">	            &amp;= <span class="pl-c1">\hat</span>{p} <span class="pl-c1">\pm</span> z_<span class="pl-c1">\gamma</span> <span class="pl-c1">\sigma</span>_{<span class="pl-c1">\hat</span>{p}} <span class="pl-c1">\\</span></td>
      </tr>
      <tr>
        <td id="L11" class="blob-num js-line-number" data-line-number="11"></td>
        <td id="LC11" class="blob-code blob-code-inner js-file-line">	            <span class="pl-c1">\label</span>{eq:quatro-a-estim}</td>
      </tr>
      <tr>
        <td id="L12" class="blob-num js-line-number" data-line-number="12"></td>
        <td id="LC12" class="blob-code blob-code-inner js-file-line">	            &amp;= <span class="pl-c1">\hat</span>{p} <span class="pl-c1">\pm</span> z_<span class="pl-c1">\gamma</span> <span class="pl-c1">\sqrt</span>{<span class="pl-c1">\frac</span>{<span class="pl-c1">\hat</span>{p}(1-<span class="pl-c1">\hat</span>{p})}{n}} <span class="pl-c1">\\</span></td>
      </tr>
      <tr>
        <td id="L13" class="blob-num js-line-number" data-line-number="13"></td>
        <td id="LC13" class="blob-code blob-code-inner js-file-line">	            &amp;= <span class="pl-c1">\QUATROpAmostral</span> <span class="pl-c1">\pm</span> <span class="pl-c1">\QUATROzy</span> <span class="pl-c1">\sqrt</span>{<span class="pl-c1">\frac</span>{<span class="pl-c1">\QUATROpAmostral</span> (1- <span class="pl-c1">\QUATROpAmostral</span>)}{<span class="pl-c1">\QUATROn</span>}} <span class="pl-c1">\nonumber</span> <span class="pl-c1">\\</span></td>
      </tr>
      <tr>
        <td id="L14" class="blob-num js-line-number" data-line-number="14"></td>
        <td id="LC14" class="blob-code blob-code-inner js-file-line">	            &amp;= <span class="pl-c1">\QUATROpAmostral</span> <span class="pl-c1">\pm</span> <span class="pl-c1">\QUATROAdelta</span> <span class="pl-c1">\nonumber</span> <span class="pl-c1">\\</span></td>
      </tr>
      <tr>
        <td id="L15" class="blob-num js-line-number" data-line-number="15"></td>
        <td id="LC15" class="blob-code blob-code-inner js-file-line">	            <span class="pl-c1">\label</span>{eq:quatro-a-result}</td>
      </tr>
      <tr>
        <td id="L16" class="blob-num js-line-number" data-line-number="16"></td>
        <td id="LC16" class="blob-code blob-code-inner js-file-line">	            &amp;= [<span class="pl-c1">\QUATROAICinf</span>, <span class="pl-c1">\QUATROAICsup</span>]</td>
      </tr>
      <tr>
        <td id="L17" class="blob-num js-line-number" data-line-number="17"></td>
        <td id="LC17" class="blob-code blob-code-inner js-file-line"><span class="pl-c1">\end</span>{align}</td>
      </tr>
      <tr>
        <td id="L18" class="blob-num js-line-number" data-line-number="18"></td>
        <td id="LC18" class="blob-code blob-code-inner js-file-line">
</td>
      </tr>
      <tr>
        <td id="L19" class="blob-num js-line-number" data-line-number="19"></td>
        <td id="LC19" class="blob-code blob-code-inner js-file-line">Dada uma amostra aleatória simples de <span class="pl-c1">\QUATROn</span> alunos tomados dentre os <span class="pl-c1">\QUATRON</span> da população, há 95<span class="pl-cce">\%</span> de chance de que a proporção de alunos cuja fonte de recursos são ``Incentivos Federais&#39;&#39; seja um valor contido no intervalo <span class="pl-c1">\eqref</span>{eq:quatro-a-result}.</td>
      </tr>
      <tr>
        <td id="L20" class="blob-num js-line-number" data-line-number="20"></td>
        <td id="LC20" class="blob-code blob-code-inner js-file-line">
</td>
      </tr>
      <tr>
        <td id="L21" class="blob-num js-line-number" data-line-number="21"></td>
        <td id="LC21" class="blob-code blob-code-inner js-file-line">Em <span class="pl-c1">\eqref</span>{eq:quatro-a-expr}, <span class="pl-s"><span class="pl-pds">$</span><span class="pl-c1">\sigma</span>_{<span class="pl-c1">\hat</span>{p}}<span class="pl-pds">$</span></span> simboliza o desvio padrão de <span class="pl-s"><span class="pl-pds">$</span><span class="pl-c1">\hat</span>{p}<span class="pl-pds">$</span></span> em todas as amostras aleatórias que podem ser tomadas da população. Como não possuímos tal parâmetro, e como <span class="pl-s"><span class="pl-pds">$</span>n<span class="pl-pds">$</span></span> é grande (<span class="pl-s"><span class="pl-pds">$</span>n = <span class="pl-c1">\QUATROn</span> <span class="pl-c1">\geq</span> <span class="pl-c1">50</span><span class="pl-pds">$</span></span>), <span class="pl-s"><span class="pl-pds">$</span><span class="pl-c1">\sigma</span>_{<span class="pl-c1">\hat</span>{p}}<span class="pl-pds">$</span></span> foi aproximado por <span class="pl-s"><span class="pl-pds">$</span>s_{<span class="pl-c1">\hat</span>{p}}<span class="pl-pds">$</span></span>, o desvio padrão da amostra tomada.</td>
      </tr>
      <tr>
        <td id="L22" class="blob-num js-line-number" data-line-number="22"></td>
        <td id="LC22" class="blob-code blob-code-inner js-file-line">
</td>
      </tr>
      <tr>
        <td id="L23" class="blob-num js-line-number" data-line-number="23"></td>
        <td id="LC23" class="blob-code blob-code-inner js-file-line"><span class="pl-c1">\subsection</span>{Precisão}</td>
      </tr>
      <tr>
        <td id="L24" class="blob-num js-line-number" data-line-number="24"></td>
        <td id="LC24" class="blob-code blob-code-inner js-file-line">No contexto dessa questão, é possível calcular o tamanho da amostra necessária para que <span class="pl-s"><span class="pl-pds">$</span>|<span class="pl-c1">\hat</span>{p} - E(<span class="pl-c1">\hat</span>{p})| <span class="pl-c1">\leq</span> E_<span class="pl-c1">0</span><span class="pl-pds">$</span></span> usando as fórmulas <span class="pl-c1">\eqref</span>{eq:quatro-b-n0} e <span class="pl-c1">\eqref</span>{eq:quatro-b-n}.</td>
      </tr>
      <tr>
        <td id="L25" class="blob-num js-line-number" data-line-number="25"></td>
        <td id="LC25" class="blob-code blob-code-inner js-file-line">
</td>
      </tr>
      <tr>
        <td id="L26" class="blob-num js-line-number" data-line-number="26"></td>
        <td id="LC26" class="blob-code blob-code-inner js-file-line"><span class="pl-c1">\begin</span>{align}</td>
      </tr>
      <tr>
        <td id="L27" class="blob-num js-line-number" data-line-number="27"></td>
        <td id="LC27" class="blob-code blob-code-inner js-file-line">	n_0 &amp;= <span class="pl-c1">\label</span>{eq:quatro-b-n0}</td>
      </tr>
      <tr>
        <td id="L28" class="blob-num js-line-number" data-line-number="28"></td>
        <td id="LC28" class="blob-code blob-code-inner js-file-line">	       <span class="pl-c1">\frac</span>{z_<span class="pl-c1">\gamma</span>^2 p(1-p)}{E_0^2} <span class="pl-c1">\\</span></td>
      </tr>
      <tr>
        <td id="L29" class="blob-num js-line-number" data-line-number="29"></td>
        <td id="LC29" class="blob-code blob-code-inner js-file-line">	n &amp;= <span class="pl-c1">\label</span>{eq:quatro-b-n}</td>
      </tr>
      <tr>
        <td id="L30" class="blob-num js-line-number" data-line-number="30"></td>
        <td id="LC30" class="blob-code blob-code-inner js-file-line">	     <span class="pl-c1">\frac</span>{N n_0}{N + n_0 - 1} </td>
      </tr>
      <tr>
        <td id="L31" class="blob-num js-line-number" data-line-number="31"></td>
        <td id="LC31" class="blob-code blob-code-inner js-file-line"><span class="pl-c1">\end</span>{align}</td>
      </tr>
      <tr>
        <td id="L32" class="blob-num js-line-number" data-line-number="32"></td>
        <td id="LC32" class="blob-code blob-code-inner js-file-line">
</td>
      </tr>
      <tr>
        <td id="L33" class="blob-num js-line-number" data-line-number="33"></td>
        <td id="LC33" class="blob-code blob-code-inner js-file-line">Para obter o erro amostral máximo, dado o tamanho da amostra, as equações <span class="pl-c1">\eqref</span>{eq:quatro-b-e0-expr} e <span class="pl-c1">\eqref</span>{eq:quatro-b-n0-expr} podem ser usadas. </td>
      </tr>
      <tr>
        <td id="L34" class="blob-num js-line-number" data-line-number="34"></td>
        <td id="LC34" class="blob-code blob-code-inner js-file-line">
</td>
      </tr>
      <tr>
        <td id="L35" class="blob-num js-line-number" data-line-number="35"></td>
        <td id="LC35" class="blob-code blob-code-inner js-file-line">Sabemos <span class="pl-s"><span class="pl-pds">$</span>n<span class="pl-pds">$</span></span>, <span class="pl-s"><span class="pl-pds">$</span>N<span class="pl-pds">$</span></span> <span class="pl-s"><span class="pl-pds">$</span>z<span class="pl-c1">\gamma</span><span class="pl-pds">$</span></span> e <span class="pl-s"><span class="pl-pds">$</span><span class="pl-c1">\hat</span>{p}<span class="pl-pds">$</span></span>, que pode ser usado para aproximar <span class="pl-s"><span class="pl-pds">$</span>p<span class="pl-pds">$</span></span>. Para encontrar <span class="pl-s"><span class="pl-pds">$</span>E_<span class="pl-c1">0</span><span class="pl-pds">$</span></span>, podemos deduzir <span class="pl-c1">\eqref</span>{eq:quatro-b-e0-expr} algebricamente de <span class="pl-c1">\eqref</span>{eq:quatro-b-n0}. Para encontrar <span class="pl-s"><span class="pl-pds">$</span>n_<span class="pl-c1">0</span><span class="pl-pds">$</span></span>, usado em <span class="pl-c1">\eqref</span>{eq:quatro-b-e0-expr}, <span class="pl-s"><span class="pl-pds">$</span>n_<span class="pl-c1">0</span><span class="pl-pds">$</span></span> pode ser isolada em <span class="pl-c1">\eqref</span>{eq:quatro-b-n}, resultando em <span class="pl-c1">\eqref</span>{eq:quatro-b-n0-expr}. Para tal foi utilizada a ferramenta WolframAlpha<span class="pl-c1">\footnote</span>{<span class="pl-c1">\url</span>{https://www.wolframalpha.com/input/?i=solve+n<span class="pl-cce">\%</span>3D(N*m)<span class="pl-cce">\%</span>2F(N<span class="pl-cce">\%</span>2Bm-1)+for+m}}.</td>
      </tr>
      <tr>
        <td id="L36" class="blob-num js-line-number" data-line-number="36"></td>
        <td id="LC36" class="blob-code blob-code-inner js-file-line">
</td>
      </tr>
      <tr>
        <td id="L37" class="blob-num js-line-number" data-line-number="37"></td>
        <td id="LC37" class="blob-code blob-code-inner js-file-line"><span class="pl-c1">\begin</span>{align}</td>
      </tr>
      <tr>
        <td id="L38" class="blob-num js-line-number" data-line-number="38"></td>
        <td id="LC38" class="blob-code blob-code-inner js-file-line">	E_0 &amp;= <span class="pl-c1">\label</span>{eq:quatro-b-e0-expr}</td>
      </tr>
      <tr>
        <td id="L39" class="blob-num js-line-number" data-line-number="39"></td>
        <td id="LC39" class="blob-code blob-code-inner js-file-line">	       <span class="pl-c1">\sqrt</span>{<span class="pl-c1">\frac</span>{z_<span class="pl-c1">\gamma</span> p(1 - p) }{n_0}} <span class="pl-c1">\\</span></td>
      </tr>
      <tr>
        <td id="L40" class="blob-num js-line-number" data-line-number="40"></td>
        <td id="LC40" class="blob-code blob-code-inner js-file-line">	n_0 &amp;= <span class="pl-c1">\label</span>{eq:quatro-b-n0-expr}</td>
      </tr>
      <tr>
        <td id="L41" class="blob-num js-line-number" data-line-number="41"></td>
        <td id="LC41" class="blob-code blob-code-inner js-file-line">	       <span class="pl-c1">\frac</span>{n-n N}{n-N}</td>
      </tr>
      <tr>
        <td id="L42" class="blob-num js-line-number" data-line-number="42"></td>
        <td id="LC42" class="blob-code blob-code-inner js-file-line"><span class="pl-c1">\end</span>{align}</td>
      </tr>
      <tr>
        <td id="L43" class="blob-num js-line-number" data-line-number="43"></td>
        <td id="LC43" class="blob-code blob-code-inner js-file-line">
</td>
      </tr>
      <tr>
        <td id="L44" class="blob-num js-line-number" data-line-number="44"></td>
        <td id="LC44" class="blob-code blob-code-inner js-file-line">Combinando <span class="pl-c1">\eqref</span>{eq:quatro-b-e0-expr} e <span class="pl-c1">\eqref</span>{eq:quatro-b-n0-expr} e substituindo os valores, o erro amostral máximo de uma amostra com <span class="pl-c1">\QUATROn</span> elementos é dado por <span class="pl-c1">\eqref</span>{eq:quatro-b-e0-calc}.</td>
      </tr>
      <tr>
        <td id="L45" class="blob-num js-line-number" data-line-number="45"></td>
        <td id="LC45" class="blob-code blob-code-inner js-file-line">
</td>
      </tr>
      <tr>
        <td id="L46" class="blob-num js-line-number" data-line-number="46"></td>
        <td id="LC46" class="blob-code blob-code-inner js-file-line"><span class="pl-c1">\begin</span>{align}</td>
      </tr>
      <tr>
        <td id="L47" class="blob-num js-line-number" data-line-number="47"></td>
        <td id="LC47" class="blob-code blob-code-inner js-file-line">	<span class="pl-c1">\label</span>{eq:quatro-b-e0-calc}</td>
      </tr>
      <tr>
        <td id="L48" class="blob-num js-line-number" data-line-number="48"></td>
        <td id="LC48" class="blob-code blob-code-inner js-file-line">	E_0 &amp;= <span class="pl-c1">\sqrt</span>{<span class="pl-c1">\frac</span>{z_<span class="pl-c1">\gamma</span> <span class="pl-c1">\hat</span>{p}(1 - <span class="pl-c1">\hat</span>{p}) }{<span class="pl-c1">\frac</span>{n-n N}{n - N}}} <span class="pl-c1">\\</span></td>
      </tr>
      <tr>
        <td id="L49" class="blob-num js-line-number" data-line-number="49"></td>
        <td id="LC49" class="blob-code blob-code-inner js-file-line">	    &amp;= <span class="pl-c1">\sqrt</span>{<span class="pl-c1">\frac</span>{<span class="pl-c1">\QUATROzy</span> <span class="pl-cce">\;</span><span class="pl-c1">\cdot</span><span class="pl-cce">\;</span> <span class="pl-c1">\QUATROpAmostral</span> (1 - <span class="pl-c1">\QUATROpAmostral</span>) }{<span class="pl-c1">\frac</span>{<span class="pl-c1">\QUATROn</span> - <span class="pl-c1">\QUATROn</span> <span class="pl-cce">\;</span><span class="pl-c1">\cdot</span><span class="pl-cce">\;</span> <span class="pl-c1">\QUATRON</span>}{<span class="pl-c1">\QUATROn</span> - <span class="pl-c1">\QUATRON</span>}}} <span class="pl-c1">\nonumber</span> <span class="pl-c1">\\</span></td>
      </tr>
      <tr>
        <td id="L50" class="blob-num js-line-number" data-line-number="50"></td>
        <td id="LC50" class="blob-code blob-code-inner js-file-line">	    &amp;= <span class="pl-c1">\QUATROBE</span> <span class="pl-c1">\nonumber</span></td>
      </tr>
      <tr>
        <td id="L51" class="blob-num js-line-number" data-line-number="51"></td>
        <td id="LC51" class="blob-code blob-code-inner js-file-line"><span class="pl-c1">\end</span>{align}</td>
      </tr>
      <tr>
        <td id="L52" class="blob-num js-line-number" data-line-number="52"></td>
        <td id="LC52" class="blob-code blob-code-inner js-file-line">
</td>
      </tr>
      <tr>
        <td id="L53" class="blob-num js-line-number" data-line-number="53"></td>
        <td id="LC53" class="blob-code blob-code-inner js-file-line">Como <span class="pl-s"><span class="pl-pds">$</span>E_<span class="pl-c1">0</span> = <span class="pl-c1">\QUATROBE</span><span class="pl-pds">$</span></span>, a amostra não é suficiente para uma margem de erro de 2<span class="pl-cce">\%</span>. Usando <span class="pl-c1">\eqref</span>{eq:quatro-b-n0} obtemos <span class="pl-s"><span class="pl-pds">$</span>n_<span class="pl-c1">0</span><span class="pl-pds">$</span></span> necessário para uma margem de erro de 2<span class="pl-cce">\%</span> em <span class="pl-c1">\eqref</span>{eq:quatro-b-n0-result}.</td>
      </tr>
      <tr>
        <td id="L54" class="blob-num js-line-number" data-line-number="54"></td>
        <td id="LC54" class="blob-code blob-code-inner js-file-line"><span class="pl-c">% e \eqref{eq:quatro-b-n} (pois sabemos o tamanho da população), necessitaríamos de uma amostra com \QUATROBn elementos. Os detalhes são mostrados em \eqref{eq:quatro-b-n-2p}.</span></td>
      </tr>
      <tr>
        <td id="L55" class="blob-num js-line-number" data-line-number="55"></td>
        <td id="LC55" class="blob-code blob-code-inner js-file-line">
</td>
      </tr>
      <tr>
        <td id="L56" class="blob-num js-line-number" data-line-number="56"></td>
        <td id="LC56" class="blob-code blob-code-inner js-file-line"><span class="pl-c1">\begin</span>{align}</td>
      </tr>
      <tr>
        <td id="L57" class="blob-num js-line-number" data-line-number="57"></td>
        <td id="LC57" class="blob-code blob-code-inner js-file-line">	n_0 &amp;= <span class="pl-c1">\frac</span>{z_<span class="pl-c1">\gamma</span>^2 p(1-p)}{E_0^2} <span class="pl-c1">\nonumber</span> <span class="pl-c1">\\</span></td>
      </tr>
      <tr>
        <td id="L58" class="blob-num js-line-number" data-line-number="58"></td>
        <td id="LC58" class="blob-code blob-code-inner js-file-line">	    &amp;= <span class="pl-c1">\frac</span>{<span class="pl-c1">\QUATROzy</span>^2 <span class="pl-c1">\QUATROpAmostral</span>(1-<span class="pl-c1">\QUATROpAmostral</span>)}{0.02^2} <span class="pl-c1">\nonumber</span> <span class="pl-c1">\\</span></td>
      </tr>
      <tr>
        <td id="L59" class="blob-num js-line-number" data-line-number="59"></td>
        <td id="LC59" class="blob-code blob-code-inner js-file-line">	    &amp;= <span class="pl-c1">\label</span>{eq:quatro-b-n0-result}</td>
      </tr>
      <tr>
        <td id="L60" class="blob-num js-line-number" data-line-number="60"></td>
        <td id="LC60" class="blob-code blob-code-inner js-file-line">	       <span class="pl-c1">\QUATROBnz</span>	</td>
      </tr>
      <tr>
        <td id="L61" class="blob-num js-line-number" data-line-number="61"></td>
        <td id="LC61" class="blob-code blob-code-inner js-file-line"><span class="pl-c1">\end</span>{align}</td>
      </tr>
      <tr>
        <td id="L62" class="blob-num js-line-number" data-line-number="62"></td>
        <td id="LC62" class="blob-code blob-code-inner js-file-line">
</td>
      </tr>
      <tr>
        <td id="L63" class="blob-num js-line-number" data-line-number="63"></td>
        <td id="LC63" class="blob-code blob-code-inner js-file-line">Como o tamanho da população é conhecido, o valor em <span class="pl-c1">\eqref</span>{eq:quatro-b-n0-result} pode ser reduzido para o valor em <span class="pl-c1">\eqref</span>{eq:quatro-b-n-result}.</td>
      </tr>
      <tr>
        <td id="L64" class="blob-num js-line-number" data-line-number="64"></td>
        <td id="LC64" class="blob-code blob-code-inner js-file-line">
</td>
      </tr>
      <tr>
        <td id="L65" class="blob-num js-line-number" data-line-number="65"></td>
        <td id="LC65" class="blob-code blob-code-inner js-file-line"><span class="pl-c1">\begin</span>{align}</td>
      </tr>
      <tr>
        <td id="L66" class="blob-num js-line-number" data-line-number="66"></td>
        <td id="LC66" class="blob-code blob-code-inner js-file-line">	n &amp;= <span class="pl-c1">\frac</span>{N n_0}{N + n_0 - 1} <span class="pl-c1">\nonumber</span> <span class="pl-c1">\\</span></td>
      </tr>
      <tr>
        <td id="L67" class="blob-num js-line-number" data-line-number="67"></td>
        <td id="LC67" class="blob-code blob-code-inner js-file-line">    &amp;= <span class="pl-c1">\QUATROBn</span> <span class="pl-c1">\nonumber</span> <span class="pl-c1">\\</span></td>
      </tr>
      <tr>
        <td id="L68" class="blob-num js-line-number" data-line-number="68"></td>
        <td id="LC68" class="blob-code blob-code-inner js-file-line">	  &amp;<span class="pl-c1">\sim</span> <span class="pl-c1">\label</span>{eq:quatro-b-n-result} </td>
      </tr>
      <tr>
        <td id="L69" class="blob-num js-line-number" data-line-number="69"></td>
        <td id="LC69" class="blob-code blob-code-inner js-file-line">	     <span class="pl-c1">\QUATROBnceil</span></td>
      </tr>
      <tr>
        <td id="L70" class="blob-num js-line-number" data-line-number="70"></td>
        <td id="LC70" class="blob-code blob-code-inner js-file-line"><span class="pl-c1">\end</span>{align}</td>
      </tr>
      <tr>
        <td id="L71" class="blob-num js-line-number" data-line-number="71"></td>
        <td id="LC71" class="blob-code blob-code-inner js-file-line">
</td>
      </tr>
      <tr>
        <td id="L72" class="blob-num js-line-number" data-line-number="72"></td>
        <td id="LC72" class="blob-code blob-code-inner js-file-line"><span class="pl-c1">\subsection</span>{Tamanho da amostra sem amostra piloto}</td>
      </tr>
      <tr>
        <td id="L73" class="blob-num js-line-number" data-line-number="73"></td>
        <td id="LC73" class="blob-code blob-code-inner js-file-line">
</td>
      </tr>
      <tr>
        <td id="L74" class="blob-num js-line-number" data-line-number="74"></td>
        <td id="LC74" class="blob-code blob-code-inner js-file-line">Para estimar o tamanho da amostra que garanta erro amostral máximo de 2<span class="pl-cce">\%</span> sem uma estimativa adequada para a proporção populacional <span class="pl-s"><span class="pl-pds">$</span>p<span class="pl-pds">$</span></span>, deve ser utilizada a fórmula <span class="pl-c1">\eqref</span>{eq:quatro-c-n0-expr} para obtenção de <span class="pl-s"><span class="pl-pds">$</span>n_<span class="pl-c1">0</span><span class="pl-pds">$</span></span>. A fórmula em <span class="pl-c1">\eqref</span>{eq:quatro-c-n0-expr} foi obtida super-estimando <span class="pl-s"><span class="pl-pds">$</span>p<span class="pl-pds">$</span></span> como <span class="pl-s"><span class="pl-pds">$</span><span class="pl-c1">0.5</span><span class="pl-pds">$</span></span> e aplicando isso em <span class="pl-c1">\eqref</span>{eq:quatro-b-n0}</td>
      </tr>
      <tr>
        <td id="L75" class="blob-num js-line-number" data-line-number="75"></td>
        <td id="LC75" class="blob-code blob-code-inner js-file-line">
</td>
      </tr>
      <tr>
        <td id="L76" class="blob-num js-line-number" data-line-number="76"></td>
        <td id="LC76" class="blob-code blob-code-inner js-file-line"><span class="pl-c1">\begin</span>{align}</td>
      </tr>
      <tr>
        <td id="L77" class="blob-num js-line-number" data-line-number="77"></td>
        <td id="LC77" class="blob-code blob-code-inner js-file-line">	n_0 &amp;= <span class="pl-c1">\label</span>{eq:quatro-c-n0-expr}</td>
      </tr>
      <tr>
        <td id="L78" class="blob-num js-line-number" data-line-number="78"></td>
        <td id="LC78" class="blob-code blob-code-inner js-file-line">	       <span class="pl-c1">\frac</span>{z_<span class="pl-c1">\gamma</span>^2}{4 E_0^2} <span class="pl-c1">\\</span></td>
      </tr>
      <tr>
        <td id="L79" class="blob-num js-line-number" data-line-number="79"></td>
        <td id="LC79" class="blob-code blob-code-inner js-file-line">	    &amp;= <span class="pl-c1">\frac</span>{<span class="pl-c1">\QUATROzy</span>^2}{4 0.02^2} <span class="pl-c1">\nonumber</span> <span class="pl-c1">\\</span></td>
      </tr>
      <tr>
        <td id="L80" class="blob-num js-line-number" data-line-number="80"></td>
        <td id="LC80" class="blob-code blob-code-inner js-file-line">	    &amp;= <span class="pl-c1">\label</span>{eq:quatro-c-n0-result}</td>
      </tr>
      <tr>
        <td id="L81" class="blob-num js-line-number" data-line-number="81"></td>
        <td id="LC81" class="blob-code blob-code-inner js-file-line">	       <span class="pl-c1">\QUATROCnz</span></td>
      </tr>
      <tr>
        <td id="L82" class="blob-num js-line-number" data-line-number="82"></td>
        <td id="LC82" class="blob-code blob-code-inner js-file-line"><span class="pl-c1">\end</span>{align}</td>
      </tr>
      <tr>
        <td id="L83" class="blob-num js-line-number" data-line-number="83"></td>
        <td id="LC83" class="blob-code blob-code-inner js-file-line">
</td>
      </tr>
      <tr>
        <td id="L84" class="blob-num js-line-number" data-line-number="84"></td>
        <td id="LC84" class="blob-code blob-code-inner js-file-line">Novamente, <span class="pl-c1">\eqref</span>{eq:quatro-c-n0-result} pode ser substituído em <span class="pl-c1">\eqref</span>{eq:quatro-b-n}, resultando no tamanho da amostra necessário, em <span class="pl-c1">\eqref</span>{eq:quatro-c-n-result}.</td>
      </tr>
      <tr>
        <td id="L85" class="blob-num js-line-number" data-line-number="85"></td>
        <td id="LC85" class="blob-code blob-code-inner js-file-line">
</td>
      </tr>
      <tr>
        <td id="L86" class="blob-num js-line-number" data-line-number="86"></td>
        <td id="LC86" class="blob-code blob-code-inner js-file-line"><span class="pl-c1">\begin</span>{align}</td>
      </tr>
      <tr>
        <td id="L87" class="blob-num js-line-number" data-line-number="87"></td>
        <td id="LC87" class="blob-code blob-code-inner js-file-line">	n &amp;= <span class="pl-c1">\frac</span>{N n_0}{N + n_0 - 1} <span class="pl-c1">\nonumber</span> <span class="pl-c1">\\</span></td>
      </tr>
      <tr>
        <td id="L88" class="blob-num js-line-number" data-line-number="88"></td>
        <td id="LC88" class="blob-code blob-code-inner js-file-line">	  &amp;= <span class="pl-c1">\frac</span>{<span class="pl-c1">\QUATRON</span> <span class="pl-c1">\cdot</span> <span class="pl-c1">\QUATROCnz</span>}{<span class="pl-c1">\QUATRON</span> + <span class="pl-c1">\QUATROCnz</span> - 1} <span class="pl-c1">\nonumber</span> <span class="pl-c1">\\</span></td>
      </tr>
      <tr>
        <td id="L89" class="blob-num js-line-number" data-line-number="89"></td>
        <td id="LC89" class="blob-code blob-code-inner js-file-line">	  &amp;= <span class="pl-c1">\QUATROCn</span> <span class="pl-c1">\nonumber</span> <span class="pl-c1">\\</span></td>
      </tr>
      <tr>
        <td id="L90" class="blob-num js-line-number" data-line-number="90"></td>
        <td id="LC90" class="blob-code blob-code-inner js-file-line">	  &amp;= <span class="pl-c1">\label</span>{eq:quatro-c-n-result} </td>
      </tr>
      <tr>
        <td id="L91" class="blob-num js-line-number" data-line-number="91"></td>
        <td id="LC91" class="blob-code blob-code-inner js-file-line">	     <span class="pl-c1">\QUATROCnceil</span></td>
      </tr>
      <tr>
        <td id="L92" class="blob-num js-line-number" data-line-number="92"></td>
        <td id="LC92" class="blob-code blob-code-inner js-file-line"><span class="pl-c1">\end</span>{align}</td>
      </tr>
</table>

  </div>

</div>

<button type="button" data-facebox="#jump-to-line" data-facebox-class="linejump" data-hotkey="l" class="hidden">Jump to Line</button>
<div id="jump-to-line" style="display:none">
  <!-- </textarea> --><!-- '"` --><form accept-charset="UTF-8" action="" class="js-jump-to-line-form" method="get"><div style="margin:0;padding:0;display:inline"><input name="utf8" type="hidden" value="&#x2713;" /></div>
    <input class="form-control linejump-input js-jump-to-line-field" type="text" placeholder="Jump to line&hellip;" aria-label="Jump to line" autofocus>
    <button type="submit" class="btn">Go</button>
</form></div>

  </div>
  <div class="modal-backdrop"></div>
</div>


    </div>
  </div>

    </div>

        <div class="container site-footer-container">
  <div class="site-footer" role="contentinfo">
    <ul class="site-footer-links right">
        <li><a href="https://status.github.com/" data-ga-click="Footer, go to status, text:status">Status</a></li>
      <li><a href="https://developer.github.com" data-ga-click="Footer, go to api, text:api">API</a></li>
      <li><a href="https://training.github.com" data-ga-click="Footer, go to training, text:training">Training</a></li>
      <li><a href="https://shop.github.com" data-ga-click="Footer, go to shop, text:shop">Shop</a></li>
        <li><a href="https://github.com/blog" data-ga-click="Footer, go to blog, text:blog">Blog</a></li>
        <li><a href="https://github.com/about" data-ga-click="Footer, go to about, text:about">About</a></li>

    </ul>

    <a href="https://github.com" aria-label="Homepage" class="site-footer-mark" title="GitHub">
      <svg aria-hidden="true" class="octicon octicon-mark-github" height="24" version="1.1" viewBox="0 0 16 16" width="24"><path d="M8 0C3.58 0 0 3.58 0 8c0 3.54 2.29 6.53 5.47 7.59 0.4 0.07 0.55-0.17 0.55-0.38 0-0.19-0.01-0.82-0.01-1.49-2.01 0.37-2.53-0.49-2.69-0.94-0.09-0.23-0.48-0.94-0.82-1.13-0.28-0.15-0.68-0.52-0.01-0.53 0.63-0.01 1.08 0.58 1.23 0.82 0.72 1.21 1.87 0.87 2.33 0.66 0.07-0.52 0.28-0.87 0.51-1.07-1.78-0.2-3.64-0.89-3.64-3.95 0-0.87 0.31-1.59 0.82-2.15-0.08-0.2-0.36-1.02 0.08-2.12 0 0 0.67-0.21 2.2 0.82 0.64-0.18 1.32-0.27 2-0.27 0.68 0 1.36 0.09 2 0.27 1.53-1.04 2.2-0.82 2.2-0.82 0.44 1.1 0.16 1.92 0.08 2.12 0.51 0.56 0.82 1.27 0.82 2.15 0 3.07-1.87 3.75-3.65 3.95 0.29 0.25 0.54 0.73 0.54 1.48 0 1.07-0.01 1.93-0.01 2.2 0 0.21 0.15 0.46 0.55 0.38C13.71 14.53 16 11.53 16 8 16 3.58 12.42 0 8 0z"></path></svg>
</a>
    <ul class="site-footer-links">
      <li>&copy; 2016 <span title="0.07007s from github-fe123-cp1-prd.iad.github.net">GitHub</span>, Inc.</li>
        <li><a href="https://github.com/site/terms" data-ga-click="Footer, go to terms, text:terms">Terms</a></li>
        <li><a href="https://github.com/site/privacy" data-ga-click="Footer, go to privacy, text:privacy">Privacy</a></li>
        <li><a href="https://github.com/security" data-ga-click="Footer, go to security, text:security">Security</a></li>
        <li><a href="https://github.com/contact" data-ga-click="Footer, go to contact, text:contact">Contact</a></li>
        <li><a href="https://help.github.com" data-ga-click="Footer, go to help, text:help">Help</a></li>
    </ul>
  </div>
</div>



    
    

    <div id="ajax-error-message" class="ajax-error-message flash flash-error">
      <svg aria-hidden="true" class="octicon octicon-alert" height="16" version="1.1" viewBox="0 0 16 16" width="16"><path d="M15.72 12.5l-6.85-11.98C8.69 0.21 8.36 0.02 8 0.02s-0.69 0.19-0.87 0.5l-6.85 11.98c-0.18 0.31-0.18 0.69 0 1C0.47 13.81 0.8 14 1.15 14h13.7c0.36 0 0.69-0.19 0.86-0.5S15.89 12.81 15.72 12.5zM9 12H7V10h2V12zM9 9H7V5h2V9z"></path></svg>
      <button type="button" class="flash-close js-flash-close js-ajax-error-dismiss" aria-label="Dismiss error">
        <svg aria-hidden="true" class="octicon octicon-x" height="16" version="1.1" viewBox="0 0 12 16" width="12"><path d="M7.48 8l3.75 3.75-1.48 1.48-3.75-3.75-3.75 3.75-1.48-1.48 3.75-3.75L0.77 4.25l1.48-1.48 3.75 3.75 3.75-3.75 1.48 1.48-3.75 3.75z"></path></svg>
      </button>
      Something went wrong with that request. Please try again.
    </div>


      
      <script crossorigin="anonymous" integrity="sha256-+BdcIzYLQqTrGLIxn+/q4lLP7qSC+4BDVvQTalK/3bM=" src="https://assets-cdn.github.com/assets/frameworks-f8175c23360b42a4eb18b2319fefeae252cfeea482fb804356f4136a52bfddb3.js"></script>
      <script async="async" crossorigin="anonymous" integrity="sha256-cjtmWi/wrMkDj4U/NpCKKrUjWV111ODe0dkTi1p/VSY=" src="https://assets-cdn.github.com/assets/github-723b665a2ff0acc9038f853f36908a2ab523595d75d4e0ded1d9138b5a7f5526.js"></script>
      
      
      
      
      
      
    <div class="js-stale-session-flash stale-session-flash flash flash-warn flash-banner hidden">
      <svg aria-hidden="true" class="octicon octicon-alert" height="16" version="1.1" viewBox="0 0 16 16" width="16"><path d="M15.72 12.5l-6.85-11.98C8.69 0.21 8.36 0.02 8 0.02s-0.69 0.19-0.87 0.5l-6.85 11.98c-0.18 0.31-0.18 0.69 0 1C0.47 13.81 0.8 14 1.15 14h13.7c0.36 0 0.69-0.19 0.86-0.5S15.89 12.81 15.72 12.5zM9 12H7V10h2V12zM9 9H7V5h2V9z"></path></svg>
      <span class="signed-in-tab-flash">You signed in with another tab or window. <a href="">Reload</a> to refresh your session.</span>
      <span class="signed-out-tab-flash">You signed out in another tab or window. <a href="">Reload</a> to refresh your session.</span>
    </div>
    <div class="facebox" id="facebox" style="display:none;">
  <div class="facebox-popup">
    <div class="facebox-content" role="dialog" aria-labelledby="facebox-header" aria-describedby="facebox-description">
    </div>
    <button type="button" class="facebox-close js-facebox-close" aria-label="Close modal">
      <svg aria-hidden="true" class="octicon octicon-x" height="16" version="1.1" viewBox="0 0 12 16" width="12"><path d="M7.48 8l3.75 3.75-1.48 1.48-3.75-3.75-3.75 3.75-1.48-1.48 3.75-3.75L0.77 4.25l1.48-1.48 3.75 3.75 3.75-3.75 1.48 1.48-3.75 3.75z"></path></svg>
    </button>
  </div>
</div>

  </body>
</html>



\section{Variável Opinião}
\label{questao:5}
\newcommand{\TRESicmin}{1.85\xspace}
\newcommand{\TRESicmax}{3.10\xspace}
\newcommand{\TRESicNoveNoveMin}{1.62\xspace}
\newcommand{\TRESicNoveNoveMax}{3.33\xspace}
\newcommand{\TRESnZero}{26\xspace}


	Foi retirada uma amostra aleatória simples sem reposição com \CINCOn
	elementos, dentre os \CINCON alunos que possuíam um valor definido para
	a variável Opinião.  A proporção amostral $\hat{p}$ de alunos com
	opiniões negativas (ou seja, alunos que consideraram ``Insatisfeitos''
	ou ``Muito insatisfeitos'') foi de \CINCOpAmostral.

	\begin{align*} 
		\hat{p}  &= {\frac{\CINCOqtdOpinioesNegativas}{\CINCOn}} = \CINCOpAmostral \nonumber \\
	\end{align*}

\subsection{Intervalo de confiança}

	O intervalo de confiança para a proporção $p$ é dado pela
    \autoref{equation: intervalo de confianca para proporcao 1}, mas como $\sigma_{\hat{p}}$ 
    não é conhecido e $n$ é grande, a 
    \autoref{equation: intervalo de confianca para proporcao 2} pode ser usada.

	\begin{align} 
		IC(p, 95\%) 
					&= \CINCOpAmostral \pm \CINCOzy \sqrt{\frac{\CINCOpAmostral (1- \CINCOpAmostral)}{\CINCOn}} \nonumber \\
					&= \CINCOpAmostral \pm \CINCOAdelta \nonumber \\
					\label{eq:cinco-a-result}
					&= [\CINCOAICinf, \CINCOAICsup]
	\end{align}

	Portanto, dada uma amostra aleatória simples de \CINCOn alunos tomados
	dentre os \CINCON da população, há 95\% de chance de que a proporção de
	alunos com opiniões negativas seja um valor entre \CINCOAICinf e
	\CINCOAICsup.

\subsection{Precisão}

	Como já apresentado anteriormente, é possível aplicar a fórmula
    definida na \autoref{equation: erro amostral maximo}, de modo a obter
    o valor do erro amostral máximo de uma amostra com \CINCOn elementos.

	\begin{align}
		\label{eq:cinco-b-e0-calc}
		E_0 &= \sqrt{\frac{\CINCOzy \;\cdot\; \CINCOpAmostral (1 - \CINCOpAmostral) }{\frac{\CINCOn - \CINCOn \;\cdot\; \CINCON}{\CINCOn - \CINCON}}} \\
			&= \CINCOBE \nonumber
	\end{align}

	Como o valor de $E_0 = \CINCOBE$, a amostra não é suficiente para uma
	margem de erro de 2\%. Aplicando a fórmula \autoref{equation: tamanho amostra 1} já
	mencionada, obtem-se o $n_0$ necessário para uma margem de erro de 2\%
	em \eqref{eq:cinco-b-n0-result}.

	\begin{align}
		n_0 &= \frac{\CINCOzy^2 \cdot \CINCOpAmostral(1-\CINCOpAmostral)}{\num{0.02}^2} \nonumber \\
			&= \label{eq:cinco-b-n0-result}
			   \CINCOBnz	
	\end{align}

	Como o tamanho da população é conhecido, o valor em
	\eqref{eq:cinco-b-n0-result} pode ainda ser reduzido para o valor obtido
	em \eqref{eq:cinco-b-n-result}.

	\begin{align}
		n &= \lceil \CINCOBn \nonumber \rceil \\
		  &= \label{eq:cinco-b-n-result} 
			 \CINCOBnceil
	\end{align}

	Pode-se concluir que a amostra de tamanho 200 não é suficiente para um
	intervalo de confiança de 95\% com precisão de 2\%, sendo, portanto,
	neccessário obter uma amostra de tamanho mínimo de \CINCOBnceil.

\subsection{Tamanho da amostra sem amostra piloto}

	Para estimar o tamanho da amostra que garanta erro amostral máximo de
	$2\%$ com um nível de confiança de $95\%$ e sem uma estimativa adequada
	para a proporção populacional $p$, foi utilizada a fórmula
	\autoref{equation: tamanho amostra 1} para obtenção de $n_0$.  
	De modo análogo a qurestão 4, o valor de $p$ foi superestimado para $\num{0.5}$ 
	aplicando na \autoref{equation: tamanho amostra 1} supracitada.

	\begin{align}
		n_0 &= \frac{\CINCOzy^2}{4 \cdot (\num{0.02})^2} \nonumber \\
			&= \label{eq:cinco-c-n0-result}
			   \CINCOCnz
	\end{align}

	Em seguida, como o tamanho da população é conhecido, o resultado obtido
	em \eqref{eq:cinco-c-n0-result} pode ser substituído na fórmula
	\autoref{equation: tamanho amostra 2}, resultando em \eqref{eq:cinco-c-n-result}, o qual
	representa o tamanho da amostra necessário.

	\begin{align}
		n &= \Big\lceil \frac{\CINCON \cdot \CINCOCnz}{\CINCON + \CINCOCnz - 1}
		  \nonumber \Big\rceil \\
		  &= \lceil \CINCOCn \nonumber \rceil\\
		  &= \label{eq:cinco-c-n-result} 
			 \CINCOCnceil
	\end{align}

	Portanto, considerando a ausência de uma amostra piloto, é preciso ter
	uma amostra de tamanho mínimo de \CINCOCnceil para estimar com $95\%$ de
	confiança e precisão de $2\%$ a proporção populacional de alunos com
	opiniões negativas sobre a EAD da TYU.


\section{Variável Renda}
\label{questao:6}
\newcommand{\TRESicmin}{1.85\xspace}
\newcommand{\TRESicmax}{3.10\xspace}
\newcommand{\TRESicNoveNoveMin}{1.62\xspace}
\newcommand{\TRESicNoveNoveMax}{3.33\xspace}
\newcommand{\TRESnZero}{26\xspace}


Para avaliação da variável Renda, primeiramente foi feita uma recodificação
da mesma em uma variável quantiatativa de dois valores: alunos com renda
familiar inferior a $\num{2,5}$ salários mínimos e alunos com renda igual ou
superior a $\num{2,5}$ salários mínimos. Nesse processo, apenas foram considerados
aqueles alunos que possuíam um valor definido para a variável, totalizando
$N=\SEISN$ alunos. Em seguida, retirou-se uma amostra aleatório simples de
\SEISn elementos dessa população. A proporção amostral $\hat{p}$ de alunos
com renda superior a $\num{2.5}$ salários mínimos é de \SEISpAmostral.

\subsection{Intervalo de Confiança}

	O intervalo de confiança para a a estatística $\hat{p}$ é dado por:
	%
	\begin{align*}
		IC(p, 95\%) &= \hat{p} \pm z_\gamma \sigma_{\hat{p}} \\
					&= \hat{p} \pm z_\gamma \sqrt{\frac{\hat{p}(1-\hat{p})}{n}} \\
					&= \SEISpAmostral \pm \SEISzy \sqrt{\frac{\SEISpAmostral (1- \SEISpAmostral)}{\SEISn}} \nonumber \\
					&= \SEISpAmostral \pm \SEISAdelta \nonumber \\
					&= [\SEISAICinf, \SEISAICsup]
	\end{align*}

	\noindent Dada uma amostra aleatória simples de \SEISn alunos
	selecionados dentre os \SEISN da população há $95\%$ de chance de que a
	proporção de alunos que possuem renda familiar superior a $\num{2,5}$ salários
	mínimos seja um valor no intervalo $[\SEISAICinf, \SEISAICsup]$

\subsection{Precisão}
	
	É possível calcular o tamanho da amostra necessária para que $|\hat{p} -
	E(\hat{p})| \leq E_0$ aplicando as seguintes equações:
	%
	\begin{align}
		n_0 &= \label{eq:seis-b-n0}
			   \frac{z_\gamma^2 p(1-p)}{E_0^2} \\
		n &= \label{eq:seis-b-n}
			 \frac{N n_0}{N + n_0 - 1} 
	\end{align}

	Para obter o erro amostral máximo, dado o tamanho da amostra, as equações
	\eqref{eq:seis-b-e0-expr} e \eqref{eq:seis-b-n0-expr} podem ser usadas. 

	Sabemos $n$, $N$ $z\gamma$ e $\hat{p}$, que pode ser usado para aproximar
	$p$. Para encontrar $E_0$, podemos deduzir \eqref{eq:seis-b-e0-expr}
	algebricamente de \eqref{eq:seis-b-n0}. Para encontrar $n_0$, usado em
	\eqref{eq:seis-b-e0-expr}, $n_0$ pode ser isolada em
	\eqref{eq:seis-b-n}, resultando em \eqref{eq:seis-b-n0-expr}. Para tal
	foi utilizada a ferramenta
	WolframAlpha\footnote{\url{https://www.wolframalpha.com/input/?i=solve+n\%3D(N*m)\%2F(N\%2Bm-1)+for+m}}.

	\begin{align}
		E_0 &= \label{eq:seis-b-e0-expr}
			   \sqrt{\frac{z_\gamma p(1 - p) }{n_0}} \\
		n_0 &= \label{eq:seis-b-n0-expr}
			   \frac{n-n N}{n-N}
	\end{align}

	Combinando \eqref{eq:seis-b-e0-expr} e \eqref{eq:seis-b-n0-expr} e
	substituindo os valores, o erro amostral máximo de uma amostra com
	\SEISn elementos é dado por \eqref{eq:seis-b-e0-calc}.

	\begin{align}
		\label{eq:seis-b-e0-calc}
		E_0 &= \sqrt{\frac{z_\gamma \hat{p}(1 - \hat{p}) }{\frac{n-n N}{n - N}}} \\
			&= \sqrt{\frac{\SEISzy \;\cdot\; \SEISpAmostral (1 - \SEISpAmostral) }{\frac{\SEISn - \SEISn \;\cdot\; \SEISN}{\SEISn - \SEISN}}} \nonumber \\
			&= \SEISBE \nonumber
	\end{align}

	Como $E_0 = \SEISBE$, a amostra é suficientemente grande para seafirmar
	uma margem de erro de $2\%$.

\subsection{Tamanho da Amostra sem Amostra-Piloto}

	Para estimar o tamanho da amostra que garanta erro amostral máximo de 2\%
	sem uma estimativa adequada para a proporção populacional $p$, deve ser
	utilizada a fórmula \eqref{eq:seis-c-n0-expr} para obtenção de $n_0$. A
	fórmula em \eqref{eq:seis-c-n0-expr} foi obtida superestimando $p$ como
	$\num{0.5}$ e aplicando isso em \eqref{eq:seis-b-n0}

	\begin{align}
		n_0 &= \label{eq:seis-c-n0-expr}
			   \frac{z_\gamma^2 \cdot \num{0.5} \cdot (1 - \num{0.5})}{E_0^2} = \frac{z_\gamma^2}{4 \cdot E_0^2} \\
			&= \frac{\SEISzy^2}{4 \cdot (\num{0.02})^2} \nonumber \\
			&= \label{eq:seis-c-n0-result}
			   \SEISCnz
	\end{align}

	Novamente, \eqref{eq:seis-c-n0-result} pode ser substituído em
	\eqref{eq:seis-b-n}, resultando no tamanho da amostra necessário, em
	\eqref{eq:seis-c-n-result}.

	\begin{align}
		n &= \frac{N n_0}{N + n_0 - 1} \nonumber \\
		  &= \frac{\SEISN \cdot \SEISCnz}{\SEISN + \SEISCnz - 1} \nonumber \\
		  &= \SEISCn \nonumber \\
		  &= \label{eq:seis-c-n-result} 
			 \SEISCnceil
	\end{align}

	Portanto, considerando a ausência de uma amostra piloto, é preciso ter uma
	amostra de tamanho mínimo de \SEISCnceil para estimar com 95\% de
	confiança e precisão de 2\% a proporção populacional de alunos de EAD da TYU
	apresentam renda familiar superior a $\num{2,5}$ salários mínimosi.

\subsection{Impactos na Recodificação da Variável}

	A escolha da recodificação ou não da variável renda depende do parâmetro
	sendo estudado, e não da estimação do intervalo de confiança. Caso o
	intuito seja estudar o parâmetro proporção, a recodificação se faz
	necessária para a variável \textit{Renda} e não implicará em perda de
	informação.  No entanto, caso o objetivo seja estudar os parâmetros
	média e variância, uma recodificação impactaria em perda de informação.
	Em ambos os cenários -- estudo do parâmetro proporção e estudo da
	média/variância -- a estimação do intervalo de confiança não seria
	comprometida.


\end{document}
