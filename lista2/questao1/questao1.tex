\newcommand{\QUATROpAmostral}{\num{0.2700}\xspace}
\newcommand{\QUATROn}{200\xspace}
\newcommand{\QUATROy}{\num{54.0000}\xspace}
\newcommand{\QUATROyLinha}{\num{54.5000}\xspace}
\newcommand{\QUATROz}{\num{2.5633}\xspace}
\newcommand{\QUATROpValue}{\num{0.0052}\xspace}
\newcommand{\QUATROesVinte}{\num{0.0000}\xspace}
\newcommand{\QUATROesVinteUm}{\num{0.0248}\xspace}
\newcommand{\QUATROesVinteDois}{\num{0.0491}\xspace}
\newcommand{\QUATROesVinteTres}{\num{0.0731}\xspace}
\newcommand{\QUATROesVinteQuatro}{\num{0.0967}\xspace}
\newcommand{\QUATROesVinteCinco}{\num{0.1199}\xspace}
\newcommand{\QUATROesVinteSeis}{\num{0.1428}\xspace}
\newcommand{\QUATROesVinteSete}{\num{0.1655}\xspace}
\newcommand{\QUATROpVinte}{\num{0.0100}\xspace}
\newcommand{\QUATROpVinteUm}{\num{0.0241}\xspace}
\newcommand{\QUATROpVinteDois}{\num{0.0514}\xspace}
\newcommand{\QUATROpVinteTres}{\num{0.0980}\xspace}
\newcommand{\QUATROpVinteQuatro}{\num{0.1687}\xspace}
\newcommand{\QUATROpVinteCinco}{\num{0.2641}\xspace}
\newcommand{\QUATROpVinteSeis}{\num{0.3797}\xspace}
\newcommand{\QUATROpVinteSete}{\num{0.5057}\xspace}
\newcommand{\QUATROesAmostra}{\num{0.0731}\xspace}
\newcommand{\QUATROtamanhoAmostra}{\num{4055.1080}\xspace}
\newcommand{\QUATROtamanhoAmostraRounded}{4056\xspace}


Foram retiradas 200 amostras, sem reposição, com tamanhos 4, 8, 16, 30 e 100, 
dentre os 4986 alunos que possuíam um valor definido para a variável Renda.
A população possuí média de ${\UMx}$ para a variável Renda, com desvio padrão de
${\UMsd}$. A tabela \ref{tab:q1} apresentam os valores esperados da média amostral
e seu respectivos desvios padrões.

\begin{table}[h]
\centering
\caption{Valores das médias amostrais para a variável \textit{Renda}}
\label{tab:q1}
\vspace{0.5em}
\begin{tabular}{l r r r}
	\toprule
	\textbf{n} & \textbf{Média} & \textbf{Desvio Padrão} & \textbf{\sigma/\sqrt{n}\\
	\midrule
	$4$       & ${\UMx4}$   & ${\UMsd4}$   & ${\UMsde4}$   \\
	$8$       & ${\UMx8}$   & ${\UMsd8}$   & ${\UMsde8}$   \\
	$16$      & ${\UMx16}$  & ${\UMsd16}$  & ${\UMsde16}$  \\
	$30$      & ${\UMx30}$  & ${\UMsd30}$  & ${\UMsde30}$  \\
	$100$     & ${\UMx100}$ & ${\UMsd100}$ & ${\UMsde100}$ \\
	\bottomrule
\end{tabular}
\end{table}

\subsection{(item a)}
A medida que se aumentou o tamanho da amostra, de 4 para 16, o resultado se aproximou à media populacional.
Porém para amostra de tamanho 30 e 100, a média amostral se distanciou da média populacional.


\subsection{(item b)}
Sim, as amostras retiradas confirmam a afirmação. Conforme indicado na tabela \ref{tab:q1}, os valores 
de desvio padrão das amostras são próximos aos valores de ^\sigma/_sqrt{n} e para a amostra de tamanho 100, 
são exatamente iguais.

\subsection{(item c)}
Como gerar o gráfico? Escolher uma amostra dentre as 200 para algum tamanho? Qual seria o valor da distribuição?
