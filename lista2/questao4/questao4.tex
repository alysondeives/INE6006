\newcommand{\QUATROpAmostral}{\num{0.2700}\xspace}
\newcommand{\QUATROn}{200\xspace}
\newcommand{\QUATROy}{\num{54.0000}\xspace}
\newcommand{\QUATROyLinha}{\num{54.5000}\xspace}
\newcommand{\QUATROz}{\num{2.5633}\xspace}
\newcommand{\QUATROpValue}{\num{0.0052}\xspace}
\newcommand{\QUATROesVinte}{\num{0.0000}\xspace}
\newcommand{\QUATROesVinteUm}{\num{0.0248}\xspace}
\newcommand{\QUATROesVinteDois}{\num{0.0491}\xspace}
\newcommand{\QUATROesVinteTres}{\num{0.0731}\xspace}
\newcommand{\QUATROesVinteQuatro}{\num{0.0967}\xspace}
\newcommand{\QUATROesVinteCinco}{\num{0.1199}\xspace}
\newcommand{\QUATROesVinteSeis}{\num{0.1428}\xspace}
\newcommand{\QUATROesVinteSete}{\num{0.1655}\xspace}
\newcommand{\QUATROpVinte}{\num{0.0100}\xspace}
\newcommand{\QUATROpVinteUm}{\num{0.0241}\xspace}
\newcommand{\QUATROpVinteDois}{\num{0.0514}\xspace}
\newcommand{\QUATROpVinteTres}{\num{0.0980}\xspace}
\newcommand{\QUATROpVinteQuatro}{\num{0.1687}\xspace}
\newcommand{\QUATROpVinteCinco}{\num{0.2641}\xspace}
\newcommand{\QUATROpVinteSeis}{\num{0.3797}\xspace}
\newcommand{\QUATROpVinteSete}{\num{0.5057}\xspace}
\newcommand{\QUATROesAmostra}{\num{0.0731}\xspace}
\newcommand{\QUATROtamanhoAmostra}{\num{4055.1080}\xspace}
\newcommand{\QUATROtamanhoAmostraRounded}{4056\xspace}


Foi retirada uma amostra aleatória simples, sem reposição com \QUATROn elementos, dentre os \QUATRON alunos que possuíam um valor definido para a variável Pagamento. A proporção amostral $\hat{p}$ de alunos cuja fonte de Pagamento são ``Incentivos Federais'' foi de \QUATROpAmostral.

\subsection{}
O intervalo de confiança para a a estatística $\hat{p}$ é dado pela \autoref{eq:quatro-expr}.

\begin{align} 
	IC(p, 95\%) \label{eq:quatro-expr}
	            &= \hat{p} \pm z_\gamma \sigma_{\hat{p}} \\
	            \label{eq:quatro-estim}
	            &= \hat{p} \pm z_\gamma \sqrt{\frac{\hat{p}(1-\hat{p})}{n}} \\
	            &= \QUATROpAmostral \pm \sqrt{\frac{\QUATROpAmostral (1- \QUATROpAmostral)}{\QUATROn}} \nonumber \\
	            &= \QUATROpAmostral \pm \QUATROAdelta \nonumber \\
	            \label{eq:quatro-a-result}
	            &= [\QUATROAICinf, \QUATROAICsup]
\end{align}

Dada uma amostra aleatória simples de \QUATROn alunos tomados dentre os \QUATRON da população, há 95\% de chance de que a proporção de alunos cuja fonte de recursos são ``Incentivos Federais'' seja um valor contido no intervalo \eqref{eq:quatro-a-result}.

Em \eqref{eq:quatro-expr}, $\sigma_{\hat{p}}$ simboliza o desvio padrão de $\hat{p}$ em todas as amostras aleatórias que podem ser tomadas da população. Como não possuímos tal parâmetro, e como $n$ é grande ($n = \QUATROn \geq 50$), $\sigma_{\hat{p}}$ foi aproximado por $s_{\hat{p}}$, o desvio padrão da amostra tomada.

\subsection{(item b)}
Lorem ipsum...

\subsection{(item c)}
Lorem ipsum...