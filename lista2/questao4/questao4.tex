\newcommand{\TRESicmin}{1.85\xspace}
\newcommand{\TRESicmax}{3.10\xspace}
\newcommand{\TRESicNoveNoveMin}{1.62\xspace}
\newcommand{\TRESicNoveNoveMax}{3.33\xspace}
\newcommand{\TRESnZero}{26\xspace}


Foi retirada uma amostra aleatória simples, sem reposição com \QUATROn elementos, dentre os \QUATRON alunos que possuíam um valor definido para a variável Pagamento. A proporção amostral $\hat{p}$ de alunos cuja fonte de Pagamento são ``Incentivos Federais'' foi de \QUATROpAmostral.

\subsection{}
O intervalo de confiança para a a estatística $\hat{p}$ é dado pela \autoref{eq:quatro-expr}.

\begin{align} 
	IC(p, 95\%) \label{eq:quatro-expr}
	            &= \hat{p} \pm z_\gamma \sigma_{\hat{p}} \\
	            \label{eq:quatro-estim}
	            &= \hat{p} \pm z_\gamma \sqrt{\frac{\hat{p}(1-\hat{p})}{n}} \\
	            &= \QUATROpAmostral \pm \sqrt{\frac{\QUATROpAmostral (1- \QUATROpAmostral)}{\QUATROn}} \nonumber \\
	            &= \QUATROpAmostral \pm \QUATROAdelta \nonumber \\
	            \label{eq:quatro-a-result}
	            &= [\QUATROAICinf, \QUATROAICsup]
\end{align}

Dada uma amostra aleatória simples de \QUATROn alunos tomados dentre os \QUATRON da população, há 95\% de chance de que a proporção de alunos cuja fonte de recursos são ``Incentivos Federais'' seja um valor contido no intervalo \eqref{eq:quatro-a-result}.

Em \eqref{eq:quatro-expr}, $\sigma_{\hat{p}}$ simboliza o desvio padrão de $\hat{p}$ em todas as amostras aleatórias que podem ser tomadas da população. Como não possuímos tal parâmetro, e como $n$ é grande ($n = \QUATROn \geq 50$), $\sigma_{\hat{p}}$ foi aproximado por $s_{\hat{p}}$, o desvio padrão da amostra tomada.

\subsection{(item b)}
Lorem ipsum...

\subsection{(item c)}
Lorem ipsum...