\newcommand{\QUATROpAmostral}{\num{0.2700}\xspace}
\newcommand{\QUATROn}{200\xspace}
\newcommand{\QUATROy}{\num{54.0000}\xspace}
\newcommand{\QUATROyLinha}{\num{54.5000}\xspace}
\newcommand{\QUATROz}{\num{2.5633}\xspace}
\newcommand{\QUATROpValue}{\num{0.0052}\xspace}
\newcommand{\QUATROesVinte}{\num{0.0000}\xspace}
\newcommand{\QUATROesVinteUm}{\num{0.0248}\xspace}
\newcommand{\QUATROesVinteDois}{\num{0.0491}\xspace}
\newcommand{\QUATROesVinteTres}{\num{0.0731}\xspace}
\newcommand{\QUATROesVinteQuatro}{\num{0.0967}\xspace}
\newcommand{\QUATROesVinteCinco}{\num{0.1199}\xspace}
\newcommand{\QUATROesVinteSeis}{\num{0.1428}\xspace}
\newcommand{\QUATROesVinteSete}{\num{0.1655}\xspace}
\newcommand{\QUATROpVinte}{\num{0.0100}\xspace}
\newcommand{\QUATROpVinteUm}{\num{0.0241}\xspace}
\newcommand{\QUATROpVinteDois}{\num{0.0514}\xspace}
\newcommand{\QUATROpVinteTres}{\num{0.0980}\xspace}
\newcommand{\QUATROpVinteQuatro}{\num{0.1687}\xspace}
\newcommand{\QUATROpVinteCinco}{\num{0.2641}\xspace}
\newcommand{\QUATROpVinteSeis}{\num{0.3797}\xspace}
\newcommand{\QUATROpVinteSete}{\num{0.5057}\xspace}
\newcommand{\QUATROesAmostra}{\num{0.0731}\xspace}
\newcommand{\QUATROtamanhoAmostra}{\num{4055.1080}\xspace}
\newcommand{\QUATROtamanhoAmostraRounded}{4056\xspace}


	Foi retirada uma amostra aleatória simples, sem reposição, com \QUATROn
	elementos, dentre os \QUATRON alunos que possuíam um valor definido para a
	variável Pagamento. A proporção amostral $\hat{p}$ de alunos cuja fonte de
	Pagamento era ``Incentivos Federais'' foi de \QUATROpAmostral.

\subsection{Intervalo de confiança}

	Como $\sigma_{\hat{p}}$ não é conhecido e $n$ é suficientemente grande,
	o intervalo de confiança para a proporção $p$ foi calculado pela
	\autoref{equation: intervalo de confianca para proporcao 2}  
    aproximando $\sigma_{\hat{p}}$ por $s_{\hat{p}}$.

	\begin{align} 
		IC(p, 95\%) &= \QUATROpAmostral \pm \QUATROzy \sqrt{\frac{\QUATROpAmostral (1- \QUATROpAmostral)}{\QUATROn}} \nonumber \\
					&= \QUATROpAmostral \pm \QUATROAdelta \nonumber \\
					\label{eq:quatro-a-result}
					&= [\QUATROAICinf, \QUATROAICsup]
	\end{align}

	Dada uma amostra aleatória simples de \QUATROn alunos tomados dentre os
	\QUATRON da população, há 95\% de chance de que a proporção amostral $\hat{p}$ 
    de alunos cuja fonte de recursos são ``Incentivos Federais'' seja um valor 
    contido no intervalo \eqref{eq:quatro-a-result}.

\subsection{Precisão}

	Combinando a \autoref{equation: erro amostral maximo} e a \autoref{equation: tamanho amostral maximo} e
	substituindo os valores, o erro amostral máximo de uma amostra com
	\QUATROn elementos é dado por \eqref{eq:quatro-b-e0-calc}.

	\begin{align}
		E_0 &= \label{eq:quatro-b-e0-calc}
               \sqrt{\frac{\QUATROzy \;\cdot\; \QUATROpAmostral (1 - \QUATROpAmostral) }{\frac{\QUATROn - \QUATROn \;\cdot\; \QUATRON}{\QUATROn - \QUATRON}}}  \\
			&= \QUATROBE \nonumber
	\end{align}

	Como $E_0 = \QUATROBE$, a amostra não é suficiente para uma margem de
	erro de 2\%. Usando a \autoref{equation: tamanho amostra 1} obtemos o valor de $n_0$ 
    necessário para uma margem de erro de 2\% em \eqref{eq:quatro-b-n0-result}.

	\begin{align}
		n_0 &= \frac{\QUATROzy^2 \cdot \QUATROpAmostral(1-\QUATROpAmostral)}{\num{0.02}^2} \nonumber \\
			&= \label{eq:quatro-b-n0-result}
			   \QUATROBnz	
	\end{align}

	Como o tamanho da população é conhecido, o valor em
	\eqref{eq:quatro-b-n0-result} pode ser reduzido para o valor em
	\eqref{eq:quatro-b-n-result} aplicando a \autoref{equation: tamanho amostra 2}.

	\begin{align}
		n &= \Big\lceil \frac{\QUATRON \cdot  \QUATROBnz}{\QUATRON + \QUATROBnz - 1} \nonumber \Big\rceil \\
		  &= \lceil \QUATROBn \nonumber \rceil \\
		  &= \label{eq:quatro-b-n-result} \QUATROBnceil
	\end{align}

\subsection{Tamanho da amostra sem amostra piloto}

	Para estimar o tamanho da amostra que garanta erro amostral máximo de
	2\% sem uma estimativa adequada para a proporção populacional $p$, deve
	ser utilizada a fórmula \autoref{equation: tamanho amostra 1} para obtenção de
	$n_0$. A fórmula em \eqref{eq:quatro-c-n0-expr} foi obtida
	superestimando $p$ como $\num{0.5}$.

	\begin{align}
		n_0 &= \label{eq:quatro-c-n0-expr}
               \frac{z_\gamma^2 \cdot \num{0.5} \cdot (1 - \num{0.5})}{E_0^2} = \frac{z_\gamma^2}{4 \cdot E_0^2} \\
			&= \frac{\QUATROzy^2}{4 \cdot (\num{0.02})^2} \nonumber \\
			&= \label{eq:quatro-c-n0-result}
			   \QUATROCnz
	\end{align}

	Novamente, \eqref{eq:quatro-c-n0-result} pode ser substituído na 
    \autoref{equation: tamanho amostra 2}, resultando no tamanho da amostra necessário, em
	\eqref{eq:quatro-c-n-result}.

	\begin{align}
		n &= \Big\lceil \frac{\QUATRON \cdot \QUATROCnz}{\QUATRON + \QUATROCnz -1} \nonumber \Big\rceil \\
		  &= \lceil \QUATROCn \nonumber \rceil \\
		  &= \label{eq:quatro-c-n-result} 
			 \QUATROCnceil
	\end{align}

	Portanto, considerando a ausência de uma amostra piloto, é preciso ter
	uma amostra de tamanho mínimo de \QUATROCnceil para estimar com 95\% de
	confiança e precisão de 2\% a proporção populacional de alunos de EAD da
	TYU que usam incentivos federais.
