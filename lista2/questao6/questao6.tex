\newcommand{\TRESicmin}{1.85\xspace}
\newcommand{\TRESicmax}{3.10\xspace}
\newcommand{\TRESicNoveNoveMin}{1.62\xspace}
\newcommand{\TRESicNoveNoveMax}{3.33\xspace}
\newcommand{\TRESnZero}{26\xspace}


Para avaliação da variável Renda, primeiramente foi feita uma recodificação
da mesma em uma variável qualitativa de dois valores: alunos com renda
familiar inferior a $\num{2,5}$ salários mínimos e alunos com renda igual ou
superior a $\num{2,5}$ salários mínimos. Nesse processo, apenas foram considerados
aqueles alunos que possuíam um valor definido para a variável, totalizando
$N=\SEISN$ alunos. Em seguida, retirou-se uma amostra aleatória simples de
\SEISn elementos dessa população. A proporção amostral $\hat{p}$ de alunos
com renda superior a $\num{2.5}$ salários mínimos é de \SEISpAmostral.

\subsection{Intervalo de Confiança}

	\todo[inline]{O intervalo de confiança para a \hl{proporção $p$} é (...)}
	Como $\sigma_{\hat{p}}$ não é conhecido e $n$ é suficientemente grande,
	o intervalo de confiança para a proporção $p$ foi calculado pela
	\autoref{equation: intervalo de confianca para proporcao 2}.
	%
	\begin{align*}
		IC(p, 95\%) &= \SEISpAmostral \pm \SEISzy \sqrt{\frac{\SEISpAmostral (1- \SEISpAmostral)}{\SEISn}} \nonumber \\
					&= \SEISpAmostral \pm \SEISAdelta \nonumber \\
					&= [\SEISAICinf, \SEISAICsup]
	\end{align*}

	\noindent Dada uma amostra aleatória simples de \SEISn alunos
	selecionados dentre os \SEISN da população há $95\%$ de chance de que a
	proporção de alunos que possuem renda familiar superior a $\num{2,5}$ salários
	mínimos seja um valor no intervalo $[\SEISAICinf, \SEISAICsup]$

\subsection{Precisão}

	Combinando \eqref{eq:seis-b-e0-expr} e \eqref{eq:seis-b-n0-expr} e
	substituindo os valores, o erro amostral máximo de uma amostra com
	\SEISn elementos é dado por \eqref{eq:seis-b-e0-calc}.

	\begin{align}
		\label{eq:seis-b-e0-calc}
		E_0 &= \sqrt{\frac{z_\gamma \hat{p}(1 - \hat{p}) }{\frac{n-n N}{n - N}}} \\
			&= \sqrt{\frac{\SEISzy \;\cdot\; \SEISpAmostral (1 - \SEISpAmostral) }{\frac{\SEISn - \SEISn \;\cdot\; \SEISN}{\SEISn - \SEISN}}} \nonumber \\
			&= \SEISBE \nonumber
	\end{align}

	Como $E_0 = \SEISBE$, a amostra é suficientemente grande para se afirmar
	uma margem de erro de $2\%$.

\subsection{Tamanho da Amostra sem Amostra-Piloto}

	Para estimar o tamanho da amostra que garanta erro amostral máximo de 2\%
	sem uma estimativa adequada para a proporção populacional $p$, deve ser
	utilizada a fórmula \eqref{eq:seis-c-n0-expr} para obtenção de $n_0$. A
	fórmula em \eqref{eq:seis-c-n0-expr} foi obtida superestimando $p$ como
	$\num{0.5}$ e aplicando isso em \eqref{eq:seis-b-n0}

	\begin{align}
		n_0 &= \label{eq:seis-c-n0-expr}
			   \frac{z_\gamma^2 \cdot \num{0.5} \cdot (1 - \num{0.5})}{E_0^2} = \frac{z_\gamma^2}{4 \cdot E_0^2} \\
			&= \frac{\SEISzy^2}{4 \cdot (\num{0.02})^2} \nonumber \\
			&= \label{eq:seis-c-n0-result}
			   \SEISCnz
	\end{align}

	Novamente, \eqref{eq:seis-c-n0-result} pode ser substituído em
	\eqref{eq:seis-b-n}, resultando no tamanho da amostra necessário, em
	\eqref{eq:seis-c-n-result}.

	\begin{align}
		n &= \Big\lceil \frac{N n_0}{N + n_0 - 1} \nonumber \Big\rceil \\
		  &= \Big\lceil \frac{\SEISN \cdot \SEISCnz}{\SEISN + \SEISCnz - 1} \Big\rceil \nonumber \\
		  &= \lceil \SEISCn \rceil \nonumber \\
		  &= \label{eq:seis-c-n-result} 
			 \SEISCnceil
	\end{align}

	Portanto, considerando a ausência de uma amostra piloto, é preciso ter uma
	amostra de tamanho mínimo de \SEISCnceil para estimar com 95\% de
	confiança e precisão de 2\% a proporção populacional de alunos de EAD da TYU
	apresentam renda familiar superior a $\num{2,5}$ salários mínimos.

\subsection{Impactos na Recodificação da Variável}
	
	A recodificação de uma varíavel quantitativa em uma qualitativa sempre
	resulta em perda de informação. Por isso, para a análise do intervalo de
	confiança para o parâmetro média da variável \textit{Renda}, o mais
	indicado seria que a estimação fosse feita com a variável original.

	No entanto, é interessante ressaltar que o intervalo de confiança se
	aplica ao parâmetro sendo estimado. Portanto, caso o intuito seja
	estudar o parâmetro proporção, a recodificação se faz necessária para a
	variável \textit{Renda}.
