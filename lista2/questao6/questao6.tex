\newcommand{\TRESicmin}{1.85\xspace}
\newcommand{\TRESicmax}{3.10\xspace}
\newcommand{\TRESicNoveNoveMin}{1.62\xspace}
\newcommand{\TRESicNoveNoveMax}{3.33\xspace}
\newcommand{\TRESnZero}{26\xspace}


Para avaliação da variável Renda, primeiramente foi feita uma recodificação
da mesma em uma variável quantiatativa de dois valores: alunos com renda
familiar inferior a $2,5$ salários mínimos e alunos com renda igual ou
superior a $2,5$ salários mínimos. Nesse processo, apenas foram considerados
aqueles alunos que possuíam um valor definido para a variável, totalizando
$N=\SEISN$ alunos. Em seguida, retirou-se uma amostra aleatório simples de
\SEISn elementos dessa população. A proporção amostral $\hat{p}$ de alunos
com renda superior a $2.5$ salários mínimos é de \SEISpAmostral.

\subsection{Intervalo de Confiança}

	O intervalo de confiança para a a estatística $\hat{p}$ é dado por:
	%
	\begin{align*}
		IC(p, 95\%) &= \hat{p} \pm z_\gamma \sigma_{\hat{p}} \\
					&= \hat{p} \pm z_\gamma \sqrt{\frac{\hat{p}(1-\hat{p})}{n}} \\
					&= \SEISpAmostral \pm \SEISzy \sqrt{\frac{\SEISpAmostral (1- \SEISpAmostral)}{\SEISn}} \nonumber \\
					&= \SEISpAmostral \pm \SEISAdelta \nonumber \\
					&= [\SEISAICinf, \SEISAICsup]
	\end{align*}

	\noindent Dada uma amostra aleatória simples de \SEISn alunos
	selecionados dentre os \SEISN da população há $95\%$ de chance de que a
	proporção de alunos que possuem renda familiar superior a $2,5$ salários
	mínimos seja um valor no intervalo $[\SEISAICinf, \SEISAICsup]$

\subsection{(item b)}
Lorem ipsum...

\subsection{(item c)}
Lorem ipsum...
