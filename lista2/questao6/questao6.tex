\newcommand{\QUATROpAmostral}{\num{0.2700}\xspace}
\newcommand{\QUATROn}{200\xspace}
\newcommand{\QUATROy}{\num{54.0000}\xspace}
\newcommand{\QUATROyLinha}{\num{54.5000}\xspace}
\newcommand{\QUATROz}{\num{2.5633}\xspace}
\newcommand{\QUATROpValue}{\num{0.0052}\xspace}
\newcommand{\QUATROesVinte}{\num{0.0000}\xspace}
\newcommand{\QUATROesVinteUm}{\num{0.0248}\xspace}
\newcommand{\QUATROesVinteDois}{\num{0.0491}\xspace}
\newcommand{\QUATROesVinteTres}{\num{0.0731}\xspace}
\newcommand{\QUATROesVinteQuatro}{\num{0.0967}\xspace}
\newcommand{\QUATROesVinteCinco}{\num{0.1199}\xspace}
\newcommand{\QUATROesVinteSeis}{\num{0.1428}\xspace}
\newcommand{\QUATROesVinteSete}{\num{0.1655}\xspace}
\newcommand{\QUATROpVinte}{\num{0.0100}\xspace}
\newcommand{\QUATROpVinteUm}{\num{0.0241}\xspace}
\newcommand{\QUATROpVinteDois}{\num{0.0514}\xspace}
\newcommand{\QUATROpVinteTres}{\num{0.0980}\xspace}
\newcommand{\QUATROpVinteQuatro}{\num{0.1687}\xspace}
\newcommand{\QUATROpVinteCinco}{\num{0.2641}\xspace}
\newcommand{\QUATROpVinteSeis}{\num{0.3797}\xspace}
\newcommand{\QUATROpVinteSete}{\num{0.5057}\xspace}
\newcommand{\QUATROesAmostra}{\num{0.0731}\xspace}
\newcommand{\QUATROtamanhoAmostra}{\num{4055.1080}\xspace}
\newcommand{\QUATROtamanhoAmostraRounded}{4056\xspace}


Para avaliação da variável Renda, primeiramente foi feita uma recodificação
da mesma em uma variável quantiatativa de dois valores: alunos com renda
familiar inferior a $2,5$ salários mínimos e alunos com renda igual ou
superior a $2,5$ salários mínimos. Nesse processo, apenas foram considerados
aqueles alunos que possuíam um valor definido para a variável, totalizando
$N=\SEISN$ alunos. Em seguida, retirou-se uma amostra aleatório simples de
\SEISn elementos dessa população. A proporção amostral $\hat{p}$ de alunos
com renda superior a $2.5$ salários mínimos é de \SEISpAmostral.

\subsection{Intervalo de Confiança}

	O intervalo de confiança para a a estatística $\hat{p}$ é dado por:
	%
	\begin{align*}
		IC(p, 95\%) &= \hat{p} \pm z_\gamma \sigma_{\hat{p}} \\
					&= \hat{p} \pm z_\gamma \sqrt{\frac{\hat{p}(1-\hat{p})}{n}} \\
					&= \SEISpAmostral \pm \SEISzy \sqrt{\frac{\SEISpAmostral (1- \SEISpAmostral)}{\SEISn}} \nonumber \\
					&= \SEISpAmostral \pm \SEISAdelta \nonumber \\
					&= [\SEISAICinf, \SEISAICsup]
	\end{align*}

	\noindent Dada uma amostra aleatória simples de \SEISn alunos
	selecionados dentre os \SEISN da população há $95\%$ de chance de que a
	proporção de alunos que possuem renda familiar superior a $2,5$ salários
	mínimos seja um valor no intervalo $[\SEISAICinf, \SEISAICsup]$

\subsection{(item b)}
Lorem ipsum...

\subsection{(item c)}
Lorem ipsum...
