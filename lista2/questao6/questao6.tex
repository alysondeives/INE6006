\newcommand{\TRESicmin}{1.85\xspace}
\newcommand{\TRESicmax}{3.10\xspace}
\newcommand{\TRESicNoveNoveMin}{1.62\xspace}
\newcommand{\TRESicNoveNoveMax}{3.33\xspace}
\newcommand{\TRESnZero}{26\xspace}


Para avaliação da variável Renda, primeiramente foi feita uma recodificação
da mesma em uma variável qualitativa de dois valores: alunos com renda
familiar inferior a $\num{2,5}$ salários mínimos e alunos com renda igual ou
superior a $\num{2,5}$ salários mínimos. Nesse processo, apenas foram considerados
aqueles alunos que possuíam um valor definido para a variável, totalizando
$N=\SEISN$ alunos. Em seguida, retirou-se uma amostra aleatória simples de
\SEISn elementos dessa população. A proporção amostral $\hat{p}$ de alunos
com renda superior a $\num{2.5}$ salários mínimos é de \SEISpAmostral.

\subsection{Intervalo de Confiança}

	Como $\sigma_{\hat{p}}$ não é conhecido e $n$ é suficientemente grande,
	o intervalo de confiança para a proporção $p$ foi calculado pela
	\autoref{equation: intervalo de confianca para proporcao 2}.
	%
	\begin{align*}
		IC(p, 95\%) &= \SEISpAmostral \pm \SEISzy \sqrt{\frac{\SEISpAmostral (1- \SEISpAmostral)}{\SEISn}} \\
		IC(p, 95\%) &= \SEISpAmostral \pm \SEISAdelta \\
		IC(p, 95\%) &= [\SEISAICinf, \SEISAICsup]
	\end{align*}

	Dada uma amostra aleatória simples de \SEISn alunos
	selecionados dentre os \SEISN da população há $95\%$ de chance de que a
	proporção de alunos que possuem renda familiar superior a $\num{2,5}$ salários
	mínimos seja um valor no intervalo $[\SEISAICinf, \SEISAICsup]$

\subsection{Precisão}

	O erro amostral máximo de uma amostra com \SEISn elementos é dado pela
	Equação \ref{equation: erro amostral maximo}.
	%
	\begin{align*}
		E_0 &= \sqrt{\frac{\SEISzy \;\cdot\; \SEISpAmostral (1 - \SEISpAmostral) }{\frac{\SEISn - \SEISn \;\cdot\; \SEISN}{\SEISn - \SEISN}}} \\
		E_0 &= \SEISBE
	\end{align*}

	Como $E_0 = \SEISBE$, a amostra é suficientemente grande para se afirmar
	um erro de $2\%$.

\subsection{Tamanho da Amostra sem Amostra-Piloto}

	Para se estimar o tamanho da amostra que um garanta erro amostral máximo de
	2\% sem uma estimativa adequada para a proporção populacional $p$, é
	feita uma super estimação da Equação \ref{equation: tamanho amostral
	maximo} com $p = 0.5$.
	%
	\begin{align*}
		n_0 &= \frac{z_\gamma^2 \cdot \num{0.5} \cdot (1 - \num{0.5})}{E_0^2} \\
		n_0 &= \frac{z_\gamma^2}{4 \cdot E_0^2} \\
		n_0 &= \frac{\SEISzy^2}{4 \cdot (\num{0.02})^2} \\
		n_0 &= \SEISCnz
	\end{align*}

	Usando esse valor para $n_0$ superestimado, pode-se calcular o tamanho da
	amostra necessária pela Equação \ref{equation: tamanho amostra 2}.
	%
	\begin{align*}
		n &= \Big\lceil \frac{\SEISN \cdot \SEISCnz}{\SEISN + \SEISCnz - 1} \Big\rceil \\
		n &= \lceil \SEISCn \rceil \\
		n &= \SEISCnceil
	\end{align*}

	Portanto, considerando a ausência de uma amostra piloto, é preciso ter uma
	amostra de tamanho mínimo de \SEISCnceil para estimar com 95\% de
	confiança e precisão de 2\% a proporção populacional de alunos de EAD da TYU
	apresentam renda familiar superior a $\num{2,5}$ salários mínimos.

\subsection{Impactos na Recodificação da Variável}
	
	A recodificação de uma varíavel quantitativa em uma qualitativa sempre
	resulta em perda de informação. Por isso, para a análise do intervalo de
	confiança para o parâmetro média da variável \textit{Renda}, o mais
	indicado seria que a estimação fosse feita com a variável original.

	No entanto, é interessante ressaltar que o intervalo de confiança se
	aplica ao parâmetro sendo estimado. Portanto, caso o intuito seja
	estudar o parâmetro proporção, a recodificação se faz necessária para a
	variável \textit{Renda}.
