\newcommand{\TRESicmin}{1.85\xspace}
\newcommand{\TRESicmax}{3.10\xspace}
\newcommand{\TRESicNoveNoveMin}{1.62\xspace}
\newcommand{\TRESicNoveNoveMax}{3.33\xspace}
\newcommand{\TRESnZero}{26\xspace}


\todo[inline]{``recodificação da mesma em uma variável \st{quantiatativa}\hl{qualitativa} de dois valores''}
Para avaliação da variável Renda, primeiramente foi feita uma recodificação
da mesma em uma variável quantiatativa de dois valores: alunos com renda
familiar inferior a $\num{2,5}$ salários mínimos e alunos com renda igual ou
superior a $\num{2,5}$ salários mínimos. Nesse processo, apenas foram considerados
aqueles alunos que possuíam um valor definido para a variável, totalizando
$N=\SEISN$ alunos. Em seguida, retirou-se uma amostra aleatória simples de
\SEISn elementos dessa população. A proporção amostral $\hat{p}$ de alunos
com renda superior a $\num{2.5}$ salários mínimos é de \SEISpAmostral.

\subsection{Intervalo de Confiança}

	\todo[inline]{O intervalo de confiança para a \hl{proporção $p$} é (...)}
	O intervalo de confiança para a a estatística $\hat{p}$ é dado por:
	%
	\begin{align*}
		IC(p, 95\%) &= \hat{p} \pm z_\gamma \sigma_{\hat{p}} \\
					&= \hat{p} \pm z_\gamma \sqrt{\frac{\hat{p}(1-\hat{p})}{n}} \\
					&= \SEISpAmostral \pm \SEISzy \sqrt{\frac{\SEISpAmostral (1- \SEISpAmostral)}{\SEISn}} \nonumber \\
					&= \SEISpAmostral \pm \SEISAdelta \nonumber \\
					&= [\SEISAICinf, \SEISAICsup]
	\end{align*}

	\noindent Dada uma amostra aleatória simples de \SEISn alunos
	selecionados dentre os \SEISN da população há $95\%$ de chance de que a
	proporção de alunos que possuem renda familiar superior a $\num{2,5}$ salários
	mínimos seja um valor no intervalo $[\SEISAICinf, \SEISAICsup]$

\subsection{Precisão}
	
	É possível calcular o tamanho da amostra necessária para que $|\hat{p} -
	E(\hat{p})| \leq E_0$ aplicando as seguintes equações:
	%
	\begin{align}
		n_0 &= \label{eq:seis-b-n0}
			   \frac{z_\gamma^2 p(1-p)}{E_0^2} \\
		n &= \label{eq:seis-b-n}
			 \frac{N n_0}{N + n_0 - 1} 
	\end{align}

	Para obter o erro amostral máximo, dado o tamanho da amostra, as equações
	\eqref{eq:seis-b-e0-expr} e \eqref{eq:seis-b-n0-expr} podem ser usadas. 

	Sabemos $n$, $N$ $z\gamma$ e $\hat{p}$, que pode ser usado para aproximar
	$p$. Para encontrar $E_0$, podemos deduzir \eqref{eq:seis-b-e0-expr}
	algebricamente de \eqref{eq:seis-b-n0}. Para encontrar $n_0$, usado em
	\eqref{eq:seis-b-e0-expr}, $n_0$ pode ser isolada em
	\eqref{eq:seis-b-n}, resultando em \eqref{eq:seis-b-n0-expr}. Para tal
	foi utilizada a ferramenta
	WolframAlpha\footnote{\url{https://www.wolframalpha.com/input/?i=solve+n\%3D(N*m)\%2F(N\%2Bm-1)+for+m}}.

	\begin{align}
		E_0 &= \label{eq:seis-b-e0-expr}
			   \sqrt{\frac{z_\gamma p(1 - p) }{n_0}} \\
		n_0 &= \label{eq:seis-b-n0-expr}
			   \frac{n-n N}{n-N}
	\end{align}

	Combinando \eqref{eq:seis-b-e0-expr} e \eqref{eq:seis-b-n0-expr} e
	substituindo os valores, o erro amostral máximo de uma amostra com
	\SEISn elementos é dado por \eqref{eq:seis-b-e0-calc}.

	\begin{align}
		\label{eq:seis-b-e0-calc}
		E_0 &= \sqrt{\frac{z_\gamma \hat{p}(1 - \hat{p}) }{\frac{n-n N}{n - N}}} \\
			&= \sqrt{\frac{\SEISzy \;\cdot\; \SEISpAmostral (1 - \SEISpAmostral) }{\frac{\SEISn - \SEISn \;\cdot\; \SEISN}{\SEISn - \SEISN}}} \nonumber \\
			&= \SEISBE \nonumber
	\end{align}

	Como $E_0 = \SEISBE$, a amostra é suficientemente grande para seafirmar
	uma margem de erro de $2\%$.

\subsection{Tamanho da Amostra sem Amostra-Piloto}

	Para estimar o tamanho da amostra que garanta erro amostral máximo de 2\%
	sem uma estimativa adequada para a proporção populacional $p$, deve ser
	utilizada a fórmula \eqref{eq:seis-c-n0-expr} para obtenção de $n_0$. A
	fórmula em \eqref{eq:seis-c-n0-expr} foi obtida superestimando $p$ como
	$\num{0.5}$ e aplicando isso em \eqref{eq:seis-b-n0}

	\begin{align}
		n_0 &= \label{eq:seis-c-n0-expr}
			   \frac{z_\gamma^2 \cdot \num{0.5} \cdot (1 - \num{0.5})}{E_0^2} = \frac{z_\gamma^2}{4 \cdot E_0^2} \\
			&= \frac{\SEISzy^2}{4 \cdot (\num{0.02})^2} \nonumber \\
			&= \label{eq:seis-c-n0-result}
			   \SEISCnz
	\end{align}

	Novamente, \eqref{eq:seis-c-n0-result} pode ser substituído em
	\eqref{eq:seis-b-n}, resultando no tamanho da amostra necessário, em
	\eqref{eq:seis-c-n-result}.

	\begin{align}
		n &= \Big\lceil \frac{N n_0}{N + n_0 - 1} \nonumber \Big\rceil \\
		  &= \Big\lceil \frac{\SEISN \cdot \SEISCnz}{\SEISN + \SEISCnz - 1} \Big\rceil \nonumber \\
		  &= \lceil \SEISCn \rceil \nonumber \\
		  &= \label{eq:seis-c-n-result} 
			 \SEISCnceil
	\end{align}

	Portanto, considerando a ausência de uma amostra piloto, é preciso ter uma
	amostra de tamanho mínimo de \SEISCnceil para estimar com 95\% de
	confiança e precisão de 2\% a proporção populacional de alunos de EAD da TYU
	apresentam renda familiar superior a $\num{2,5}$ salários mínimosi.

\subsection{Impactos na Recodificação da Variável}

	A escolha da recodificação ou não da variável renda depende do parâmetro
	sendo estudado, e não da estimação do intervalo de confiança. Caso o
	intuito seja estudar o parâmetro proporção, a recodificação se faz
	necessária para a variável \textit{Renda} e não implicará em perda de
	informação.  No entanto, caso o objetivo seja estudar os parâmetros
	média e variância, uma recodificação impactaria em perda de informação.
	Em ambos os cenários -- estudo do parâmetro proporção e estudo da
	média/variância -- a estimação do intervalo de confiança não seria
	comprometida.
