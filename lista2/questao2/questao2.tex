\subsection{Intervalo de 95\% de confiança para a média populacional da Idade dos alunos}
\label{sub:1a}
Após retirar os dados perdidos da variável Idade, foi retirada uma amostra aleatória de 20 idades, por meio de sorteio. Com a amostra adquirida, tem-se média $\bar{x}$ = 31,85 e desvio padrão s 
= 6,11534. Com o desvio padrão da população desconhecido, utiliza-se o desvio padrão da amostra (s = 6,11534). O tamanho da população é conhecido (N = 4987) e o da amostra é pequeno, n < 30, então, 
nesse caso utiliza-se a \textit{distribuição t de Student}.

Para obter o intervalo de confiança para a média em uma população conhecida, é possível utilizar a seguinte equação:

\begin{equation*}
 IC (\mu, \gamma) = \bar{x} \pm t_\gamma \frac{s}{\sqrt{n}} \sqrt{\frac{N-n}{N-1}}
\end{equation*}

Para um intervalo de confiança de 95\%, obtém-se o valor da cauda superior 0,025. Diante disso, procura-se na tabela \textit{distribuição t de student} o grau de liberdade 19, pois gl = n-1, e o 
valor 0,025, resultando em \textit{t = 2,093}. Inserindo os valores na equação tem-se:

\begin{equation*}
 IC (\mu, 95\%) = 31,85 \pm 2,093 \frac{6,11534}{\sqrt{20}} \sqrt{\frac{4987 - 20}{4987 - 1}} 
                = 31,85 \pm 2,8565
\end{equation*}

A média de idades de amostra é 31,85 anos, com o nível de confiança de 95\%, a margem de erro é de 2,8565 anos para mais ou para menos. Diante disso, o intervalo de confiança é de [28,9935; 34,7065].

\subsection{Intervalo de 99\% de confiança para a média populacional da Idade dos alunos, com uma precisão de 2 anos}

Com o desvio padrão da população desconhecido, utiliza-se como amostra piloto n = 20 e desvio padrão s = 6,11534, obtidos anteriormente. Como o tamanho da amostra é pequeno (n < 30) utiliza-se a 
\textit{distribuição t student}.

Para saber o tamanho da amostra necessário para estimar a média populacional da idade dos alunos, com erro amostral máximo tolerado de 2 anos, utiliza-se a seguinte equação:

\begin{equation*}
  n_0 = \left (\frac{t_\gamma s}{E_0} \right)^2
\end{equation*}

Sabendo-se que o grau de liberdade é 19, pois gl = n - 1, e o nível de confiança é de 99\%, o valor da cauda superior será de 0,005. Obtendo os valores da tabela de \textit{distribuição t student}, t 
= 2,861. Inserindo os valores na equação, tem-se:

\begin{equation*}
 n_0 = \left (\frac{2,861 \times 6,11534}{2} \right)^2 = 76,52 \cong 77
\end{equation*}

Como o tamanho da população não é grande e é conhecido, adicionalmente, utiliza-se a seguinte equação para calcular o tamanho da amostra.

\begin{equation*}
 n = \frac{N . n_0}{N + n_0}
\end{equation*}

Inserindo os valores na equação, tem-se:

\begin{equation*}
 n = \frac{4987 \times 76,52}{4987 + 76,52} = 75,3636 \cong 75
\end{equation*}

Diante dos cálculos apresentados acima, pode-se dizer que uma amostra com 20 elementos não é suficiente para encontrar um intervalo de 99\% de confiança para a média populacional da idade dos alunos 
com um erro máximo tolerado de 2 anos. Seria necessário, no mínimo 77 elementos da amostra para garantir certa precisão. 

\subsection{Você concorda com o plano de amostragem usado, que considerou a população homogênea}
Com o objetivo de caracterizar melhor o perfil dos alunos através da variável idade, o plano de amostragem utilizado, considerando a população homogênea, não é o ideal, uma vez que a variável idade 
está associada a pelo menos uma outra variável. Por exemplo, existe uma associação entre a variável Idade e a variável Opinião, mostrando que os alunos com idades inferiores a 30 anos possuem maiores 
percentuais (72,57\%) para a opinião \textquotedblleft Muito Satisfeitos\textquotedblright, enquanto os alunos com idades superiores a 30 anos possuem maiores percentuais (25,62\%) para a opinião 
\textquotedblleft Indiferente\textquotedblright, não obtendo flutuação maior do que 5\% para as opiniões \textquotedblleft Satisfeitos\textquotedblright (21,94\%) e \textquotedblleft 
Insatisfeitos\textquotedblright (21,28\%). Além disso, o perfil dos alunos pode ser analisado através da associações entre outras variáveis, como por exemplo, Opinião e Pagamento e Pagamento e 
Região, conforme apresentado na Lista 1. 

Como resultado da associação entre Opinião e Pagamento, observou-se que os alunos que pagam sua mensalidade com Auxílio de Familiares são mais \textquotedblleft Indiferentes\textquotedblright 
(36,96\%), com Bolsas de Estudos são mais \textquotedblleft Indiferentes\textquotedblright (38,72\%), com Financiamento Bancário são mais \textquotedblleft Muito Satisfeitos\textquotedblright 
(66,36\%), com Incentivos Federal são mais \textquotedblleft Insatisfeitos\textquotedblright (35,30\%) e com Recursos Próprios são mais \textquotedblleft Satisfeitos\textquotedblright (34,48\%).

Por outro lado, a associação entre as variáveis Pagamento e Região apresentou o seguinte resultado: os alunos de Aratibutantã utilizam, maioritariamente, a forma de pagamento por 
\textquotedblleft Incentivos Federais\textquotedblright (47,51\%), os de Baependinha utilizam o \textquotedblleft Financiamento Bancário\textquotedblright (53,01\%), os de Itamaracanã utilizam o 
\textquotedblleft Financiamento Bancário \textquotedblright (90,98\%), os de Jaquereçaba utilizam os \textquotedblleft Incentivos Federais\textquotedblright (76,68\%) e por fim, os alunos de 
Paranapitanga utilizam, maioritariamente, a forma de pagamento \textquotedblleft Incentivos Federais\textquotedblright (95,87\%). 


Para caracterizar melhor o perfil dos alunos, poderia ser realizado um plano de amostragem estratificada proporcional, uma vez que se conhece, neste caso, as características da população (quantos e 
quais são os tamanhos dos estratos). 