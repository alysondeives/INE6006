\newcommand{\TRESicmin}{1.85\xspace}
\newcommand{\TRESicmax}{3.10\xspace}
\newcommand{\TRESicNoveNoveMin}{1.62\xspace}
\newcommand{\TRESicNoveNoveMax}{3.33\xspace}
\newcommand{\TRESnZero}{26\xspace}


	Foi retirada uma amostra aleatória simples sem reposição com \CINCOn
	elementos, dentre os \CINCON alunos que possuíam um valor definido para
	a variável Opinião.  A proporção amostral $\hat{p}$ de alunos com
	opiniões negativas (ou seja, alunos que consideraram ``Insatisfeitos''
	ou ``Muito insatisfeitos'') foi de \CINCOpAmostral.

	\begin{align*} 
		\hat{p}  &= {\frac{\CINCOqtdOpinioesNegativas}{\CINCOn}} = \CINCOpAmostral \nonumber \\
	\end{align*}

\subsection{Intervalo de confiança}

	O intervalo de confiança para a proporção $p$ é dado pela
    \autoref{equation: intervalo de confianca para proporcao 1}, mas como $\sigma_{\hat{p}}$ 
    não é conhecido e $n$ é grande, a 
    \autoref{equation: intervalo de confianca para proporcao 2} pode ser usada.

	\begin{align} 
		IC(p, 95\%) 
					&= \CINCOpAmostral \pm \CINCOzy \sqrt{\frac{\CINCOpAmostral (1- \CINCOpAmostral)}{\CINCOn}} \nonumber \\
		IC(p, 95\%)			&= \CINCOpAmostral \pm \CINCOAdelta \nonumber \\
					\label{eq:cinco-a-result}
		IC(p, 95\%)			&= [\CINCOAICinf, \CINCOAICsup]
	\end{align}

	Portanto, dada uma amostra aleatória simples de \CINCOn alunos tomados
	dentre os \CINCON da população, há 95\% de chance de que a proporção de
	alunos com opiniões negativas seja um valor entre \CINCOAICinf e
	\CINCOAICsup.

\subsection{Precisão}

	Como já apresentado anteriormente, é possível aplicar a fórmula
    definida na \autoref{equation: erro amostral maximo}, de modo a obter
    o valor do erro amostral máximo de uma amostra com \CINCOn elementos.

	\begin{align}
		\label{eq:cinco-b-e0-calc}
		E_0 &= \sqrt{\frac{\CINCOzy \;\cdot\; \CINCOpAmostral (1 - \CINCOpAmostral) }{\frac{\CINCOn - \CINCOn \;\cdot\; \CINCON}{\CINCOn - \CINCON}}} \\
	    E_0		&= \CINCOBE \nonumber
	\end{align}

	Como o valor de $E_0 = \CINCOBE$, a amostra não é suficiente para uma
	margem de erro de 2\%. Aplicando a fórmula \autoref{equation: tamanho amostra 1} já
	mencionada, obtem-se o $n_0$ necessário para uma margem de erro de 2\%
	em \eqref{eq:cinco-b-n0-result}.

	\begin{align}
		n_0 &= \frac{\CINCOzy^2 \cdot \CINCOpAmostral(1-\CINCOpAmostral)}{\num{0.02}^2} \nonumber \\
		n_0	&= \label{eq:cinco-b-n0-result}
			   \CINCOBnz	
	\end{align}

	Como o tamanho da população é conhecido, o valor em
	\eqref{eq:cinco-b-n0-result} pode ainda ser reduzido para o valor obtido
	em \eqref{eq:cinco-b-n-result}.

	\begin{align}
		n &= \lceil \CINCOBn \nonumber \rceil \\
		n  &= \label{eq:cinco-b-n-result} 
			 \CINCOBnceil
	\end{align}

	Pode-se concluir que a amostra de tamanho 200 não é suficiente para um
	intervalo de confiança de 95\% com precisão de 2\%, sendo, portanto,
	neccessário obter uma amostra de tamanho mínimo de \CINCOBnceil.

\subsection{Tamanho da amostra sem amostra piloto}

	Para estimar o tamanho da amostra que garanta erro amostral máximo de
	$2\%$ com um nível de confiança de $95\%$ e sem uma estimativa adequada
	para a proporção populacional $p$, foi utilizada a fórmula da
	\autoref{equation: tamanho amostra 1} para obtenção de $n_0$.  
	De modo análogo à questão 4, o valor de $p$ foi superestimado para $\num{0.5}$ 
	aplicando na \autoref{equation: tamanho amostra 1} supracitada.

	\begin{align}
		n_0 &= \frac{\CINCOzy^2}{4 \cdot (\num{0.02})^2} \nonumber \\
		n_0	&= \label{eq:cinco-c-n0-result}
			   \CINCOCnz
	\end{align}

	Em seguida, como o tamanho da população é conhecido, o resultado obtido
	em \eqref{eq:cinco-c-n0-result} pode ser substituído na fórmula da
	\autoref{equation: tamanho amostra 2}, resultando em \eqref{eq:cinco-c-n-result}, o qual
	representa o tamanho da amostra necessário.

	\begin{align}
		n &= \Big\lceil \frac{\CINCON \cdot \CINCOCnz}{\CINCON + \CINCOCnz - 1}
		  \nonumber \Big\rceil \\
		n  &= \lceil \CINCOCn \nonumber \rceil\\
		n  &= \label{eq:cinco-c-n-result} 
			 \CINCOCnceil
	\end{align}

	Portanto, considerando a ausência de uma amostra piloto, é preciso ter
	uma amostra de tamanho mínimo de \CINCOCnceil para estimar com $95\%$ de
	confiança e precisão de $2\%$ a proporção populacional de alunos com
	opiniões negativas sobre a EAD da TYU.
