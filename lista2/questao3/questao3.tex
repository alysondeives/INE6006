\newcommand{\QUATROpAmostral}{\num{0.2700}\xspace}
\newcommand{\QUATROn}{200\xspace}
\newcommand{\QUATROy}{\num{54.0000}\xspace}
\newcommand{\QUATROyLinha}{\num{54.5000}\xspace}
\newcommand{\QUATROz}{\num{2.5633}\xspace}
\newcommand{\QUATROpValue}{\num{0.0052}\xspace}
\newcommand{\QUATROesVinte}{\num{0.0000}\xspace}
\newcommand{\QUATROesVinteUm}{\num{0.0248}\xspace}
\newcommand{\QUATROesVinteDois}{\num{0.0491}\xspace}
\newcommand{\QUATROesVinteTres}{\num{0.0731}\xspace}
\newcommand{\QUATROesVinteQuatro}{\num{0.0967}\xspace}
\newcommand{\QUATROesVinteCinco}{\num{0.1199}\xspace}
\newcommand{\QUATROesVinteSeis}{\num{0.1428}\xspace}
\newcommand{\QUATROesVinteSete}{\num{0.1655}\xspace}
\newcommand{\QUATROpVinte}{\num{0.0100}\xspace}
\newcommand{\QUATROpVinteUm}{\num{0.0241}\xspace}
\newcommand{\QUATROpVinteDois}{\num{0.0514}\xspace}
\newcommand{\QUATROpVinteTres}{\num{0.0980}\xspace}
\newcommand{\QUATROpVinteQuatro}{\num{0.1687}\xspace}
\newcommand{\QUATROpVinteCinco}{\num{0.2641}\xspace}
\newcommand{\QUATROpVinteSeis}{\num{0.3797}\xspace}
\newcommand{\QUATROpVinteSete}{\num{0.5057}\xspace}
\newcommand{\QUATROesAmostra}{\num{0.0731}\xspace}
\newcommand{\QUATROtamanhoAmostra}{\num{4055.1080}\xspace}
\newcommand{\QUATROtamanhoAmostraRounded}{4056\xspace}


	Foi retirada uma amostra aleatória simples, sem reposição, com $n = 20$
	elementos, dentre os \TRESN alunos que possuíam um valor definido para a
	variável \textit{Renda}.
	A população possui média $\mu = \TRESX$ e desvio padrão $\sigma = \TRESSD$.
	A amostra obtida possui média $\bar{x}$ = \TRESx com desvio padrão $s$ = \TRESsd.

\subsection{IC de 95\% para a Média Populacional da Renda dos Alunos}

	Como o tamanho da amostra é pequeno ($n < 30$), devemos calcular o intervalo
	de confiança utilizando a \textit{distribuiçao t de Student}. Para um
	intervalo de confiança de 95\% e 19 graus de liberdade (pois $gl = 1-n$),
	obtemos um valor $t_{\gamma} = 2,093$.  Assim, podemos calcular o intervalo de
	confiança de acordo com a \autoref{eq:dois-a-expr}:

	\begin{align*}
		IC (\mu, 95\%) &= \TRESx \pm 2,093 \frac{\TRESsd}{\sqrt{\TRESn}} \sqrt{\frac{\TRESN - \TRESn}{\TRESN - 1}} \\
		IC (\mu, 95\%) &= \TRESx \pm \TRESdelta
	\end{align*}

	Assim, o intervalo de (\TRESicmin;\TRESicmax) contêm, com 95\% confiança, o
	valor da média populacional para a variável Renda.

\subsection{IC de 99\% para a Média Populacional da Renda dos Alunos}

	Para um intervalo de confiança de 99\% e 19 graus de liberdade, obtemos
	um valor $t_{\gamma} = 2,861$.  Utilizando novamente a \autoref{eq:dois-a-expr}:

	\begin{align*}
		IC (\mu, 99\%) &= \TRESx \pm 2,861 \frac{\TRESsd}{\sqrt{\TRESn}} \sqrt{\frac{\TRESN - \TRESn}{\TRESN - 1}} \\
		IC (\mu, 99\%) &= \TRESx \pm \TRESdeltaNoveNove
	\end{align*}

	Portanto, concluí-se que o intervalo de \TRESicNoveNoveMin à \TRESicNoveNoveMax contêm,
	com grau de confiança de 99\%, o valor da média populacional para a
	variável \textit{Renda}. Nota-se que o intervalo ficou maior em relação ao
	intervalo com grau de confiança de 95\%.  Para diminuir este intervalo é
	necessário aumentar o tamanho da amostra.

\subsection{Tamanho Mínimo da Amostra para um IC de 99\% e $E_0$ de 1,5}

	Para se obter o tamanho mínimo da amostra, primeiramente utilizamos a
	\autoref{eq:dois-b-expr1}. Consideramos como amostra piloto a mesma
	amostra retirada anteriormente, assim $t_{\gamma}$ = 2,861 e $s = \TRESsd$:

	\begin{align*}
		n_0 = \left (\frac{2,861 \times \TRESsd}{1,5} \right)^2 = \TRESnZero
	\end{align*}

	Como tamanho da população não é grande e é conhecido, adicionalmente
	utilizamos a \autoref{eq:dois-b-expr2}:

	\begin{align*}
			n &= \Big\lceil \frac{\TRESN \times \TRESnZero}{\TRESN + \TRESnZero} \Big\rceil \\
			n &= \Big\lceil \TRESnZeroCorrigido \Big\rceil \\
			n &= \TRESnZeroCorrigidoRounded
	\end{align*}

Portanto, são necessários \TRESnZeroCorrigidoRounded amostras para se obter, com 99\% de confiança,
um intervalo com precisão de 1,5 salários mínimos em que se encontre a média populacional.
