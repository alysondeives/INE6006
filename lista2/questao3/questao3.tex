\newcommand{\TRESicmin}{1.85\xspace}
\newcommand{\TRESicmax}{3.10\xspace}
\newcommand{\TRESicNoveNoveMin}{1.62\xspace}
\newcommand{\TRESicNoveNoveMax}{3.33\xspace}
\newcommand{\TRESnZero}{26\xspace}


	Foi retirada uma amostra aleatória simples, sem reposição, com $n = 20$
	elementos, dentre os \TRESN alunos que possuíam um valor definido para a
	variável Renda.  A população possui média \TRESX e desvio padrão de
	\TRESSD.  A média amostral obtida foi de \TRESx com desvio padrão de
	\TRESsd.

\subsection{IC de 95\% para a Média Populacional da Renda dos Alunos}

	Como o tamanho da amostra é pequeno ($n < 30$), devemos calcular o intervalo
	de confiança utilizando a \textit{distribuiçao t de Student}. Para um
	intervalo de confiança de 95\% e 19 graus de liberdade (pois $gl = 1-n$),
	obtemos um valor $t = 2,093$.  Assim, podemos calcular o intervalo de
	confiança de acordo com a \autoref{eq:dois-a-expr}:

	\begin{align*}
		IC (\mu, 95\%) &= \TRESx \pm 2,093 \frac{\TRESsd}{\sqrt{\TRESn}} \sqrt{\frac{\TRESN - \TRESn}{\TRESN - 1}} \\
		IC (\mu, 95\%) &= \TRESx \pm \TRESdelta
	\end{align*}

	Assim, o intervalo de (\TRESicmin;\TRESicmax) contêm, com 95\% confiança, o
	valor da média populacional para a variável Renda.

\subsection{IC de 99\% para a Média Populacional da Renda dos Alunos}

	Para um intervalo de confiança de 99\% e 19 graus de liberdade, obtemos
	um valor $t$ = 2,861.  Utilizando novamente a \autoref{eq:dois-a-expr}:

	\begin{align*}
		IC (\mu, 99\%) &= \TRESx \pm 2,861 \frac{\TRESsd}{\sqrt{\TRESn}} \sqrt{\frac{\TRESN - \TRESn}{\TRESN - 1}} \\
		IC (\mu, 99\%) &= \TRESx \pm \TRESdeltaNoveNove
	\end{align*}

	Assim, o intervalo de \TRESicNoveNoveMin à \TRESicNoveNoveMax contêm,
	com grau de confiança de 99\%, o valor da média populacional para a
	variável Renda. Nota-se que o intervalo ficou maior em relação ao
	intervalo com grau de confiança de 95\%.  Para diminuir este intervalo é
	necessário aumentar o tamanho da amostra.

\subsection{Tamanho Mínimo da Amostra para um IC de 99\% e $E_0$ de 1,5}

	Para se obter o tamanho mínimo da amostra, primeiramente utilizamos a
	\autoref{eq:dois-b-expr1}, considerando como amostra piloto a mesma
	amostra retirada anteriormente:

	\begin{align*}
		n_0 = \left (\frac{2,861 \times \TRESsd}{1,5} \right)^2 = \TRESnZero %\cong \TRESnZeroRounded
	\end{align*}

	Como tamanho da população não é grande e é conhecido, adicionalmente
	utilizamos a \autoref{eq:dois-b-expr2}:

	\begin{align*}
			n &= \Big\lceil \frac{\TRESN \times \TRESnZero}{\TRESN + \TRESnZero} \Big\rceil \\
			n &= \Big\lceil \TRESnZeroCorrigido \Big\rceil \\
			n &= \TRESnZeroCorrigidoRounded
	\end{align*}

Portanto, são necessários \TRESnZeroCorrigidoRounded amostras para se obter, com 99\% de confiança,
um intervalo de precisão de 1,5 salários mínimos em que se encontre a média populacional.
