\newcommand{\QUATROpAmostral}{\num{0.2700}\xspace}
\newcommand{\QUATROn}{200\xspace}
\newcommand{\QUATROy}{\num{54.0000}\xspace}
\newcommand{\QUATROyLinha}{\num{54.5000}\xspace}
\newcommand{\QUATROz}{\num{2.5633}\xspace}
\newcommand{\QUATROpValue}{\num{0.0052}\xspace}
\newcommand{\QUATROesVinte}{\num{0.0000}\xspace}
\newcommand{\QUATROesVinteUm}{\num{0.0248}\xspace}
\newcommand{\QUATROesVinteDois}{\num{0.0491}\xspace}
\newcommand{\QUATROesVinteTres}{\num{0.0731}\xspace}
\newcommand{\QUATROesVinteQuatro}{\num{0.0967}\xspace}
\newcommand{\QUATROesVinteCinco}{\num{0.1199}\xspace}
\newcommand{\QUATROesVinteSeis}{\num{0.1428}\xspace}
\newcommand{\QUATROesVinteSete}{\num{0.1655}\xspace}
\newcommand{\QUATROpVinte}{\num{0.0100}\xspace}
\newcommand{\QUATROpVinteUm}{\num{0.0241}\xspace}
\newcommand{\QUATROpVinteDois}{\num{0.0514}\xspace}
\newcommand{\QUATROpVinteTres}{\num{0.0980}\xspace}
\newcommand{\QUATROpVinteQuatro}{\num{0.1687}\xspace}
\newcommand{\QUATROpVinteCinco}{\num{0.2641}\xspace}
\newcommand{\QUATROpVinteSeis}{\num{0.3797}\xspace}
\newcommand{\QUATROpVinteSete}{\num{0.5057}\xspace}
\newcommand{\QUATROesAmostra}{\num{0.0731}\xspace}
\newcommand{\QUATROtamanhoAmostra}{\num{4055.1080}\xspace}
\newcommand{\QUATROtamanhoAmostraRounded}{4056\xspace}


Foi retirada uma amostra aleatória simples, sem reposição, com $n$ = 20 elementos,
dentre os \TRESN alunos que possuíam um valor definido para a variável Renda.
A população possui média \TRESX e desvio padrão de \TRESSD.
A média amostral obtida foi de \TRESx com desvio padrão de \TRESsd.

%<<<<<<< HEAD
\subsection{Intervalo de 95\% de confiança para a média populacional da Renda dos alunos}
Como o tamanho da amostra é pequeno ($n$ < 30), devemos calcular o intervalo
de confiança utilizando a \textit{distribuiçao t de Student}. Para um intervalo
de confiança de 95\% e 19 graus de liberdade (pois gl = 1-$n$), obtemos um valor $t$ = 2,093.
Assim, podemos calcular o intervalo de confiança de acordo com a \autoref{eq:dois-a-expr}:

\begin{align*}
	IC (\mu, 95\%) &= \TRESx \pm 2,093 \frac{\TRESsd}{\sqrt{\TRESn}} \sqrt{\frac{\TRESN - \TRESn}{\TRESN - 1}} \\
	IC (\mu, 95\%) &= \TRESx \pm \TRESdelta
\end{align*}

Assim, o intervalo de (\TRESicmin;\TRESicmax) contêm, com 95\% confiança, o
valor da média populacional para a variável Renda.

\subsection{Intervalo de 99\% de confiança para a média populacional da Renda dos alunos}
Para um intervalo de confiança de 99\% e 19 graus de liberdade, obtemos um valor $t$ = 2,861.
Utilizando novamente a \autoref{eq:dois-a-expr}:

\begin{align*}
	IC (\mu, 99\%) &= \TRESx \pm 2,861 \frac{\TRESsd}{\sqrt{\TRESn}} \sqrt{\frac{\TRESN - \TRESn}{\TRESN - 1}} \\
	IC (\mu, 99\%) &= \TRESx \pm \TRESdeltaNoveNove
\end{align*}

Assim, o intervalo de \TRESicNoveNoveMin à \TRESicNoveNoveMax contêm, com grau de confiança de 99\%, o valor
da média populacional para a variável Renda. Nota-se que o intervalo ficou maior em relação ao intervalo com grau de confiança de 95\%. 
Para diminuir este intervalo é necessário aumentar o tamanho da amostra.

\subsection{Tamanho mínimo da amostra para um intervalo de 99\% de confiança e precisão de 1,5 salários}
Para se obter o tamanho mínimo da amostra, primeiramente utilizamos a \autoref{eq:dois-b-expr1}, considerando 
como amostra piloto a mesma amostra retirada anteriormente:

\begin{align*}
	n_0 = \left (\frac{2,861 \times \TRESsd}{1,5} \right)^2 = \TRESnZero %\cong \TRESnZeroRounded
\end{align*}

Como tamanho da população não é grande e é conhecido, adicionalmente utilizamos a \autoref{eq:dois-b-expr2}:

\begin{align*}
		n &= \lceil \frac{\TRESN \times \TRESnZero}{\TRESN + \TRESnZero} \rceil \\
		n &= \lceil \TRESnZeroCorrigido \rceil \\
		n &= \TRESnZeroCorrigidoRounded
\end{align*}

Portanto, são necessários \TRESnZeroCorrigidoRounded amostras para se obter, com 99\% de confiança,
um intervalo de precisão de 1,5 salários mínimos em que se encontre a média populacional.
%=======
%\subsection{Intervalo de confiança de 95\% para Renda}
%\todo[inline]{Desenvolvimento?}
%O intervalo encontrado com grau de confiança de 95\% é de \TRESicmin à \TRESicmax.
%\todo[inline]{Interpretação do resultado?}

%\subsection{Intervalo de Confiança de 99\% para Renda}
%\todo[inline]{Desenvolvimento?}
%O intervalo encontrado com grau de confiança de 99\% é de \TRESicNoveNoveMin à \TRESicNoveNoveMax.
%O intervalo ficou maior em relação ao intervalo com grau de confiança de 95\%.{}
%Para diminuir esse intervalo seria necessário aumentar o tamanho da amostra.

%\subsection{Tamanho mínimo de amostra}
%\todo[inline]{Desenvolvimento? No mínimo citar as equações usadas (o professor não tem acesso ao script)}
%\todo[inline]{Erro: Aqui está sendo calculado $n_0$. Como $N$ (tamanho da população) é conhecido, \\ $n=\frac{N \cdot n_0}{N + n_0 - 1}$}
%\todo[inline]{Mencionar que a variância populacional foi estimada a partir da amostra já usada nessa questão, considerando a distribuição $t$ com $gl=19$.}
%Para se obter um intervalo de confiança de 99\% com precisão de \num{1.5} salários
%minimos, seria necessário uma amostra de tamanho \TRESnZero.

%Obs: precisao de \num{1.5} quer dizer um intervalo de \num{1.5}, entao E = \num{1.5}/2?
%\todo[inline]{Precisão de \num{1.5} quer dizer $E_0=\num{1.5}$}
%>>>>>>> c837ae06ebc7ea456f04253385b1ccb383b46cef
