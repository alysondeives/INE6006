\newcommand{\TRESicmin}{1.85\xspace}
\newcommand{\TRESicmax}{3.10\xspace}
\newcommand{\TRESicNoveNoveMin}{1.62\xspace}
\newcommand{\TRESicNoveNoveMax}{3.33\xspace}
\newcommand{\TRESnZero}{26\xspace}


\subsection{Intervalo de confiança de 95\% para Renda}
\todo[inline]{Desenvolvimento?}
O intervalo encontrado com grau de confiança de 95\% é de \TRESicmin à \TRESicmax.
\todo[inline]{Interpretação do resultado?}

\subsection{Intervalo de Confiança de 99\% para Renda}
\todo[inline]{Desenvolvimento?}
O intervalo encontrado com grau de confiança de 99\% é de \TRESicNoveNoveMin à \TRESicNoveNoveMax.
O intervalo ficou maior em relação ao intervalo com grau de confiança de 95\%.{}
Para diminuir esse intervalo seria necessário aumentar o tamanho da amostra.

\subsection{Tamanho mínimo de amostra}
\todo[inline]{Desenvolvimento? No mínimo citar as equações usadas (o professor não tem acesso ao script)}
\todo[inline]{Erro: Aqui está sendo calculado $n_0$. Como $N$ (tamanho da população) é conhecido, \\ $n=\frac{N \cdot n_0}{N + n_0 - 1}$}
\todo[inline]{Mencionar que a variância populacional foi estimada a partir da amostra já usada nessa questão, considerando a distribuição $t$ com $gl=19$.}
Para se obter um intervalo de confiança de 99\% com precisão de 1.5 salários
minimos, seria necessário uma amostra de tamanho \TRESnZero.

Obs: precisao de 1.5 quer dizer um intervalo de 1.5, entao E = 1.5/2?
\todo[inline]{Precisão de 1.5 quer dizer $E_0=1.5$}
