% latex table generated in R 3.2.4 by xtable 1.8-2 package
% Fri Apr  8 21:31:49 2016
\begin{table}[h]
\footnotesize
\centering
\caption{Grau de satisfação dos alunos por classe econômica e região}
\label{tabela:q16}
\vspace{0.5em}
\begin{tabular}{ll rrrrr}
  \toprule
 \textbf{Classe}        & \textbf{Região}  & \textbf{\specialcell{c}{Muito\\insatisfeito}} & \textbf{Insatisfeito} & \textbf{Indiferente} & \textbf{Satisfeito} & \textbf{\specialcell{c}{Muito\\satisfeito}} \\ 
   \midrule
			& Aratibutantã            &                101 &          263 &         127 &         16 &                1 \\ 
                & Baependinha             &                 11 &           58 &          54 &         25 &                7 \\ 
	Abastados       & Itamaracanã             &                  0 &            0 &           0 &          0 &                4 \\ 
                & Jaquereçaba             &                234 &          178 &          41 &          3 &                0 \\ 
                & Paranapitanga           &                111 &           10 &           0 &          0 &                0 \\ 
\midrule
			& Aratibutantã            &                  6 &          118 &         329 &        173 &               40 \\ 
                & Baependinha             &                  4 &           84 &         411 &        670 &              450 \\ 
	Intermediario & Itamaracanã             &                  0 &            0 &           6 &         30 &               76 \\ 
                & Jaquereçaba             &                  2 &           36 &          31 &          9 &                0 \\ 
                & Paranapitanga           &                  0 &            0 &           0 &          0 &                0 \\ 
\midrule
			& Aratibutantã            &                  0 &            0 &           0 &          4 &                4 \\ 
                & Baependinha             &                  0 &            0 &           5 &         84 &              419 \\ 
	Pobres        & Itamaracanã             &                  0 &            0 &           1 &         14 &              710 \\ 
                & Jaquereçaba             &                  0 &            0 &           0 &          0 &                0 \\ 
                & Paranapitanga           &                  0 &            0 &           0 &          0 &                0 \\ 
   \bottomrule
\end{tabular}
\end{table}

