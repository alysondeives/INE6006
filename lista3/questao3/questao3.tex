\subsection{Teste de hipótese}
\label{questao:3a}
Para verificar se o colega estava certo em afirmar que menos de $35\%$ dos alunos utilizam a forma de pagamento ``Incentivos Federais'', 
foram formuladas as hipóteses nula e alternativa.
Consideramos como hipótese nula que a proporção de alunos que usam ``Incentivos Federais`` é igual ou superior a $0.35$. 
Para a hipótese alternativa, consideramos que a proporção de alunos que usam ``Incentivos Federais'' é menor que $0.35$.

\begin{align*}
  H_0\!:   &\; p = 0.35 \\
  H_1\!:   &\; p < 0.35  \\
   \alpha\!:&\; 0.05
\end{align*}

A proporção amostral $\hat{p}$ de alunos (obtida anteriormente na \textit{Lista 2}) cuja fonte de pagamento
é ``Incentivos Federais'' foi de $0.32$, para uma amostra aleatória de $200$ registros.
Como procura-se evidência de que a proporção de alunos, cuja fonte de pagamento é ``Incentivos Federais'', 
seja inferior a $35\%$ ($0.35$), realiza-se um teste unilateral a esquerda. É possível utilizar a aproximação pela normal, pois:

\begin{align*}
  n \cdot p_0\!  &\; = 200 \cdot 0.35 \\
  &\; = 70 \geq 5 ,\\
  & \text{e} \\
  n \cdot (1 - p_0)\!	&\;=\; 200 \cdot (1 - 0.35) \\
  &\;= 130  \geq 5 \\
\end{align*}

É recomendável, ainda, aplicar a correção de continuidade. 
Como, neste caso, estamos trabalhando com uma aproximação pela normal e  
$y < n \cdot p_0$ – conforme demonstrado na fórmula \ref{eq:cinco-a-y-correcao} – 
incrementamos $0.5$ do $y$ para obter $y'$. Logo, $y' = 64.5$, sendo $y$ o número de alunos utilizam a forma de pagamento ``Incentivos Federais'' 

\begin{align}
  \label{eq:cinco-a-y-correcao}
  y   &\; =  n \cdot \hat{p} < n \cdot p_0 \\ 
      &\; = 200 \cdot 0.32 < 200 \cdot 0.35 \nonumber \\
      &\; = 64 < 70 \nonumber
\end{align}

Logo, aplicamos a equação \ref{eq:cinco-b-n-result} com o intuito de obter o \textit{valor-p} 
apropriado e compará-lo com o nível de significância de $5\%$.

\begin{equation}
  \label{eq:cinco-b-n-result}
  z = \frac{y' - n \cdot p_0}{ \sqrt{n \cdot p_0 (1 - p_0)} }
\end{equation}

Substituindo os valores, temos:

\begin{align*}
  z &\;= \frac{64.5 - 200 \cdot 0.35}{ \sqrt{200 \cdot 0.35 \cdot (1 - 0.35)} } \\
    &\;= - 0.8154 \nonumber \\
\end{align*}

Procurando na tabela da distribuição normal pelo $z$, 
obtém-se uma cauda inferior de aproximadamente $p = 0.2074$.
Portanto, como $p > 0.05$, aceita-se $H_0$ ao nível de significância de 5\%. 
Não se pode afirmar, ao nível de significância de $5\%$, que o colega estava certo, ou seja, 
não é possível afirmar que a proporção de  alunos que usam ``Incentivos Federais'' é menor do que $35\%$.


\subsection{Poder do Teste}
\label{questao:3b}

Para obter o tamanho da amostra foi utilizado o software R e a função \r|pwr.p.test|, a qual calcula o poder para testes de proporção de uma amostra.
A função \r|ES.h|, por sua vez, foi utilizada também para calcular o \textit{Effect Size} para proporção.
Em vista disso, o seguinte fragmento de código foi executado no ambiente R:

\inputr{questao3/b.R}

Na função \r|pwr.p.test| foram passados como parâmetros os \textit{Effect Sizes} das proporções contidas em \r|propsList3b|;
o tamanho da amostra (\r|n = 200|); 
o nível de significância de $5\%$ (\r|sig.level = 0.05|);
a indicação de que se trata de um teste unilateral à esquerda (\r|alternative = less|); 
e o poder (\r|power|) como NULL, visto que se trata do valor desejado a ser calculado.
A partir da execução do código R, foi gerada a \autoref{tb:3b}, a qual mostra os resultados obtidos do poder do teste calculado para os valores $0.30$, $0.32$ $0.33$ e $0.34$.

\begin{table}[ht]
\centering
\caption{Poder do teste para diferentes valores de proporção de alunos que usam Incentivos Federais.} 
\label{tb:3b}
\begin{tabular}{rrrrr}
  \toprule
 $p$ & 0.30 & 0.32 & 0.33 & 0.34 \\ 
  \midrule
$ES$ 	& -0.10682 & -0.06358 & -0.04222 & -0.02104 \\ 
$1 - \beta$ & 0.44665   & 0.22791   & 0.14739   & 0.08893  \\ 
   \bottomrule
\end{tabular}
\end{table}

Observou-se que à medida que a proporção $p$ se aproxima da proporção $p_0$, o poder do teste diminui.


\subsection{Tamanho mínimo de amostra}
\label{questao:3c}
O mesmo comando \r|pwr.p.test| do ambiente R, utilizado no \autoref{questao:3b}, pode ser ser usado para calcular o tamanho mínimo da amostra: 

\inputr{questao3/c.R}

Como o objetivo é encontrar o tamanho mínimo da amostra, o parâmetro $n$ é passado na função como NULL. Além disso, consideramos o poder do teste como $0.99$.
Como resultado, foram obtidos os dados apresentados pela \autoref{tb:3c}. 
Visto que o tamanho de amostra encontrado pela função não é um inteiro ($2177.447$), é necessário efetuar um arredondamento para cima, obtendo o valor final de $2178$.

\begin{table}[ht]
\centering
\caption{Poder do Teste para detectar o tamanho da amostra.} 
\label{tb:3c}
\begin{tabular}{rrrrr}
  \toprule
 $p$ & $p_0$ & $n$ & $Effect Size$ & $1 - \beta$ \\ 
  \midrule
  $0.31$ & $0.35$ & $2177.447$ & $-0.08510364$ & $0.99$ \\ 
   \bottomrule
\end{tabular}
\end{table}

A amostra de $200$ elementos, portanto, não é suficiente para detectar com 99\% de probabilidade que a proporção
de alunos que usam a fonte de pagamento ``Incentivos Federais'' é igual a 31\%, considerando 5\% de significância.
Através do cálculo para o tamanho mínimo de amostra realizado no R, 
obteve-se que este tamanho deve ser de pelo menos $2178$ elementos.

%%% Local Variables:
%%% mode: latex
%%% TeX-master: "../main"
%%% End:
