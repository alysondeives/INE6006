\newcommand{\TRESicmin}{1.85\xspace}
\newcommand{\TRESicmax}{3.10\xspace}
\newcommand{\TRESicNoveNoveMin}{1.62\xspace}
\newcommand{\TRESicNoveNoveMax}{3.33\xspace}
\newcommand{\TRESnZero}{26\xspace}


\subsection{Teste de hipótese}
\label{questao:1a}

Para testar a hipótese descrita é realizadodeve ser realizado um teste de hipótese assimétrico para a média. Como essa é a situação mais comum na prática, a variância populacional foi considerada como desconhecida. Considere como hipótese nula que a média populacional da variável Idade é de \UMAu0 anos. Para a hipótese alternativa, considere que a média populacional da variável Idade é maior que \UMAu0 anos. O nível de significância do teste é de 5\%.

\begin{align*}
  H_0\!:   &\; \mu = \UMAu0 \\
  H_1\!:   &\; \mu > \UMAu0  \\
  \alpha\!:&\; \UMAalpha  
\end{align*}

Por se tratar de um teste sobre a média, e como o desvio padrão populacional é desconhecido, a estatística do teste é o valor $t$ (da distribuição de Student) com $gl=\UMAgl$.

\begin{align*}
  t &= \frac{(\bar{x} - \mu_0)\cdot\sqrt{n}}{s} \\
  t &= \frac{(\UMAbarx - \UMAu0)\cdot\sqrt{\UMAn}}{\UMAs} \\
  t &= \UMAt
\end{align*}

O valor $p$ da amostra é de \UMAp, calculado com R usando o seguinte código :\texttt{pt(x = \UMAt, df = \UMAgl, lower.tail = FALSE)}. Logo, como $p \geq \alpha$, $H_0$ é aceita e não se pode afirmar que há evidencia em favor de $H_1$. A hipótese da direção de que a média de idade é maior que 27 anos não pôde ser confirmada (com uma amostra de \UMAn alunos).

\subsection{Poder do Teste}
\label{questao:1b}

O poder do teste ($\beta$) foi calculado usando a função \r|power.t.test|, de maneira análoga ao mostrado em documento disponibilizado no Moodle da disciplina. A \autoref{tb:1b} mostra o poder do teste calculado para os valores listados no enunciado. Essa tabela foi gerada a partir do resultado do seguinte frgmento de código R:

\inputminted{r}{questao1/b.R}

\todo[inline]{Discussão: Eu não consigo explicar o que fiz em texto sem enrolar demais. E o pior é que além de enrolado a explicação ficaria confusa, o que não é nada bom. \\

Tentei muitas coisas, a melhor solução for dividir o script.R em script.R, calc.R e b.R. Todas as coisas relacionadas com I/O ficam no script.R, calc.R só tem cálculos e b.R foi separado para que pudesse ser incluído. Eu também mudei as variáveis para que o nome delas fosse representativo do significado. Assim, eu espero, nós nao precisamos colocar o calc.R como anexo. Se adotarmos essa abordagem, precisamos todos entrar em acordo quanto a convenção de nomenclatura das variáveis no R pensando no que vai ficar claro pro professor.
}

%Pensando em editar o tex gerado? Pense denovo.
% latex table generated in R 3.3.0 by xtable 1.8-2 package
% Wed Jun 15 23:21:45 2016
\begin{table}[ht]
\centering
\caption{Poder do teste para diferentes valores da média populacional real.} 
\label{tb:1b}
\begin{tabular}{rrrrrrrr}
  \toprule
 & 28 & 30 & 31 & 32 & 33 & 34 & 35 \\ 
  \midrule
$1-\beta$ & 0.2207 & 0.8359 & 0.9679 & 0.9968 & 0.9998 & 1.0000 & 1.0000 \\ 
   \bottomrule
\end{tabular}
\end{table}


\subsection{Tamanho mínimo de amostra}
\label{questao:1c}

A mesma função, \r|power.t.test| do ambiente R, usada em \autoref{questao:1b} pode ser ser usada para calcular o tamanho mínimo da amostra. Nesse caso, é desejado um poder do teste de 90\% ao detectar a uma diferença de 2 anos na média, ao nível de significância de 5\%. Novamente usando o desvio padrão amostral $s = \UMAs$ como estimativa para $\sigma$, o seguinte código R pode ser usado:

\inputminted{r}{questao1/c.R}

O tamanho de amostra encontrado pela função, \UMCn, não é um inteiro, portanto é necessário efetuar um arredondamento para cima e proceder com um novo cálculo de $\beta$. Desse modo, o tamanho mínimo de amostra necessário é de \UMCnMin, que para média real $\mu = 29$, concede ao teste um poder de \UMCbeta.

A amostra atual não é suficiente, pois os \UMAn elementos da amostras possibilitariam um teste com poder de apenas \UMCbetaOld, mantidos $\alpha = 0.05$ e $\mu = 29$.

\subsection{Nova amostra}
\label{questao:1d}

Foi retirada uma amostra aleatória simples com \UMCnMin elementos, que apresentou desvio padrão amostral $s = \UMDs$. De maneira análoga, a função \r|power.t.test| foi usada para calcular o poder do teste considerando diversas possibilidades para a média real da população. A lista de difereças \r|deltas1b|  permanace a mesma, pois essa lista é definida pela expressão $\mu - \mu_0$, onde $\mu$ é a média real que assume os valores listados no enunciado do \autoref{questao:1b}.

\inputminted{r}{questao1/d-2.R}

Os resultados desse cálculo são apresentados na linha $\beta_d$ da \autoref{tb:1d}, que também contêm os mesmos resultados usando o desvio padrão amostral $s$ e o tamanho $n$ da amostra usada no \autoref{questao:1b}. Houve aumento do poder do teste para todos os valores de $\mu$, que possuiam $\beta_b \neq 1$. Em especial, quanto mais distante de $\beta_b$ estava de 1, maior foi o aumento proporcional observado em $\beta_d$. Esse aumento em $\beta_d$ poderia ter sido ainda maior, se o desvio padrão amostral da nova amostra, \UMDs, tivesse sido menor ou igual ao desvio padrão da amostra usada no \autoref{questao:1b}, \UMAs.

A \autoref{fig:1d} mostra gráficamente a relação entre distância da média real com $\mu_0 = \UMAu0$.

\todo[inline]{Fazer gráfico}

% latex table generated in R 3.3.0 by xtable 1.8-2 package
% Wed Jun 15 23:21:45 2016
\begin{table}[ht]
\centering
\caption{Poder do teste usando $s$ e $n$ das amostras dos itens d e b.} 
\label{tb:1d}
\begin{tabular}{rrrrrrrr}
  \toprule
 & 28 & 30 & 31 & 32 & 33 & 34 & 35 \\ 
  \midrule
$1-\beta_d$ & 0.3187 & 0.9696 & 0.9988 & 1.0000 & 1.0000 & 1.0000 & 1.0000 \\ 
  $1-\beta_b$ & 0.2207 & 0.8359 & 0.9679 & 0.9968 & 0.9998 & 1.0000 & 1.0000 \\ 
   \bottomrule
\end{tabular}
\end{table}


%%% Local Variables:
%%% mode: latex
%%% TeX-master: "../main"
%%% End:
