\newcommand{\QUATROpAmostral}{\num{0.2700}\xspace}
\newcommand{\QUATROn}{200\xspace}
\newcommand{\QUATROy}{\num{54.0000}\xspace}
\newcommand{\QUATROyLinha}{\num{54.5000}\xspace}
\newcommand{\QUATROz}{\num{2.5633}\xspace}
\newcommand{\QUATROpValue}{\num{0.0052}\xspace}
\newcommand{\QUATROesVinte}{\num{0.0000}\xspace}
\newcommand{\QUATROesVinteUm}{\num{0.0248}\xspace}
\newcommand{\QUATROesVinteDois}{\num{0.0491}\xspace}
\newcommand{\QUATROesVinteTres}{\num{0.0731}\xspace}
\newcommand{\QUATROesVinteQuatro}{\num{0.0967}\xspace}
\newcommand{\QUATROesVinteCinco}{\num{0.1199}\xspace}
\newcommand{\QUATROesVinteSeis}{\num{0.1428}\xspace}
\newcommand{\QUATROesVinteSete}{\num{0.1655}\xspace}
\newcommand{\QUATROpVinte}{\num{0.0100}\xspace}
\newcommand{\QUATROpVinteUm}{\num{0.0241}\xspace}
\newcommand{\QUATROpVinteDois}{\num{0.0514}\xspace}
\newcommand{\QUATROpVinteTres}{\num{0.0980}\xspace}
\newcommand{\QUATROpVinteQuatro}{\num{0.1687}\xspace}
\newcommand{\QUATROpVinteCinco}{\num{0.2641}\xspace}
\newcommand{\QUATROpVinteSeis}{\num{0.3797}\xspace}
\newcommand{\QUATROpVinteSete}{\num{0.5057}\xspace}
\newcommand{\QUATROesAmostra}{\num{0.0731}\xspace}
\newcommand{\QUATROtamanhoAmostra}{\num{4055.1080}\xspace}
\newcommand{\QUATROtamanhoAmostraRounded}{4056\xspace}


\subsection{Teste de hipótese}
\label{questao:1a}

Para testar a hipótese descrita é realizadodeve ser realizado um teste de hipótese assimétrico para a média. Como essa é a situação mais comum na prática, a variância populacional foi considerada como desconhecida. Considere como hipótese nula que a média populacional da variável Idade é de \UMAu0 anos. Para a hipótese alternativa, considere que a média populacional da variável Idade é maior que \UMAu0 anos. O nível de significância do teste é de 5\%.

\begin{align*}
  H_0\!:   &\; \mu = \UMAu0 \\
  H_1\!:   &\; \mu > \UMAu0  \\
  \alpha\!:&\; \UMAalpha  
\end{align*}

Por se tratar de um teste sobre a média, e como o desvio padrão populacional é desconhecido, a estatística do teste é o valor $t$ (da distribuição de Student) com $gl=\UMAgl$.

\begin{align*}
  t &= \frac{(\bar{x} - \mu_0)\cdot\sqrt{n}}{s} \\
  t &= \frac{(\UMAbarx - \UMAu0)\cdot\sqrt{\UMAn}}{\UMAs} \\
  t &= \UMAt
\end{align*}

O valor $p$ da amostra é de \UMAp, calculado com R usando o seguinte código :\texttt{pt(x = \UMAt, df = \UMAgl, lower.tail = FALSE)}. Logo, como $p \geq \alpha$, $H_0$ é aceita e não se pode afirmar que há evidencia em favor de $H_1$. A hipótese da direção de que a média de idade é maior que 27 anos não pôde ser confirmada (com uma amostra de \UMAn alunos).

\subsection{Poder do Teste}
\label{questao:1b}

\todo[inline]{Fazer}


%%% Local Variables:
%%% mode: latex
%%% TeX-master: "../main"
%%% End:
