\newcommand{\QUATROpAmostral}{\num{0.2700}\xspace}
\newcommand{\QUATROn}{200\xspace}
\newcommand{\QUATROy}{\num{54.0000}\xspace}
\newcommand{\QUATROyLinha}{\num{54.5000}\xspace}
\newcommand{\QUATROz}{\num{2.5633}\xspace}
\newcommand{\QUATROpValue}{\num{0.0052}\xspace}
\newcommand{\QUATROesVinte}{\num{0.0000}\xspace}
\newcommand{\QUATROesVinteUm}{\num{0.0248}\xspace}
\newcommand{\QUATROesVinteDois}{\num{0.0491}\xspace}
\newcommand{\QUATROesVinteTres}{\num{0.0731}\xspace}
\newcommand{\QUATROesVinteQuatro}{\num{0.0967}\xspace}
\newcommand{\QUATROesVinteCinco}{\num{0.1199}\xspace}
\newcommand{\QUATROesVinteSeis}{\num{0.1428}\xspace}
\newcommand{\QUATROesVinteSete}{\num{0.1655}\xspace}
\newcommand{\QUATROpVinte}{\num{0.0100}\xspace}
\newcommand{\QUATROpVinteUm}{\num{0.0241}\xspace}
\newcommand{\QUATROpVinteDois}{\num{0.0514}\xspace}
\newcommand{\QUATROpVinteTres}{\num{0.0980}\xspace}
\newcommand{\QUATROpVinteQuatro}{\num{0.1687}\xspace}
\newcommand{\QUATROpVinteCinco}{\num{0.2641}\xspace}
\newcommand{\QUATROpVinteSeis}{\num{0.3797}\xspace}
\newcommand{\QUATROpVinteSete}{\num{0.5057}\xspace}
\newcommand{\QUATROesAmostra}{\num{0.0731}\xspace}
\newcommand{\QUATROtamanhoAmostra}{\num{4055.1080}\xspace}
\newcommand{\QUATROtamanhoAmostraRounded}{4056\xspace}


\subsection{Teste de hipótese}
\label{questao:1a}

Para testar a hipótese descrita é realizadodeve ser realizado um teste de hipótese assimétrico para a média. Como essa é a situação mais comum na prática, a variância populacional foi considerada como desconhecida. Considere como hipótese nula que a média populacional da variável Idade é de \UMAu0 anos. Para a hipótese alternativa, considere que a média populacional da variável Idade é maior que \UMAu0 anos. O nível de significância do teste é de 5\%.

\begin{align*}
  H_0\!:   &\; \mu = \UMAu0 \\
  H_1\!:   &\; \mu > \UMAu0  \\
  \alpha\!:&\; \UMAalpha  
\end{align*}

Por se tratar de um teste sobre a média, e como o desvio padrão populacional é desconhecido, a estatística do teste é o valor $t$ (da distribuição de Student) com $gl=\UMAgl$.

\begin{align*}
  t &= \frac{(\bar{x} - \mu_0)\cdot\sqrt{n}}{s} \\
  t &= \frac{(\UMAbarx - \UMAu0)\cdot\sqrt{\UMAn}}{\UMAs} \\
  t &= \UMAt
\end{align*}

O valor $p$ da amostra é de \UMAp, calculado com R usando o seguinte código :\texttt{pt(x = \UMAt, df = \UMAgl, lower.tail = FALSE)}. Logo, como $p \geq \alpha$, $H_0$ é aceita e não se pode afirmar que há evidencia em favor de $H_1$. A hipótese da direção de que a média de idade é maior que 27 anos não pôde ser confirmada (com uma amostra de \UMAn alunos).

\subsection{Poder do Teste}
\label{questao:1b}

O poder do teste ($\beta$) foi calculado usando a função \r|power.t.test|, de maneira análoga ao mostrado em documento disponibilizado no Moodle da disciplina. A \autoref{tb:1b} mostra o poder do teste calculado para os valores listados no enunciado. Essa tabela foi gerada a partir do resultado do seguinte frgmento de código R:

\inputr{questao1/b.R}

\todo[inline]{Discussão: Eu não consigo explicar o que fiz em texto sem enrolar demais. E o pior é que além de enrolado a explicação ficaria confusa, o que não é nada bom. \\

Tentei muitas coisas, a melhor solução for dividir o script.R em script.R, calc.R e b.R. Todas as coisas relacionadas com I/O ficam no script.R, calc.R só tem cálculos e b.R foi separado para que pudesse ser incluído. Eu também mudei as variáveis para que o nome delas fosse representativo do significado. Assim, eu espero, nós nao precisamos colocar o calc.R como anexo. Se adotarmos essa abordagem, precisamos todos entrar em acordo quanto a convenção de nomenclatura das variáveis no R pensando no que vai ficar claro pro professor.
}

%Pensando em editar o tex gerado? Pense denovo.
% latex table generated in R 3.3.0 by xtable 1.8-2 package
% Wed Jun 15 23:21:45 2016
\begin{table}[ht]
\centering
\caption{Poder do teste para diferentes valores da média populacional real.} 
\label{tb:1b}
\begin{tabular}{rrrrrrrr}
  \toprule
 & 28 & 30 & 31 & 32 & 33 & 34 & 35 \\ 
  \midrule
$1-\beta$ & 0.2207 & 0.8359 & 0.9679 & 0.9968 & 0.9998 & 1.0000 & 1.0000 \\ 
   \bottomrule
\end{tabular}
\end{table}


\subsection{Tamanho mínimo de amostra}
\label{questao:1c}

A mesma função, \r|power.t.test| do ambiente R, usada em \autoref{questao:1b} pode ser ser usada para calcular o tamanho mínimo da amostra. Nesse caso, é desejado um poder do teste de 90\% ao detectar a uma diferença de 2 anos na média, ao nível de significância de 5\%. Novamente usando o desvio padrão amostral $s = \UMAs$ como estimativa para $\sigma$, o seguinte código R pode ser usado:

\inputr{questao1/c.R}

O tamanho de amostra encontrado pela função, \UMCn, não é um inteiro, portanto é necessário efetuar um arredondamento para cima e proceder com um novo cálculo de $\beta$. Desse modo, o tamanho mínimo de amostra necessário é de \UMCnMin, que para média real $\mu = 29$, concede ao teste um poder de \UMCbeta.

A amostra atual não é suficiente, pois os \UMAn elementos da amostras possibilitariam um teste com poder de apenas \UMCbetaOld, mantidos $\alpha = 0.05$ e $\mu = 29$.

\subsection{Nova amostra}
\label{questao:1d}

Foi retirada uma amostra aleatória simples com \UMCnMin elementos, que apresentou desvio padrão amostral $s = \UMDs$. De maneira análoga, a função \r|power.t.test| foi usada para calcular o poder do teste considerando diversas possibilidades para a média real da população. A lista de difereças \r|deltas1b|  permanace a mesma, pois essa lista é definida pela expressão $\mu - \mu_0$, onde $\mu$ é a média real que assume os valores listados no enunciado do \autoref{questao:1b}.

\inputr{questao1/d-2.R}

Os resultados desse cálculo são apresentados na linha $\beta_d$ da \autoref{tb:1d}, que também contêm os mesmos resultados usando o desvio padrão amostral $s$ e o tamanho $n$ da amostra usada no \autoref{questao:1b}. Houve aumento do poder do teste para todos os valores de $\mu$, que possuiam $\beta_b \neq 1$. Em especial, quanto mais distante de $\beta_b$ estava de 1, maior foi o aumento proporcional observado em $\beta_d$. Esse aumento em $\beta_d$ poderia ter sido ainda maior, se o desvio padrão amostral da nova amostra, \UMDs, tivesse sido menor ou igual ao desvio padrão da amostra usada no \autoref{questao:1b}, \UMAs.

% latex table generated in R 3.3.0 by xtable 1.8-2 package
% Thu Jun 16 13:48:54 2016
\begin{table}[ht]
\centering
\caption{Poder do teste usando $s$ e $n$ das amostras dos itens d e b.} 
\label{tb:1d}
\begin{tabular}{rrrrrrrr}
  \toprule
 & 28 & 30 & 31 & 32 & 33 & 34 & 35 \\ 
  \midrule
$1-\beta_d$ & 0.3749 & 0.9902 & 0.9999 & 1.0000 & 1.0000 & 1.0000 & 1.0000 \\ 
  $1-\beta_b$ & 0.2207 & 0.8359 & 0.9679 & 0.9968 & 0.9998 & 1.0000 & 1.0000 \\ 
   \bottomrule
\end{tabular}
\end{table}


A \autoref{fig:1-power} mostra graficamente o poder do teste usando os parâmetros do \autoref{questao:1b} e do \autoref{questao:1d} de acordo com a média real $\mu$. Os dados foram obtidos usando o mesmo método usado na \autoref{tb:1d}, e $\beta_b$ e $\beta_d$ possuem o mesmo significado.

\begin{figure}[ht]
  \centering
  % GNUPLOT: LaTeX picture
\setlength{\unitlength}{0.240900pt}
\ifx\plotpoint\undefined\newsavebox{\plotpoint}\fi
\sbox{\plotpoint}{\rule[-0.200pt]{0.400pt}{0.400pt}}%
\begin{picture}(1500,900)(0,0)
\sbox{\plotpoint}{\rule[-0.200pt]{0.400pt}{0.400pt}}%
\put(151.0,131.0){\rule[-0.200pt]{4.818pt}{0.400pt}}
\put(131,131){\makebox(0,0)[r]{$0$}}
\put(1419.0,131.0){\rule[-0.200pt]{4.818pt}{0.400pt}}
\put(151.0,200.0){\rule[-0.200pt]{4.818pt}{0.400pt}}
\put(131,200){\makebox(0,0)[r]{$0.1$}}
\put(1419.0,200.0){\rule[-0.200pt]{4.818pt}{0.400pt}}
\put(151.0,270.0){\rule[-0.200pt]{4.818pt}{0.400pt}}
\put(131,270){\makebox(0,0)[r]{$0.2$}}
\put(1419.0,270.0){\rule[-0.200pt]{4.818pt}{0.400pt}}
\put(151.0,339.0){\rule[-0.200pt]{4.818pt}{0.400pt}}
\put(131,339){\makebox(0,0)[r]{$0.3$}}
\put(1419.0,339.0){\rule[-0.200pt]{4.818pt}{0.400pt}}
\put(151.0,408.0){\rule[-0.200pt]{4.818pt}{0.400pt}}
\put(131,408){\makebox(0,0)[r]{$0.4$}}
\put(1419.0,408.0){\rule[-0.200pt]{4.818pt}{0.400pt}}
\put(151.0,478.0){\rule[-0.200pt]{4.818pt}{0.400pt}}
\put(131,478){\makebox(0,0)[r]{$0.5$}}
\put(1419.0,478.0){\rule[-0.200pt]{4.818pt}{0.400pt}}
\put(151.0,547.0){\rule[-0.200pt]{4.818pt}{0.400pt}}
\put(131,547){\makebox(0,0)[r]{$0.6$}}
\put(1419.0,547.0){\rule[-0.200pt]{4.818pt}{0.400pt}}
\put(151.0,616.0){\rule[-0.200pt]{4.818pt}{0.400pt}}
\put(131,616){\makebox(0,0)[r]{$0.7$}}
\put(1419.0,616.0){\rule[-0.200pt]{4.818pt}{0.400pt}}
\put(151.0,686.0){\rule[-0.200pt]{4.818pt}{0.400pt}}
\put(131,686){\makebox(0,0)[r]{$0.8$}}
\put(1419.0,686.0){\rule[-0.200pt]{4.818pt}{0.400pt}}
\put(151.0,755.0){\rule[-0.200pt]{4.818pt}{0.400pt}}
\put(131,755){\makebox(0,0)[r]{$0.9$}}
\put(1419.0,755.0){\rule[-0.200pt]{4.818pt}{0.400pt}}
\put(151.0,824.0){\rule[-0.200pt]{4.818pt}{0.400pt}}
\put(131,824){\makebox(0,0)[r]{$1$}}
\put(1419.0,824.0){\rule[-0.200pt]{4.818pt}{0.400pt}}
\put(151.0,131.0){\rule[-0.200pt]{0.400pt}{4.818pt}}
\put(151,90){\makebox(0,0){$20$}}
\put(151.0,839.0){\rule[-0.200pt]{0.400pt}{4.818pt}}
\put(243.0,131.0){\rule[-0.200pt]{0.400pt}{4.818pt}}
\put(243,90){\makebox(0,0){$21$}}
\put(243.0,839.0){\rule[-0.200pt]{0.400pt}{4.818pt}}
\put(335.0,131.0){\rule[-0.200pt]{0.400pt}{4.818pt}}
\put(335,90){\makebox(0,0){$22$}}
\put(335.0,839.0){\rule[-0.200pt]{0.400pt}{4.818pt}}
\put(427.0,131.0){\rule[-0.200pt]{0.400pt}{4.818pt}}
\put(427,90){\makebox(0,0){$23$}}
\put(427.0,839.0){\rule[-0.200pt]{0.400pt}{4.818pt}}
\put(519.0,131.0){\rule[-0.200pt]{0.400pt}{4.818pt}}
\put(519,90){\makebox(0,0){$24$}}
\put(519.0,839.0){\rule[-0.200pt]{0.400pt}{4.818pt}}
\put(611.0,131.0){\rule[-0.200pt]{0.400pt}{4.818pt}}
\put(611,90){\makebox(0,0){$25$}}
\put(611.0,839.0){\rule[-0.200pt]{0.400pt}{4.818pt}}
\put(703.0,131.0){\rule[-0.200pt]{0.400pt}{4.818pt}}
\put(703,90){\makebox(0,0){$26$}}
\put(703.0,839.0){\rule[-0.200pt]{0.400pt}{4.818pt}}
\put(795.0,131.0){\rule[-0.200pt]{0.400pt}{4.818pt}}
\put(795,90){\makebox(0,0){$27$}}
\put(795.0,839.0){\rule[-0.200pt]{0.400pt}{4.818pt}}
\put(887.0,131.0){\rule[-0.200pt]{0.400pt}{4.818pt}}
\put(887,90){\makebox(0,0){$28$}}
\put(887.0,839.0){\rule[-0.200pt]{0.400pt}{4.818pt}}
\put(979.0,131.0){\rule[-0.200pt]{0.400pt}{4.818pt}}
\put(979,90){\makebox(0,0){$29$}}
\put(979.0,839.0){\rule[-0.200pt]{0.400pt}{4.818pt}}
\put(1071.0,131.0){\rule[-0.200pt]{0.400pt}{4.818pt}}
\put(1071,90){\makebox(0,0){$30$}}
\put(1071.0,839.0){\rule[-0.200pt]{0.400pt}{4.818pt}}
\put(1163.0,131.0){\rule[-0.200pt]{0.400pt}{4.818pt}}
\put(1163,90){\makebox(0,0){$31$}}
\put(1163.0,839.0){\rule[-0.200pt]{0.400pt}{4.818pt}}
\put(1255.0,131.0){\rule[-0.200pt]{0.400pt}{4.818pt}}
\put(1255,90){\makebox(0,0){$32$}}
\put(1255.0,839.0){\rule[-0.200pt]{0.400pt}{4.818pt}}
\put(1347.0,131.0){\rule[-0.200pt]{0.400pt}{4.818pt}}
\put(1347,90){\makebox(0,0){$33$}}
\put(1347.0,839.0){\rule[-0.200pt]{0.400pt}{4.818pt}}
\put(1439.0,131.0){\rule[-0.200pt]{0.400pt}{4.818pt}}
\put(1439,90){\makebox(0,0){$34$}}
\put(1439.0,839.0){\rule[-0.200pt]{0.400pt}{4.818pt}}
\put(151.0,131.0){\rule[-0.200pt]{0.400pt}{175.375pt}}
\put(151.0,131.0){\rule[-0.200pt]{310.279pt}{0.400pt}}
\put(1439.0,131.0){\rule[-0.200pt]{0.400pt}{175.375pt}}
\put(151.0,859.0){\rule[-0.200pt]{310.279pt}{0.400pt}}
\put(30,495){\makebox(0,0){$1-\beta$}}
\put(795,29){\makebox(0,0){$\mu$}}
\put(1279,212){\makebox(0,0)[r]{$1-\beta_b$}}
\multiput(1299,212)(20.756,0.000){5}{\usebox{\plotpoint}}
\put(1399,212){\usebox{\plotpoint}}
\put(151,824){\usebox{\plotpoint}}
\put(151.00,824.00){\usebox{\plotpoint}}
\put(171.76,824.00){\usebox{\plotpoint}}
\put(192.51,824.00){\usebox{\plotpoint}}
\put(213.27,824.00){\usebox{\plotpoint}}
\put(234.02,824.00){\usebox{\plotpoint}}
\put(254.78,824.00){\usebox{\plotpoint}}
\put(275.53,824.00){\usebox{\plotpoint}}
\put(296.29,824.00){\usebox{\plotpoint}}
\put(316.63,823.00){\usebox{\plotpoint}}
\put(336.97,822.00){\usebox{\plotpoint}}
\put(356.90,820.00){\usebox{\plotpoint}}
\put(376.41,817.00){\usebox{\plotpoint}}
\put(395.65,813.35){\usebox{\plotpoint}}
\put(413.84,807.16){\usebox{\plotpoint}}
\put(431.64,800.00){\usebox{\plotpoint}}
\put(447.76,790.24){\usebox{\plotpoint}}
\put(463.20,778.80){\usebox{\plotpoint}}
\put(477.75,765.25){\usebox{\plotpoint}}
\put(491.01,749.99){\usebox{\plotpoint}}
\put(503.11,733.89){\usebox{\plotpoint}}
\put(514.40,717.20){\usebox{\plotpoint}}
\put(525.07,699.86){\usebox{\plotpoint}}
\put(535.72,682.28){\usebox{\plotpoint}}
\put(544.95,664.11){\usebox{\plotpoint}}
\put(553.70,645.60){\usebox{\plotpoint}}
\put(562.06,626.81){\usebox{\plotpoint}}
\put(570.96,608.08){\usebox{\plotpoint}}
\put(578.48,589.03){\usebox{\plotpoint}}
\put(586.04,569.91){\usebox{\plotpoint}}
\put(594.00,550.89){\usebox{\plotpoint}}
\put(601.43,531.71){\usebox{\plotpoint}}
\put(608.53,512.40){\usebox{\plotpoint}}
\put(616.38,493.24){\usebox{\plotpoint}}
\put(623.52,473.95){\usebox{\plotpoint}}
\put(630.47,454.58){\usebox{\plotpoint}}
\put(638.50,435.49){\usebox{\plotpoint}}
\put(645.58,416.26){\usebox{\plotpoint}}
\put(652.97,397.07){\usebox{\plotpoint}}
\put(661.01,377.99){\usebox{\plotpoint}}
\put(668.53,358.93){\usebox{\plotpoint}}
\put(676.51,339.99){\usebox{\plotpoint}}
\put(685.37,321.25){\usebox{\plotpoint}}
\put(694.13,302.74){\usebox{\plotpoint}}
\put(702.88,284.23){\usebox{\plotpoint}}
\put(712.45,266.09){\usebox{\plotpoint}}
\put(722.31,248.38){\usebox{\plotpoint}}
\put(732.99,231.03){\usebox{\plotpoint}}
\put(745.13,214.74){\usebox{\plotpoint}}
\put(758.01,198.99){\usebox{\plotpoint}}
\put(771.98,184.02){\usebox{\plotpoint}}
\put(787.12,171.88){\usebox{\plotpoint}}
\put(802.97,171.97){\usebox{\plotpoint}}
\put(818.11,184.11){\usebox{\plotpoint}}
\put(832.08,199.08){\usebox{\plotpoint}}
\put(844.93,214.85){\usebox{\plotpoint}}
\put(857.07,231.14){\usebox{\plotpoint}}
\put(867.74,248.48){\usebox{\plotpoint}}
\put(877.60,266.20){\usebox{\plotpoint}}
\put(887.17,284.34){\usebox{\plotpoint}}
\put(895.93,302.85){\usebox{\plotpoint}}
\put(904.68,321.36){\usebox{\plotpoint}}
\put(913.55,340.10){\usebox{\plotpoint}}
\put(921.52,359.04){\usebox{\plotpoint}}
\put(929.05,378.10){\usebox{\plotpoint}}
\put(937.09,397.18){\usebox{\plotpoint}}
\put(944.46,416.37){\usebox{\plotpoint}}
\put(951.54,435.61){\usebox{\plotpoint}}
\put(959.56,454.69){\usebox{\plotpoint}}
\put(966.53,474.06){\usebox{\plotpoint}}
\put(973.67,493.35){\usebox{\plotpoint}}
\put(981.50,512.51){\usebox{\plotpoint}}
\put(988.61,531.82){\usebox{\plotpoint}}
\put(996.00,551.01){\usebox{\plotpoint}}
\put(1004.01,570.02){\usebox{\plotpoint}}
\put(1011.57,589.14){\usebox{\plotpoint}}
\put(1019.10,608.19){\usebox{\plotpoint}}
\put(1027.97,626.92){\usebox{\plotpoint}}
\put(1036.35,645.70){\usebox{\plotpoint}}
\put(1045.11,664.21){\usebox{\plotpoint}}
\put(1054.36,682.36){\usebox{\plotpoint}}
\put(1064.98,699.96){\usebox{\plotpoint}}
\put(1075.65,717.31){\usebox{\plotpoint}}
\put(1086.97,733.97){\usebox{\plotpoint}}
\put(1099.07,750.07){\usebox{\plotpoint}}
\put(1112.34,765.34){\usebox{\plotpoint}}
\put(1126.89,778.89){\usebox{\plotpoint}}
\put(1142.32,790.32){\usebox{\plotpoint}}
\put(1158.49,800.00){\usebox{\plotpoint}}
\put(1176.24,807.24){\usebox{\plotpoint}}
\put(1194.43,813.43){\usebox{\plotpoint}}
\put(1213.71,817.00){\usebox{\plotpoint}}
\put(1233.22,820.00){\usebox{\plotpoint}}
\put(1253.15,822.00){\usebox{\plotpoint}}
\put(1273.49,823.00){\usebox{\plotpoint}}
\put(1293.83,824.00){\usebox{\plotpoint}}
\put(1314.59,824.00){\usebox{\plotpoint}}
\put(1335.34,824.00){\usebox{\plotpoint}}
\put(1356.10,824.00){\usebox{\plotpoint}}
\put(1376.85,824.00){\usebox{\plotpoint}}
\put(1397.61,824.00){\usebox{\plotpoint}}
\put(1418.37,824.00){\usebox{\plotpoint}}
\put(1439,824){\usebox{\plotpoint}}
\put(1279,171){\makebox(0,0)[r]{$1-\beta_d$}}
\put(1299.0,171.0){\rule[-0.200pt]{24.090pt}{0.400pt}}
\put(151,824){\usebox{\plotpoint}}
\put(428,822.67){\rule{0.241pt}{0.400pt}}
\multiput(428.00,823.17)(0.500,-1.000){2}{\rule{0.120pt}{0.400pt}}
\put(151.0,824.0){\rule[-0.200pt]{66.729pt}{0.400pt}}
\put(447,821.67){\rule{0.241pt}{0.400pt}}
\multiput(447.00,822.17)(0.500,-1.000){2}{\rule{0.120pt}{0.400pt}}
\put(429.0,823.0){\rule[-0.200pt]{4.336pt}{0.400pt}}
\put(458,820.67){\rule{0.241pt}{0.400pt}}
\multiput(458.00,821.17)(0.500,-1.000){2}{\rule{0.120pt}{0.400pt}}
\put(448.0,822.0){\rule[-0.200pt]{2.409pt}{0.400pt}}
\put(459.0,821.0){\rule[-0.200pt]{1.927pt}{0.400pt}}
\put(467.0,820.0){\usebox{\plotpoint}}
\put(473,818.67){\rule{0.241pt}{0.400pt}}
\multiput(473.00,819.17)(0.500,-1.000){2}{\rule{0.120pt}{0.400pt}}
\put(467.0,820.0){\rule[-0.200pt]{1.445pt}{0.400pt}}
\put(474.0,819.0){\rule[-0.200pt]{1.204pt}{0.400pt}}
\put(479.0,818.0){\usebox{\plotpoint}}
\put(483,816.67){\rule{0.241pt}{0.400pt}}
\multiput(483.00,817.17)(0.500,-1.000){2}{\rule{0.120pt}{0.400pt}}
\put(479.0,818.0){\rule[-0.200pt]{0.964pt}{0.400pt}}
\put(487,815.67){\rule{0.241pt}{0.400pt}}
\multiput(487.00,816.17)(0.500,-1.000){2}{\rule{0.120pt}{0.400pt}}
\put(484.0,817.0){\rule[-0.200pt]{0.723pt}{0.400pt}}
\put(490,814.67){\rule{0.241pt}{0.400pt}}
\multiput(490.00,815.17)(0.500,-1.000){2}{\rule{0.120pt}{0.400pt}}
\put(488.0,816.0){\rule[-0.200pt]{0.482pt}{0.400pt}}
\put(493,813.67){\rule{0.241pt}{0.400pt}}
\multiput(493.00,814.17)(0.500,-1.000){2}{\rule{0.120pt}{0.400pt}}
\put(491.0,815.0){\rule[-0.200pt]{0.482pt}{0.400pt}}
\put(497,812.67){\rule{0.241pt}{0.400pt}}
\multiput(497.00,813.17)(0.500,-1.000){2}{\rule{0.120pt}{0.400pt}}
\put(494.0,814.0){\rule[-0.200pt]{0.723pt}{0.400pt}}
\put(500,811.67){\rule{0.241pt}{0.400pt}}
\multiput(500.00,812.17)(0.500,-1.000){2}{\rule{0.120pt}{0.400pt}}
\put(498.0,813.0){\rule[-0.200pt]{0.482pt}{0.400pt}}
\put(502,810.67){\rule{0.241pt}{0.400pt}}
\multiput(502.00,811.17)(0.500,-1.000){2}{\rule{0.120pt}{0.400pt}}
\put(501.0,812.0){\usebox{\plotpoint}}
\put(504,809.67){\rule{0.241pt}{0.400pt}}
\multiput(504.00,810.17)(0.500,-1.000){2}{\rule{0.120pt}{0.400pt}}
\put(503.0,811.0){\usebox{\plotpoint}}
\put(507,808.67){\rule{0.241pt}{0.400pt}}
\multiput(507.00,809.17)(0.500,-1.000){2}{\rule{0.120pt}{0.400pt}}
\put(505.0,810.0){\rule[-0.200pt]{0.482pt}{0.400pt}}
\put(509,807.67){\rule{0.241pt}{0.400pt}}
\multiput(509.00,808.17)(0.500,-1.000){2}{\rule{0.120pt}{0.400pt}}
\put(508.0,809.0){\usebox{\plotpoint}}
\put(511,806.67){\rule{0.241pt}{0.400pt}}
\multiput(511.00,807.17)(0.500,-1.000){2}{\rule{0.120pt}{0.400pt}}
\put(510.0,808.0){\usebox{\plotpoint}}
\put(512.0,807.0){\usebox{\plotpoint}}
\put(513.0,806.0){\usebox{\plotpoint}}
\put(514,804.67){\rule{0.241pt}{0.400pt}}
\multiput(514.00,805.17)(0.500,-1.000){2}{\rule{0.120pt}{0.400pt}}
\put(513.0,806.0){\usebox{\plotpoint}}
\put(516,803.67){\rule{0.241pt}{0.400pt}}
\multiput(516.00,804.17)(0.500,-1.000){2}{\rule{0.120pt}{0.400pt}}
\put(515.0,805.0){\usebox{\plotpoint}}
\put(518,802.67){\rule{0.241pt}{0.400pt}}
\multiput(518.00,803.17)(0.500,-1.000){2}{\rule{0.120pt}{0.400pt}}
\put(517.0,804.0){\usebox{\plotpoint}}
\put(520,801.67){\rule{0.241pt}{0.400pt}}
\multiput(520.00,802.17)(0.500,-1.000){2}{\rule{0.120pt}{0.400pt}}
\put(519.0,803.0){\usebox{\plotpoint}}
\put(522,800.67){\rule{0.241pt}{0.400pt}}
\multiput(522.00,801.17)(0.500,-1.000){2}{\rule{0.120pt}{0.400pt}}
\put(523,799.67){\rule{0.241pt}{0.400pt}}
\multiput(523.00,800.17)(0.500,-1.000){2}{\rule{0.120pt}{0.400pt}}
\put(521.0,802.0){\usebox{\plotpoint}}
\put(524.0,800.0){\usebox{\plotpoint}}
\put(525,797.67){\rule{0.241pt}{0.400pt}}
\multiput(525.00,798.17)(0.500,-1.000){2}{\rule{0.120pt}{0.400pt}}
\put(525.0,799.0){\usebox{\plotpoint}}
\put(527,796.67){\rule{0.241pt}{0.400pt}}
\multiput(527.00,797.17)(0.500,-1.000){2}{\rule{0.120pt}{0.400pt}}
\put(528,795.67){\rule{0.241pt}{0.400pt}}
\multiput(528.00,796.17)(0.500,-1.000){2}{\rule{0.120pt}{0.400pt}}
\put(526.0,798.0){\usebox{\plotpoint}}
\put(530,794.67){\rule{0.241pt}{0.400pt}}
\multiput(530.00,795.17)(0.500,-1.000){2}{\rule{0.120pt}{0.400pt}}
\put(531,793.67){\rule{0.241pt}{0.400pt}}
\multiput(531.00,794.17)(0.500,-1.000){2}{\rule{0.120pt}{0.400pt}}
\put(532,792.67){\rule{0.241pt}{0.400pt}}
\multiput(532.00,793.17)(0.500,-1.000){2}{\rule{0.120pt}{0.400pt}}
\put(529.0,796.0){\usebox{\plotpoint}}
\put(534,791.67){\rule{0.241pt}{0.400pt}}
\multiput(534.00,792.17)(0.500,-1.000){2}{\rule{0.120pt}{0.400pt}}
\put(535,790.67){\rule{0.241pt}{0.400pt}}
\multiput(535.00,791.17)(0.500,-1.000){2}{\rule{0.120pt}{0.400pt}}
\put(533.0,793.0){\usebox{\plotpoint}}
\put(536,788.67){\rule{0.241pt}{0.400pt}}
\multiput(536.00,789.17)(0.500,-1.000){2}{\rule{0.120pt}{0.400pt}}
\put(536.0,790.0){\usebox{\plotpoint}}
\put(538,787.67){\rule{0.241pt}{0.400pt}}
\multiput(538.00,788.17)(0.500,-1.000){2}{\rule{0.120pt}{0.400pt}}
\put(539,786.67){\rule{0.241pt}{0.400pt}}
\multiput(539.00,787.17)(0.500,-1.000){2}{\rule{0.120pt}{0.400pt}}
\put(540,785.67){\rule{0.241pt}{0.400pt}}
\multiput(540.00,786.17)(0.500,-1.000){2}{\rule{0.120pt}{0.400pt}}
\put(541,784.67){\rule{0.241pt}{0.400pt}}
\multiput(541.00,785.17)(0.500,-1.000){2}{\rule{0.120pt}{0.400pt}}
\put(542,783.67){\rule{0.241pt}{0.400pt}}
\multiput(542.00,784.17)(0.500,-1.000){2}{\rule{0.120pt}{0.400pt}}
\put(543,782.67){\rule{0.241pt}{0.400pt}}
\multiput(543.00,783.17)(0.500,-1.000){2}{\rule{0.120pt}{0.400pt}}
\put(544,781.67){\rule{0.241pt}{0.400pt}}
\multiput(544.00,782.17)(0.500,-1.000){2}{\rule{0.120pt}{0.400pt}}
\put(545,780.67){\rule{0.241pt}{0.400pt}}
\multiput(545.00,781.17)(0.500,-1.000){2}{\rule{0.120pt}{0.400pt}}
\put(546,779.67){\rule{0.241pt}{0.400pt}}
\multiput(546.00,780.17)(0.500,-1.000){2}{\rule{0.120pt}{0.400pt}}
\put(547,778.67){\rule{0.241pt}{0.400pt}}
\multiput(547.00,779.17)(0.500,-1.000){2}{\rule{0.120pt}{0.400pt}}
\put(537.0,789.0){\usebox{\plotpoint}}
\put(548,776.67){\rule{0.241pt}{0.400pt}}
\multiput(548.00,777.17)(0.500,-1.000){2}{\rule{0.120pt}{0.400pt}}
\put(549,775.67){\rule{0.241pt}{0.400pt}}
\multiput(549.00,776.17)(0.500,-1.000){2}{\rule{0.120pt}{0.400pt}}
\put(550,774.67){\rule{0.241pt}{0.400pt}}
\multiput(550.00,775.17)(0.500,-1.000){2}{\rule{0.120pt}{0.400pt}}
\put(551,773.67){\rule{0.241pt}{0.400pt}}
\multiput(551.00,774.17)(0.500,-1.000){2}{\rule{0.120pt}{0.400pt}}
\put(552,772.67){\rule{0.241pt}{0.400pt}}
\multiput(552.00,773.17)(0.500,-1.000){2}{\rule{0.120pt}{0.400pt}}
\put(552.67,771){\rule{0.400pt}{0.482pt}}
\multiput(552.17,772.00)(1.000,-1.000){2}{\rule{0.400pt}{0.241pt}}
\put(554,769.67){\rule{0.241pt}{0.400pt}}
\multiput(554.00,770.17)(0.500,-1.000){2}{\rule{0.120pt}{0.400pt}}
\put(555,768.67){\rule{0.241pt}{0.400pt}}
\multiput(555.00,769.17)(0.500,-1.000){2}{\rule{0.120pt}{0.400pt}}
\put(556,767.67){\rule{0.241pt}{0.400pt}}
\multiput(556.00,768.17)(0.500,-1.000){2}{\rule{0.120pt}{0.400pt}}
\put(557,766.67){\rule{0.241pt}{0.400pt}}
\multiput(557.00,767.17)(0.500,-1.000){2}{\rule{0.120pt}{0.400pt}}
\put(557.67,765){\rule{0.400pt}{0.482pt}}
\multiput(557.17,766.00)(1.000,-1.000){2}{\rule{0.400pt}{0.241pt}}
\put(548.0,778.0){\usebox{\plotpoint}}
\put(559,762.67){\rule{0.241pt}{0.400pt}}
\multiput(559.00,763.17)(0.500,-1.000){2}{\rule{0.120pt}{0.400pt}}
\put(559.67,761){\rule{0.400pt}{0.482pt}}
\multiput(559.17,762.00)(1.000,-1.000){2}{\rule{0.400pt}{0.241pt}}
\put(561,759.67){\rule{0.241pt}{0.400pt}}
\multiput(561.00,760.17)(0.500,-1.000){2}{\rule{0.120pt}{0.400pt}}
\put(562,758.67){\rule{0.241pt}{0.400pt}}
\multiput(562.00,759.17)(0.500,-1.000){2}{\rule{0.120pt}{0.400pt}}
\put(562.67,757){\rule{0.400pt}{0.482pt}}
\multiput(562.17,758.00)(1.000,-1.000){2}{\rule{0.400pt}{0.241pt}}
\put(564,755.67){\rule{0.241pt}{0.400pt}}
\multiput(564.00,756.17)(0.500,-1.000){2}{\rule{0.120pt}{0.400pt}}
\put(564.67,754){\rule{0.400pt}{0.482pt}}
\multiput(564.17,755.00)(1.000,-1.000){2}{\rule{0.400pt}{0.241pt}}
\put(566,752.67){\rule{0.241pt}{0.400pt}}
\multiput(566.00,753.17)(0.500,-1.000){2}{\rule{0.120pt}{0.400pt}}
\put(567,751.67){\rule{0.241pt}{0.400pt}}
\multiput(567.00,752.17)(0.500,-1.000){2}{\rule{0.120pt}{0.400pt}}
\put(567.67,750){\rule{0.400pt}{0.482pt}}
\multiput(567.17,751.00)(1.000,-1.000){2}{\rule{0.400pt}{0.241pt}}
\put(569,748.67){\rule{0.241pt}{0.400pt}}
\multiput(569.00,749.17)(0.500,-1.000){2}{\rule{0.120pt}{0.400pt}}
\put(569.67,747){\rule{0.400pt}{0.482pt}}
\multiput(569.17,748.00)(1.000,-1.000){2}{\rule{0.400pt}{0.241pt}}
\put(559.0,764.0){\usebox{\plotpoint}}
\put(571,743.67){\rule{0.241pt}{0.400pt}}
\multiput(571.00,744.17)(0.500,-1.000){2}{\rule{0.120pt}{0.400pt}}
\put(571.67,742){\rule{0.400pt}{0.482pt}}
\multiput(571.17,743.00)(1.000,-1.000){2}{\rule{0.400pt}{0.241pt}}
\put(573,740.67){\rule{0.241pt}{0.400pt}}
\multiput(573.00,741.17)(0.500,-1.000){2}{\rule{0.120pt}{0.400pt}}
\put(573.67,739){\rule{0.400pt}{0.482pt}}
\multiput(573.17,740.00)(1.000,-1.000){2}{\rule{0.400pt}{0.241pt}}
\put(574.67,737){\rule{0.400pt}{0.482pt}}
\multiput(574.17,738.00)(1.000,-1.000){2}{\rule{0.400pt}{0.241pt}}
\put(576,735.67){\rule{0.241pt}{0.400pt}}
\multiput(576.00,736.17)(0.500,-1.000){2}{\rule{0.120pt}{0.400pt}}
\put(576.67,734){\rule{0.400pt}{0.482pt}}
\multiput(576.17,735.00)(1.000,-1.000){2}{\rule{0.400pt}{0.241pt}}
\put(577.67,732){\rule{0.400pt}{0.482pt}}
\multiput(577.17,733.00)(1.000,-1.000){2}{\rule{0.400pt}{0.241pt}}
\put(578.67,730){\rule{0.400pt}{0.482pt}}
\multiput(578.17,731.00)(1.000,-1.000){2}{\rule{0.400pt}{0.241pt}}
\put(580,728.67){\rule{0.241pt}{0.400pt}}
\multiput(580.00,729.17)(0.500,-1.000){2}{\rule{0.120pt}{0.400pt}}
\put(580.67,727){\rule{0.400pt}{0.482pt}}
\multiput(580.17,728.00)(1.000,-1.000){2}{\rule{0.400pt}{0.241pt}}
\put(571.0,745.0){\rule[-0.200pt]{0.400pt}{0.482pt}}
\put(581.67,723){\rule{0.400pt}{0.482pt}}
\multiput(581.17,724.00)(1.000,-1.000){2}{\rule{0.400pt}{0.241pt}}
\put(582.67,721){\rule{0.400pt}{0.482pt}}
\multiput(582.17,722.00)(1.000,-1.000){2}{\rule{0.400pt}{0.241pt}}
\put(583.67,719){\rule{0.400pt}{0.482pt}}
\multiput(583.17,720.00)(1.000,-1.000){2}{\rule{0.400pt}{0.241pt}}
\put(584.67,717){\rule{0.400pt}{0.482pt}}
\multiput(584.17,718.00)(1.000,-1.000){2}{\rule{0.400pt}{0.241pt}}
\put(586,715.67){\rule{0.241pt}{0.400pt}}
\multiput(586.00,716.17)(0.500,-1.000){2}{\rule{0.120pt}{0.400pt}}
\put(586.67,714){\rule{0.400pt}{0.482pt}}
\multiput(586.17,715.00)(1.000,-1.000){2}{\rule{0.400pt}{0.241pt}}
\put(587.67,712){\rule{0.400pt}{0.482pt}}
\multiput(587.17,713.00)(1.000,-1.000){2}{\rule{0.400pt}{0.241pt}}
\put(588.67,710){\rule{0.400pt}{0.482pt}}
\multiput(588.17,711.00)(1.000,-1.000){2}{\rule{0.400pt}{0.241pt}}
\put(589.67,708){\rule{0.400pt}{0.482pt}}
\multiput(589.17,709.00)(1.000,-1.000){2}{\rule{0.400pt}{0.241pt}}
\put(590.67,705){\rule{0.400pt}{0.723pt}}
\multiput(590.17,706.50)(1.000,-1.500){2}{\rule{0.400pt}{0.361pt}}
\put(591.67,703){\rule{0.400pt}{0.482pt}}
\multiput(591.17,704.00)(1.000,-1.000){2}{\rule{0.400pt}{0.241pt}}
\put(592.67,701){\rule{0.400pt}{0.482pt}}
\multiput(592.17,702.00)(1.000,-1.000){2}{\rule{0.400pt}{0.241pt}}
\put(582.0,725.0){\rule[-0.200pt]{0.400pt}{0.482pt}}
\put(593.67,697){\rule{0.400pt}{0.482pt}}
\multiput(593.17,698.00)(1.000,-1.000){2}{\rule{0.400pt}{0.241pt}}
\put(594.67,695){\rule{0.400pt}{0.482pt}}
\multiput(594.17,696.00)(1.000,-1.000){2}{\rule{0.400pt}{0.241pt}}
\put(595.67,693){\rule{0.400pt}{0.482pt}}
\multiput(595.17,694.00)(1.000,-1.000){2}{\rule{0.400pt}{0.241pt}}
\put(596.67,690){\rule{0.400pt}{0.723pt}}
\multiput(596.17,691.50)(1.000,-1.500){2}{\rule{0.400pt}{0.361pt}}
\put(597.67,688){\rule{0.400pt}{0.482pt}}
\multiput(597.17,689.00)(1.000,-1.000){2}{\rule{0.400pt}{0.241pt}}
\put(598.67,686){\rule{0.400pt}{0.482pt}}
\multiput(598.17,687.00)(1.000,-1.000){2}{\rule{0.400pt}{0.241pt}}
\put(599.67,684){\rule{0.400pt}{0.482pt}}
\multiput(599.17,685.00)(1.000,-1.000){2}{\rule{0.400pt}{0.241pt}}
\put(600.67,681){\rule{0.400pt}{0.723pt}}
\multiput(600.17,682.50)(1.000,-1.500){2}{\rule{0.400pt}{0.361pt}}
\put(601.67,679){\rule{0.400pt}{0.482pt}}
\multiput(601.17,680.00)(1.000,-1.000){2}{\rule{0.400pt}{0.241pt}}
\put(602.67,677){\rule{0.400pt}{0.482pt}}
\multiput(602.17,678.00)(1.000,-1.000){2}{\rule{0.400pt}{0.241pt}}
\put(603.67,674){\rule{0.400pt}{0.723pt}}
\multiput(603.17,675.50)(1.000,-1.500){2}{\rule{0.400pt}{0.361pt}}
\put(594.0,699.0){\rule[-0.200pt]{0.400pt}{0.482pt}}
\put(604.67,669){\rule{0.400pt}{0.723pt}}
\multiput(604.17,670.50)(1.000,-1.500){2}{\rule{0.400pt}{0.361pt}}
\put(605.67,667){\rule{0.400pt}{0.482pt}}
\multiput(605.17,668.00)(1.000,-1.000){2}{\rule{0.400pt}{0.241pt}}
\put(606.67,665){\rule{0.400pt}{0.482pt}}
\multiput(606.17,666.00)(1.000,-1.000){2}{\rule{0.400pt}{0.241pt}}
\put(607.67,662){\rule{0.400pt}{0.723pt}}
\multiput(607.17,663.50)(1.000,-1.500){2}{\rule{0.400pt}{0.361pt}}
\put(608.67,660){\rule{0.400pt}{0.482pt}}
\multiput(608.17,661.00)(1.000,-1.000){2}{\rule{0.400pt}{0.241pt}}
\put(609.67,657){\rule{0.400pt}{0.723pt}}
\multiput(609.17,658.50)(1.000,-1.500){2}{\rule{0.400pt}{0.361pt}}
\put(610.67,654){\rule{0.400pt}{0.723pt}}
\multiput(610.17,655.50)(1.000,-1.500){2}{\rule{0.400pt}{0.361pt}}
\put(611.67,652){\rule{0.400pt}{0.482pt}}
\multiput(611.17,653.00)(1.000,-1.000){2}{\rule{0.400pt}{0.241pt}}
\put(612.67,649){\rule{0.400pt}{0.723pt}}
\multiput(612.17,650.50)(1.000,-1.500){2}{\rule{0.400pt}{0.361pt}}
\put(613.67,647){\rule{0.400pt}{0.482pt}}
\multiput(613.17,648.00)(1.000,-1.000){2}{\rule{0.400pt}{0.241pt}}
\put(614.67,644){\rule{0.400pt}{0.723pt}}
\multiput(614.17,645.50)(1.000,-1.500){2}{\rule{0.400pt}{0.361pt}}
\put(615.67,641){\rule{0.400pt}{0.723pt}}
\multiput(615.17,642.50)(1.000,-1.500){2}{\rule{0.400pt}{0.361pt}}
\put(605.0,672.0){\rule[-0.200pt]{0.400pt}{0.482pt}}
\put(616.67,636){\rule{0.400pt}{0.723pt}}
\multiput(616.17,637.50)(1.000,-1.500){2}{\rule{0.400pt}{0.361pt}}
\put(617.67,633){\rule{0.400pt}{0.723pt}}
\multiput(617.17,634.50)(1.000,-1.500){2}{\rule{0.400pt}{0.361pt}}
\put(618.67,631){\rule{0.400pt}{0.482pt}}
\multiput(618.17,632.00)(1.000,-1.000){2}{\rule{0.400pt}{0.241pt}}
\put(619.67,628){\rule{0.400pt}{0.723pt}}
\multiput(619.17,629.50)(1.000,-1.500){2}{\rule{0.400pt}{0.361pt}}
\put(620.67,625){\rule{0.400pt}{0.723pt}}
\multiput(620.17,626.50)(1.000,-1.500){2}{\rule{0.400pt}{0.361pt}}
\put(621.67,622){\rule{0.400pt}{0.723pt}}
\multiput(621.17,623.50)(1.000,-1.500){2}{\rule{0.400pt}{0.361pt}}
\put(622.67,620){\rule{0.400pt}{0.482pt}}
\multiput(622.17,621.00)(1.000,-1.000){2}{\rule{0.400pt}{0.241pt}}
\put(623.67,617){\rule{0.400pt}{0.723pt}}
\multiput(623.17,618.50)(1.000,-1.500){2}{\rule{0.400pt}{0.361pt}}
\put(624.67,614){\rule{0.400pt}{0.723pt}}
\multiput(624.17,615.50)(1.000,-1.500){2}{\rule{0.400pt}{0.361pt}}
\put(625.67,611){\rule{0.400pt}{0.723pt}}
\multiput(625.17,612.50)(1.000,-1.500){2}{\rule{0.400pt}{0.361pt}}
\put(626.67,608){\rule{0.400pt}{0.723pt}}
\multiput(626.17,609.50)(1.000,-1.500){2}{\rule{0.400pt}{0.361pt}}
\put(617.0,639.0){\rule[-0.200pt]{0.400pt}{0.482pt}}
\put(627.67,602){\rule{0.400pt}{0.723pt}}
\multiput(627.17,603.50)(1.000,-1.500){2}{\rule{0.400pt}{0.361pt}}
\put(628.67,599){\rule{0.400pt}{0.723pt}}
\multiput(628.17,600.50)(1.000,-1.500){2}{\rule{0.400pt}{0.361pt}}
\put(629.67,597){\rule{0.400pt}{0.482pt}}
\multiput(629.17,598.00)(1.000,-1.000){2}{\rule{0.400pt}{0.241pt}}
\put(630.67,594){\rule{0.400pt}{0.723pt}}
\multiput(630.17,595.50)(1.000,-1.500){2}{\rule{0.400pt}{0.361pt}}
\put(631.67,591){\rule{0.400pt}{0.723pt}}
\multiput(631.17,592.50)(1.000,-1.500){2}{\rule{0.400pt}{0.361pt}}
\put(632.67,588){\rule{0.400pt}{0.723pt}}
\multiput(632.17,589.50)(1.000,-1.500){2}{\rule{0.400pt}{0.361pt}}
\put(633.67,585){\rule{0.400pt}{0.723pt}}
\multiput(633.17,586.50)(1.000,-1.500){2}{\rule{0.400pt}{0.361pt}}
\put(634.67,582){\rule{0.400pt}{0.723pt}}
\multiput(634.17,583.50)(1.000,-1.500){2}{\rule{0.400pt}{0.361pt}}
\put(635.67,579){\rule{0.400pt}{0.723pt}}
\multiput(635.17,580.50)(1.000,-1.500){2}{\rule{0.400pt}{0.361pt}}
\put(636.67,576){\rule{0.400pt}{0.723pt}}
\multiput(636.17,577.50)(1.000,-1.500){2}{\rule{0.400pt}{0.361pt}}
\put(637.67,573){\rule{0.400pt}{0.723pt}}
\multiput(637.17,574.50)(1.000,-1.500){2}{\rule{0.400pt}{0.361pt}}
\put(638.67,570){\rule{0.400pt}{0.723pt}}
\multiput(638.17,571.50)(1.000,-1.500){2}{\rule{0.400pt}{0.361pt}}
\put(628.0,605.0){\rule[-0.200pt]{0.400pt}{0.723pt}}
\put(639.67,563){\rule{0.400pt}{0.723pt}}
\multiput(639.17,564.50)(1.000,-1.500){2}{\rule{0.400pt}{0.361pt}}
\put(640.67,560){\rule{0.400pt}{0.723pt}}
\multiput(640.17,561.50)(1.000,-1.500){2}{\rule{0.400pt}{0.361pt}}
\put(641.67,557){\rule{0.400pt}{0.723pt}}
\multiput(641.17,558.50)(1.000,-1.500){2}{\rule{0.400pt}{0.361pt}}
\put(642.67,554){\rule{0.400pt}{0.723pt}}
\multiput(642.17,555.50)(1.000,-1.500){2}{\rule{0.400pt}{0.361pt}}
\put(643.67,551){\rule{0.400pt}{0.723pt}}
\multiput(643.17,552.50)(1.000,-1.500){2}{\rule{0.400pt}{0.361pt}}
\put(644.67,548){\rule{0.400pt}{0.723pt}}
\multiput(644.17,549.50)(1.000,-1.500){2}{\rule{0.400pt}{0.361pt}}
\put(645.67,545){\rule{0.400pt}{0.723pt}}
\multiput(645.17,546.50)(1.000,-1.500){2}{\rule{0.400pt}{0.361pt}}
\put(646.67,541){\rule{0.400pt}{0.964pt}}
\multiput(646.17,543.00)(1.000,-2.000){2}{\rule{0.400pt}{0.482pt}}
\put(647.67,538){\rule{0.400pt}{0.723pt}}
\multiput(647.17,539.50)(1.000,-1.500){2}{\rule{0.400pt}{0.361pt}}
\put(648.67,535){\rule{0.400pt}{0.723pt}}
\multiput(648.17,536.50)(1.000,-1.500){2}{\rule{0.400pt}{0.361pt}}
\put(649.67,532){\rule{0.400pt}{0.723pt}}
\multiput(649.17,533.50)(1.000,-1.500){2}{\rule{0.400pt}{0.361pt}}
\put(640.0,566.0){\rule[-0.200pt]{0.400pt}{0.964pt}}
\put(650.67,526){\rule{0.400pt}{0.723pt}}
\multiput(650.17,527.50)(1.000,-1.500){2}{\rule{0.400pt}{0.361pt}}
\put(651.67,522){\rule{0.400pt}{0.964pt}}
\multiput(651.17,524.00)(1.000,-2.000){2}{\rule{0.400pt}{0.482pt}}
\put(652.67,519){\rule{0.400pt}{0.723pt}}
\multiput(652.17,520.50)(1.000,-1.500){2}{\rule{0.400pt}{0.361pt}}
\put(653.67,516){\rule{0.400pt}{0.723pt}}
\multiput(653.17,517.50)(1.000,-1.500){2}{\rule{0.400pt}{0.361pt}}
\put(654.67,513){\rule{0.400pt}{0.723pt}}
\multiput(654.17,514.50)(1.000,-1.500){2}{\rule{0.400pt}{0.361pt}}
\put(655.67,510){\rule{0.400pt}{0.723pt}}
\multiput(655.17,511.50)(1.000,-1.500){2}{\rule{0.400pt}{0.361pt}}
\put(656.67,506){\rule{0.400pt}{0.964pt}}
\multiput(656.17,508.00)(1.000,-2.000){2}{\rule{0.400pt}{0.482pt}}
\put(657.67,503){\rule{0.400pt}{0.723pt}}
\multiput(657.17,504.50)(1.000,-1.500){2}{\rule{0.400pt}{0.361pt}}
\put(658.67,500){\rule{0.400pt}{0.723pt}}
\multiput(658.17,501.50)(1.000,-1.500){2}{\rule{0.400pt}{0.361pt}}
\put(659.67,497){\rule{0.400pt}{0.723pt}}
\multiput(659.17,498.50)(1.000,-1.500){2}{\rule{0.400pt}{0.361pt}}
\put(660.67,493){\rule{0.400pt}{0.964pt}}
\multiput(660.17,495.00)(1.000,-2.000){2}{\rule{0.400pt}{0.482pt}}
\put(661.67,490){\rule{0.400pt}{0.723pt}}
\multiput(661.17,491.50)(1.000,-1.500){2}{\rule{0.400pt}{0.361pt}}
\put(651.0,529.0){\rule[-0.200pt]{0.400pt}{0.723pt}}
\put(662.67,484){\rule{0.400pt}{0.723pt}}
\multiput(662.17,485.50)(1.000,-1.500){2}{\rule{0.400pt}{0.361pt}}
\put(663.67,480){\rule{0.400pt}{0.964pt}}
\multiput(663.17,482.00)(1.000,-2.000){2}{\rule{0.400pt}{0.482pt}}
\put(664.67,477){\rule{0.400pt}{0.723pt}}
\multiput(664.17,478.50)(1.000,-1.500){2}{\rule{0.400pt}{0.361pt}}
\put(665.67,474){\rule{0.400pt}{0.723pt}}
\multiput(665.17,475.50)(1.000,-1.500){2}{\rule{0.400pt}{0.361pt}}
\put(666.67,471){\rule{0.400pt}{0.723pt}}
\multiput(666.17,472.50)(1.000,-1.500){2}{\rule{0.400pt}{0.361pt}}
\put(667.67,467){\rule{0.400pt}{0.964pt}}
\multiput(667.17,469.00)(1.000,-2.000){2}{\rule{0.400pt}{0.482pt}}
\put(668.67,464){\rule{0.400pt}{0.723pt}}
\multiput(668.17,465.50)(1.000,-1.500){2}{\rule{0.400pt}{0.361pt}}
\put(669.67,461){\rule{0.400pt}{0.723pt}}
\multiput(669.17,462.50)(1.000,-1.500){2}{\rule{0.400pt}{0.361pt}}
\put(670.67,458){\rule{0.400pt}{0.723pt}}
\multiput(670.17,459.50)(1.000,-1.500){2}{\rule{0.400pt}{0.361pt}}
\put(671.67,454){\rule{0.400pt}{0.964pt}}
\multiput(671.17,456.00)(1.000,-2.000){2}{\rule{0.400pt}{0.482pt}}
\put(672.67,451){\rule{0.400pt}{0.723pt}}
\multiput(672.17,452.50)(1.000,-1.500){2}{\rule{0.400pt}{0.361pt}}
\put(663.0,487.0){\rule[-0.200pt]{0.400pt}{0.723pt}}
\put(673.67,445){\rule{0.400pt}{0.723pt}}
\multiput(673.17,446.50)(1.000,-1.500){2}{\rule{0.400pt}{0.361pt}}
\put(674.67,442){\rule{0.400pt}{0.723pt}}
\multiput(674.17,443.50)(1.000,-1.500){2}{\rule{0.400pt}{0.361pt}}
\put(675.67,438){\rule{0.400pt}{0.964pt}}
\multiput(675.17,440.00)(1.000,-2.000){2}{\rule{0.400pt}{0.482pt}}
\put(676.67,435){\rule{0.400pt}{0.723pt}}
\multiput(676.17,436.50)(1.000,-1.500){2}{\rule{0.400pt}{0.361pt}}
\put(677.67,432){\rule{0.400pt}{0.723pt}}
\multiput(677.17,433.50)(1.000,-1.500){2}{\rule{0.400pt}{0.361pt}}
\put(678.67,429){\rule{0.400pt}{0.723pt}}
\multiput(678.17,430.50)(1.000,-1.500){2}{\rule{0.400pt}{0.361pt}}
\put(679.67,426){\rule{0.400pt}{0.723pt}}
\multiput(679.17,427.50)(1.000,-1.500){2}{\rule{0.400pt}{0.361pt}}
\put(680.67,422){\rule{0.400pt}{0.964pt}}
\multiput(680.17,424.00)(1.000,-2.000){2}{\rule{0.400pt}{0.482pt}}
\put(681.67,419){\rule{0.400pt}{0.723pt}}
\multiput(681.17,420.50)(1.000,-1.500){2}{\rule{0.400pt}{0.361pt}}
\put(682.67,416){\rule{0.400pt}{0.723pt}}
\multiput(682.17,417.50)(1.000,-1.500){2}{\rule{0.400pt}{0.361pt}}
\put(683.67,413){\rule{0.400pt}{0.723pt}}
\multiput(683.17,414.50)(1.000,-1.500){2}{\rule{0.400pt}{0.361pt}}
\put(684.67,410){\rule{0.400pt}{0.723pt}}
\multiput(684.17,411.50)(1.000,-1.500){2}{\rule{0.400pt}{0.361pt}}
\put(674.0,448.0){\rule[-0.200pt]{0.400pt}{0.723pt}}
\put(685.67,403){\rule{0.400pt}{0.964pt}}
\multiput(685.17,405.00)(1.000,-2.000){2}{\rule{0.400pt}{0.482pt}}
\put(686.67,400){\rule{0.400pt}{0.723pt}}
\multiput(686.17,401.50)(1.000,-1.500){2}{\rule{0.400pt}{0.361pt}}
\put(687.67,397){\rule{0.400pt}{0.723pt}}
\multiput(687.17,398.50)(1.000,-1.500){2}{\rule{0.400pt}{0.361pt}}
\put(688.67,394){\rule{0.400pt}{0.723pt}}
\multiput(688.17,395.50)(1.000,-1.500){2}{\rule{0.400pt}{0.361pt}}
\put(689.67,391){\rule{0.400pt}{0.723pt}}
\multiput(689.17,392.50)(1.000,-1.500){2}{\rule{0.400pt}{0.361pt}}
\put(690.67,388){\rule{0.400pt}{0.723pt}}
\multiput(690.17,389.50)(1.000,-1.500){2}{\rule{0.400pt}{0.361pt}}
\put(691.67,385){\rule{0.400pt}{0.723pt}}
\multiput(691.17,386.50)(1.000,-1.500){2}{\rule{0.400pt}{0.361pt}}
\put(692.67,382){\rule{0.400pt}{0.723pt}}
\multiput(692.17,383.50)(1.000,-1.500){2}{\rule{0.400pt}{0.361pt}}
\put(693.67,379){\rule{0.400pt}{0.723pt}}
\multiput(693.17,380.50)(1.000,-1.500){2}{\rule{0.400pt}{0.361pt}}
\put(694.67,376){\rule{0.400pt}{0.723pt}}
\multiput(694.17,377.50)(1.000,-1.500){2}{\rule{0.400pt}{0.361pt}}
\put(695.67,373){\rule{0.400pt}{0.723pt}}
\multiput(695.17,374.50)(1.000,-1.500){2}{\rule{0.400pt}{0.361pt}}
\put(686.0,407.0){\rule[-0.200pt]{0.400pt}{0.723pt}}
\put(696.67,367){\rule{0.400pt}{0.723pt}}
\multiput(696.17,368.50)(1.000,-1.500){2}{\rule{0.400pt}{0.361pt}}
\put(697.67,364){\rule{0.400pt}{0.723pt}}
\multiput(697.17,365.50)(1.000,-1.500){2}{\rule{0.400pt}{0.361pt}}
\put(698.67,361){\rule{0.400pt}{0.723pt}}
\multiput(698.17,362.50)(1.000,-1.500){2}{\rule{0.400pt}{0.361pt}}
\put(699.67,358){\rule{0.400pt}{0.723pt}}
\multiput(699.17,359.50)(1.000,-1.500){2}{\rule{0.400pt}{0.361pt}}
\put(700.67,355){\rule{0.400pt}{0.723pt}}
\multiput(700.17,356.50)(1.000,-1.500){2}{\rule{0.400pt}{0.361pt}}
\put(701.67,352){\rule{0.400pt}{0.723pt}}
\multiput(701.17,353.50)(1.000,-1.500){2}{\rule{0.400pt}{0.361pt}}
\put(702.67,349){\rule{0.400pt}{0.723pt}}
\multiput(702.17,350.50)(1.000,-1.500){2}{\rule{0.400pt}{0.361pt}}
\put(703.67,346){\rule{0.400pt}{0.723pt}}
\multiput(703.17,347.50)(1.000,-1.500){2}{\rule{0.400pt}{0.361pt}}
\put(704.67,343){\rule{0.400pt}{0.723pt}}
\multiput(704.17,344.50)(1.000,-1.500){2}{\rule{0.400pt}{0.361pt}}
\put(705.67,340){\rule{0.400pt}{0.723pt}}
\multiput(705.17,341.50)(1.000,-1.500){2}{\rule{0.400pt}{0.361pt}}
\put(706.67,338){\rule{0.400pt}{0.482pt}}
\multiput(706.17,339.00)(1.000,-1.000){2}{\rule{0.400pt}{0.241pt}}
\put(707.67,335){\rule{0.400pt}{0.723pt}}
\multiput(707.17,336.50)(1.000,-1.500){2}{\rule{0.400pt}{0.361pt}}
\put(697.0,370.0){\rule[-0.200pt]{0.400pt}{0.723pt}}
\put(708.67,329){\rule{0.400pt}{0.723pt}}
\multiput(708.17,330.50)(1.000,-1.500){2}{\rule{0.400pt}{0.361pt}}
\put(709.67,327){\rule{0.400pt}{0.482pt}}
\multiput(709.17,328.00)(1.000,-1.000){2}{\rule{0.400pt}{0.241pt}}
\put(710.67,324){\rule{0.400pt}{0.723pt}}
\multiput(710.17,325.50)(1.000,-1.500){2}{\rule{0.400pt}{0.361pt}}
\put(711.67,321){\rule{0.400pt}{0.723pt}}
\multiput(711.17,322.50)(1.000,-1.500){2}{\rule{0.400pt}{0.361pt}}
\put(712.67,318){\rule{0.400pt}{0.723pt}}
\multiput(712.17,319.50)(1.000,-1.500){2}{\rule{0.400pt}{0.361pt}}
\put(713.67,316){\rule{0.400pt}{0.482pt}}
\multiput(713.17,317.00)(1.000,-1.000){2}{\rule{0.400pt}{0.241pt}}
\put(714.67,313){\rule{0.400pt}{0.723pt}}
\multiput(714.17,314.50)(1.000,-1.500){2}{\rule{0.400pt}{0.361pt}}
\put(715.67,310){\rule{0.400pt}{0.723pt}}
\multiput(715.17,311.50)(1.000,-1.500){2}{\rule{0.400pt}{0.361pt}}
\put(716.67,308){\rule{0.400pt}{0.482pt}}
\multiput(716.17,309.00)(1.000,-1.000){2}{\rule{0.400pt}{0.241pt}}
\put(717.67,305){\rule{0.400pt}{0.723pt}}
\multiput(717.17,306.50)(1.000,-1.500){2}{\rule{0.400pt}{0.361pt}}
\put(718.67,303){\rule{0.400pt}{0.482pt}}
\multiput(718.17,304.00)(1.000,-1.000){2}{\rule{0.400pt}{0.241pt}}
\put(709.0,332.0){\rule[-0.200pt]{0.400pt}{0.723pt}}
\put(719.67,297){\rule{0.400pt}{0.723pt}}
\multiput(719.17,298.50)(1.000,-1.500){2}{\rule{0.400pt}{0.361pt}}
\put(720.67,295){\rule{0.400pt}{0.482pt}}
\multiput(720.17,296.00)(1.000,-1.000){2}{\rule{0.400pt}{0.241pt}}
\put(721.67,292){\rule{0.400pt}{0.723pt}}
\multiput(721.17,293.50)(1.000,-1.500){2}{\rule{0.400pt}{0.361pt}}
\put(722.67,290){\rule{0.400pt}{0.482pt}}
\multiput(722.17,291.00)(1.000,-1.000){2}{\rule{0.400pt}{0.241pt}}
\put(723.67,288){\rule{0.400pt}{0.482pt}}
\multiput(723.17,289.00)(1.000,-1.000){2}{\rule{0.400pt}{0.241pt}}
\put(724.67,285){\rule{0.400pt}{0.723pt}}
\multiput(724.17,286.50)(1.000,-1.500){2}{\rule{0.400pt}{0.361pt}}
\put(725.67,283){\rule{0.400pt}{0.482pt}}
\multiput(725.17,284.00)(1.000,-1.000){2}{\rule{0.400pt}{0.241pt}}
\put(726.67,280){\rule{0.400pt}{0.723pt}}
\multiput(726.17,281.50)(1.000,-1.500){2}{\rule{0.400pt}{0.361pt}}
\put(727.67,278){\rule{0.400pt}{0.482pt}}
\multiput(727.17,279.00)(1.000,-1.000){2}{\rule{0.400pt}{0.241pt}}
\put(728.67,276){\rule{0.400pt}{0.482pt}}
\multiput(728.17,277.00)(1.000,-1.000){2}{\rule{0.400pt}{0.241pt}}
\put(729.67,273){\rule{0.400pt}{0.723pt}}
\multiput(729.17,274.50)(1.000,-1.500){2}{\rule{0.400pt}{0.361pt}}
\put(730.67,271){\rule{0.400pt}{0.482pt}}
\multiput(730.17,272.00)(1.000,-1.000){2}{\rule{0.400pt}{0.241pt}}
\put(720.0,300.0){\rule[-0.200pt]{0.400pt}{0.723pt}}
\put(731.67,266){\rule{0.400pt}{0.723pt}}
\multiput(731.17,267.50)(1.000,-1.500){2}{\rule{0.400pt}{0.361pt}}
\put(732.67,264){\rule{0.400pt}{0.482pt}}
\multiput(732.17,265.00)(1.000,-1.000){2}{\rule{0.400pt}{0.241pt}}
\put(733.67,262){\rule{0.400pt}{0.482pt}}
\multiput(733.17,263.00)(1.000,-1.000){2}{\rule{0.400pt}{0.241pt}}
\put(734.67,260){\rule{0.400pt}{0.482pt}}
\multiput(734.17,261.00)(1.000,-1.000){2}{\rule{0.400pt}{0.241pt}}
\put(735.67,258){\rule{0.400pt}{0.482pt}}
\multiput(735.17,259.00)(1.000,-1.000){2}{\rule{0.400pt}{0.241pt}}
\put(736.67,255){\rule{0.400pt}{0.723pt}}
\multiput(736.17,256.50)(1.000,-1.500){2}{\rule{0.400pt}{0.361pt}}
\put(737.67,253){\rule{0.400pt}{0.482pt}}
\multiput(737.17,254.00)(1.000,-1.000){2}{\rule{0.400pt}{0.241pt}}
\put(738.67,251){\rule{0.400pt}{0.482pt}}
\multiput(738.17,252.00)(1.000,-1.000){2}{\rule{0.400pt}{0.241pt}}
\put(739.67,249){\rule{0.400pt}{0.482pt}}
\multiput(739.17,250.00)(1.000,-1.000){2}{\rule{0.400pt}{0.241pt}}
\put(740.67,247){\rule{0.400pt}{0.482pt}}
\multiput(740.17,248.00)(1.000,-1.000){2}{\rule{0.400pt}{0.241pt}}
\put(741.67,245){\rule{0.400pt}{0.482pt}}
\multiput(741.17,246.00)(1.000,-1.000){2}{\rule{0.400pt}{0.241pt}}
\put(732.0,269.0){\rule[-0.200pt]{0.400pt}{0.482pt}}
\put(742.67,241){\rule{0.400pt}{0.482pt}}
\multiput(742.17,242.00)(1.000,-1.000){2}{\rule{0.400pt}{0.241pt}}
\put(743.67,239){\rule{0.400pt}{0.482pt}}
\multiput(743.17,240.00)(1.000,-1.000){2}{\rule{0.400pt}{0.241pt}}
\put(744.67,237){\rule{0.400pt}{0.482pt}}
\multiput(744.17,238.00)(1.000,-1.000){2}{\rule{0.400pt}{0.241pt}}
\put(745.67,235){\rule{0.400pt}{0.482pt}}
\multiput(745.17,236.00)(1.000,-1.000){2}{\rule{0.400pt}{0.241pt}}
\put(746.67,233){\rule{0.400pt}{0.482pt}}
\multiput(746.17,234.00)(1.000,-1.000){2}{\rule{0.400pt}{0.241pt}}
\put(748,231.67){\rule{0.241pt}{0.400pt}}
\multiput(748.00,232.17)(0.500,-1.000){2}{\rule{0.120pt}{0.400pt}}
\put(748.67,230){\rule{0.400pt}{0.482pt}}
\multiput(748.17,231.00)(1.000,-1.000){2}{\rule{0.400pt}{0.241pt}}
\put(749.67,228){\rule{0.400pt}{0.482pt}}
\multiput(749.17,229.00)(1.000,-1.000){2}{\rule{0.400pt}{0.241pt}}
\put(750.67,226){\rule{0.400pt}{0.482pt}}
\multiput(750.17,227.00)(1.000,-1.000){2}{\rule{0.400pt}{0.241pt}}
\put(751.67,224){\rule{0.400pt}{0.482pt}}
\multiput(751.17,225.00)(1.000,-1.000){2}{\rule{0.400pt}{0.241pt}}
\put(753,222.67){\rule{0.241pt}{0.400pt}}
\multiput(753.00,223.17)(0.500,-1.000){2}{\rule{0.120pt}{0.400pt}}
\put(753.67,221){\rule{0.400pt}{0.482pt}}
\multiput(753.17,222.00)(1.000,-1.000){2}{\rule{0.400pt}{0.241pt}}
\put(743.0,243.0){\rule[-0.200pt]{0.400pt}{0.482pt}}
\put(754.67,217){\rule{0.400pt}{0.482pt}}
\multiput(754.17,218.00)(1.000,-1.000){2}{\rule{0.400pt}{0.241pt}}
\put(756,215.67){\rule{0.241pt}{0.400pt}}
\multiput(756.00,216.17)(0.500,-1.000){2}{\rule{0.120pt}{0.400pt}}
\put(756.67,214){\rule{0.400pt}{0.482pt}}
\multiput(756.17,215.00)(1.000,-1.000){2}{\rule{0.400pt}{0.241pt}}
\put(758,212.67){\rule{0.241pt}{0.400pt}}
\multiput(758.00,213.17)(0.500,-1.000){2}{\rule{0.120pt}{0.400pt}}
\put(758.67,211){\rule{0.400pt}{0.482pt}}
\multiput(758.17,212.00)(1.000,-1.000){2}{\rule{0.400pt}{0.241pt}}
\put(759.67,209){\rule{0.400pt}{0.482pt}}
\multiput(759.17,210.00)(1.000,-1.000){2}{\rule{0.400pt}{0.241pt}}
\put(761,207.67){\rule{0.241pt}{0.400pt}}
\multiput(761.00,208.17)(0.500,-1.000){2}{\rule{0.120pt}{0.400pt}}
\put(761.67,206){\rule{0.400pt}{0.482pt}}
\multiput(761.17,207.00)(1.000,-1.000){2}{\rule{0.400pt}{0.241pt}}
\put(763,204.67){\rule{0.241pt}{0.400pt}}
\multiput(763.00,205.17)(0.500,-1.000){2}{\rule{0.120pt}{0.400pt}}
\put(763.67,203){\rule{0.400pt}{0.482pt}}
\multiput(763.17,204.00)(1.000,-1.000){2}{\rule{0.400pt}{0.241pt}}
\put(765,201.67){\rule{0.241pt}{0.400pt}}
\multiput(765.00,202.17)(0.500,-1.000){2}{\rule{0.120pt}{0.400pt}}
\put(755.0,219.0){\rule[-0.200pt]{0.400pt}{0.482pt}}
\put(766,198.67){\rule{0.241pt}{0.400pt}}
\multiput(766.00,199.17)(0.500,-1.000){2}{\rule{0.120pt}{0.400pt}}
\put(767,197.67){\rule{0.241pt}{0.400pt}}
\multiput(767.00,198.17)(0.500,-1.000){2}{\rule{0.120pt}{0.400pt}}
\put(767.67,196){\rule{0.400pt}{0.482pt}}
\multiput(767.17,197.00)(1.000,-1.000){2}{\rule{0.400pt}{0.241pt}}
\put(769,194.67){\rule{0.241pt}{0.400pt}}
\multiput(769.00,195.17)(0.500,-1.000){2}{\rule{0.120pt}{0.400pt}}
\put(770,193.67){\rule{0.241pt}{0.400pt}}
\multiput(770.00,194.17)(0.500,-1.000){2}{\rule{0.120pt}{0.400pt}}
\put(770.67,192){\rule{0.400pt}{0.482pt}}
\multiput(770.17,193.00)(1.000,-1.000){2}{\rule{0.400pt}{0.241pt}}
\put(772,190.67){\rule{0.241pt}{0.400pt}}
\multiput(772.00,191.17)(0.500,-1.000){2}{\rule{0.120pt}{0.400pt}}
\put(773,189.67){\rule{0.241pt}{0.400pt}}
\multiput(773.00,190.17)(0.500,-1.000){2}{\rule{0.120pt}{0.400pt}}
\put(773.67,188){\rule{0.400pt}{0.482pt}}
\multiput(773.17,189.00)(1.000,-1.000){2}{\rule{0.400pt}{0.241pt}}
\put(775,186.67){\rule{0.241pt}{0.400pt}}
\multiput(775.00,187.17)(0.500,-1.000){2}{\rule{0.120pt}{0.400pt}}
\put(776,185.67){\rule{0.241pt}{0.400pt}}
\multiput(776.00,186.17)(0.500,-1.000){2}{\rule{0.120pt}{0.400pt}}
\put(777,184.67){\rule{0.241pt}{0.400pt}}
\multiput(777.00,185.17)(0.500,-1.000){2}{\rule{0.120pt}{0.400pt}}
\put(766.0,200.0){\rule[-0.200pt]{0.400pt}{0.482pt}}
\put(777.67,182){\rule{0.400pt}{0.482pt}}
\multiput(777.17,183.00)(1.000,-1.000){2}{\rule{0.400pt}{0.241pt}}
\put(779,180.67){\rule{0.241pt}{0.400pt}}
\multiput(779.00,181.17)(0.500,-1.000){2}{\rule{0.120pt}{0.400pt}}
\put(780,179.67){\rule{0.241pt}{0.400pt}}
\multiput(780.00,180.17)(0.500,-1.000){2}{\rule{0.120pt}{0.400pt}}
\put(781,178.67){\rule{0.241pt}{0.400pt}}
\multiput(781.00,179.17)(0.500,-1.000){2}{\rule{0.120pt}{0.400pt}}
\put(782,177.67){\rule{0.241pt}{0.400pt}}
\multiput(782.00,178.17)(0.500,-1.000){2}{\rule{0.120pt}{0.400pt}}
\put(783,176.67){\rule{0.241pt}{0.400pt}}
\multiput(783.00,177.17)(0.500,-1.000){2}{\rule{0.120pt}{0.400pt}}
\put(784,175.67){\rule{0.241pt}{0.400pt}}
\multiput(784.00,176.17)(0.500,-1.000){2}{\rule{0.120pt}{0.400pt}}
\put(785,174.67){\rule{0.241pt}{0.400pt}}
\multiput(785.00,175.17)(0.500,-1.000){2}{\rule{0.120pt}{0.400pt}}
\put(786,173.67){\rule{0.241pt}{0.400pt}}
\multiput(786.00,174.17)(0.500,-1.000){2}{\rule{0.120pt}{0.400pt}}
\put(787,172.67){\rule{0.241pt}{0.400pt}}
\multiput(787.00,173.17)(0.500,-1.000){2}{\rule{0.120pt}{0.400pt}}
\put(788,171.67){\rule{0.241pt}{0.400pt}}
\multiput(788.00,172.17)(0.500,-1.000){2}{\rule{0.120pt}{0.400pt}}
\put(778.0,184.0){\usebox{\plotpoint}}
\put(789,169.67){\rule{0.241pt}{0.400pt}}
\multiput(789.00,170.17)(0.500,-1.000){2}{\rule{0.120pt}{0.400pt}}
\put(790,168.67){\rule{0.241pt}{0.400pt}}
\multiput(790.00,169.17)(0.500,-1.000){2}{\rule{0.120pt}{0.400pt}}
\put(791,167.67){\rule{0.241pt}{0.400pt}}
\multiput(791.00,168.17)(0.500,-1.000){2}{\rule{0.120pt}{0.400pt}}
\put(792,166.67){\rule{0.241pt}{0.400pt}}
\multiput(792.00,167.17)(0.500,-1.000){2}{\rule{0.120pt}{0.400pt}}
\put(789.0,171.0){\usebox{\plotpoint}}
\put(794,165.67){\rule{0.241pt}{0.400pt}}
\multiput(794.00,166.17)(0.500,-1.000){2}{\rule{0.120pt}{0.400pt}}
\put(793.0,167.0){\usebox{\plotpoint}}
\put(795,166){\usebox{\plotpoint}}
\put(795,165.67){\rule{0.241pt}{0.400pt}}
\multiput(795.00,165.17)(0.500,1.000){2}{\rule{0.120pt}{0.400pt}}
\put(797,166.67){\rule{0.241pt}{0.400pt}}
\multiput(797.00,166.17)(0.500,1.000){2}{\rule{0.120pt}{0.400pt}}
\put(798,167.67){\rule{0.241pt}{0.400pt}}
\multiput(798.00,167.17)(0.500,1.000){2}{\rule{0.120pt}{0.400pt}}
\put(799,168.67){\rule{0.241pt}{0.400pt}}
\multiput(799.00,168.17)(0.500,1.000){2}{\rule{0.120pt}{0.400pt}}
\put(800,169.67){\rule{0.241pt}{0.400pt}}
\multiput(800.00,169.17)(0.500,1.000){2}{\rule{0.120pt}{0.400pt}}
\put(796.0,167.0){\usebox{\plotpoint}}
\put(801,171.67){\rule{0.241pt}{0.400pt}}
\multiput(801.00,171.17)(0.500,1.000){2}{\rule{0.120pt}{0.400pt}}
\put(802,172.67){\rule{0.241pt}{0.400pt}}
\multiput(802.00,172.17)(0.500,1.000){2}{\rule{0.120pt}{0.400pt}}
\put(803,173.67){\rule{0.241pt}{0.400pt}}
\multiput(803.00,173.17)(0.500,1.000){2}{\rule{0.120pt}{0.400pt}}
\put(804,174.67){\rule{0.241pt}{0.400pt}}
\multiput(804.00,174.17)(0.500,1.000){2}{\rule{0.120pt}{0.400pt}}
\put(805,175.67){\rule{0.241pt}{0.400pt}}
\multiput(805.00,175.17)(0.500,1.000){2}{\rule{0.120pt}{0.400pt}}
\put(806,176.67){\rule{0.241pt}{0.400pt}}
\multiput(806.00,176.17)(0.500,1.000){2}{\rule{0.120pt}{0.400pt}}
\put(807,177.67){\rule{0.241pt}{0.400pt}}
\multiput(807.00,177.17)(0.500,1.000){2}{\rule{0.120pt}{0.400pt}}
\put(808,178.67){\rule{0.241pt}{0.400pt}}
\multiput(808.00,178.17)(0.500,1.000){2}{\rule{0.120pt}{0.400pt}}
\put(809,179.67){\rule{0.241pt}{0.400pt}}
\multiput(809.00,179.17)(0.500,1.000){2}{\rule{0.120pt}{0.400pt}}
\put(810,180.67){\rule{0.241pt}{0.400pt}}
\multiput(810.00,180.17)(0.500,1.000){2}{\rule{0.120pt}{0.400pt}}
\put(810.67,182){\rule{0.400pt}{0.482pt}}
\multiput(810.17,182.00)(1.000,1.000){2}{\rule{0.400pt}{0.241pt}}
\put(801.0,171.0){\usebox{\plotpoint}}
\put(812,184.67){\rule{0.241pt}{0.400pt}}
\multiput(812.00,184.17)(0.500,1.000){2}{\rule{0.120pt}{0.400pt}}
\put(813,185.67){\rule{0.241pt}{0.400pt}}
\multiput(813.00,185.17)(0.500,1.000){2}{\rule{0.120pt}{0.400pt}}
\put(814,186.67){\rule{0.241pt}{0.400pt}}
\multiput(814.00,186.17)(0.500,1.000){2}{\rule{0.120pt}{0.400pt}}
\put(814.67,188){\rule{0.400pt}{0.482pt}}
\multiput(814.17,188.00)(1.000,1.000){2}{\rule{0.400pt}{0.241pt}}
\put(816,189.67){\rule{0.241pt}{0.400pt}}
\multiput(816.00,189.17)(0.500,1.000){2}{\rule{0.120pt}{0.400pt}}
\put(817,190.67){\rule{0.241pt}{0.400pt}}
\multiput(817.00,190.17)(0.500,1.000){2}{\rule{0.120pt}{0.400pt}}
\put(817.67,192){\rule{0.400pt}{0.482pt}}
\multiput(817.17,192.00)(1.000,1.000){2}{\rule{0.400pt}{0.241pt}}
\put(819,193.67){\rule{0.241pt}{0.400pt}}
\multiput(819.00,193.17)(0.500,1.000){2}{\rule{0.120pt}{0.400pt}}
\put(820,194.67){\rule{0.241pt}{0.400pt}}
\multiput(820.00,194.17)(0.500,1.000){2}{\rule{0.120pt}{0.400pt}}
\put(820.67,196){\rule{0.400pt}{0.482pt}}
\multiput(820.17,196.00)(1.000,1.000){2}{\rule{0.400pt}{0.241pt}}
\put(822,197.67){\rule{0.241pt}{0.400pt}}
\multiput(822.00,197.17)(0.500,1.000){2}{\rule{0.120pt}{0.400pt}}
\put(823,198.67){\rule{0.241pt}{0.400pt}}
\multiput(823.00,198.17)(0.500,1.000){2}{\rule{0.120pt}{0.400pt}}
\put(812.0,184.0){\usebox{\plotpoint}}
\put(824,201.67){\rule{0.241pt}{0.400pt}}
\multiput(824.00,201.17)(0.500,1.000){2}{\rule{0.120pt}{0.400pt}}
\put(824.67,203){\rule{0.400pt}{0.482pt}}
\multiput(824.17,203.00)(1.000,1.000){2}{\rule{0.400pt}{0.241pt}}
\put(826,204.67){\rule{0.241pt}{0.400pt}}
\multiput(826.00,204.17)(0.500,1.000){2}{\rule{0.120pt}{0.400pt}}
\put(826.67,206){\rule{0.400pt}{0.482pt}}
\multiput(826.17,206.00)(1.000,1.000){2}{\rule{0.400pt}{0.241pt}}
\put(828,207.67){\rule{0.241pt}{0.400pt}}
\multiput(828.00,207.17)(0.500,1.000){2}{\rule{0.120pt}{0.400pt}}
\put(828.67,209){\rule{0.400pt}{0.482pt}}
\multiput(828.17,209.00)(1.000,1.000){2}{\rule{0.400pt}{0.241pt}}
\put(829.67,211){\rule{0.400pt}{0.482pt}}
\multiput(829.17,211.00)(1.000,1.000){2}{\rule{0.400pt}{0.241pt}}
\put(831,212.67){\rule{0.241pt}{0.400pt}}
\multiput(831.00,212.17)(0.500,1.000){2}{\rule{0.120pt}{0.400pt}}
\put(831.67,214){\rule{0.400pt}{0.482pt}}
\multiput(831.17,214.00)(1.000,1.000){2}{\rule{0.400pt}{0.241pt}}
\put(833,215.67){\rule{0.241pt}{0.400pt}}
\multiput(833.00,215.17)(0.500,1.000){2}{\rule{0.120pt}{0.400pt}}
\put(833.67,217){\rule{0.400pt}{0.482pt}}
\multiput(833.17,217.00)(1.000,1.000){2}{\rule{0.400pt}{0.241pt}}
\put(824.0,200.0){\rule[-0.200pt]{0.400pt}{0.482pt}}
\put(834.67,221){\rule{0.400pt}{0.482pt}}
\multiput(834.17,221.00)(1.000,1.000){2}{\rule{0.400pt}{0.241pt}}
\put(836,222.67){\rule{0.241pt}{0.400pt}}
\multiput(836.00,222.17)(0.500,1.000){2}{\rule{0.120pt}{0.400pt}}
\put(836.67,224){\rule{0.400pt}{0.482pt}}
\multiput(836.17,224.00)(1.000,1.000){2}{\rule{0.400pt}{0.241pt}}
\put(837.67,226){\rule{0.400pt}{0.482pt}}
\multiput(837.17,226.00)(1.000,1.000){2}{\rule{0.400pt}{0.241pt}}
\put(838.67,228){\rule{0.400pt}{0.482pt}}
\multiput(838.17,228.00)(1.000,1.000){2}{\rule{0.400pt}{0.241pt}}
\put(839.67,230){\rule{0.400pt}{0.482pt}}
\multiput(839.17,230.00)(1.000,1.000){2}{\rule{0.400pt}{0.241pt}}
\put(841,231.67){\rule{0.241pt}{0.400pt}}
\multiput(841.00,231.17)(0.500,1.000){2}{\rule{0.120pt}{0.400pt}}
\put(841.67,233){\rule{0.400pt}{0.482pt}}
\multiput(841.17,233.00)(1.000,1.000){2}{\rule{0.400pt}{0.241pt}}
\put(842.67,235){\rule{0.400pt}{0.482pt}}
\multiput(842.17,235.00)(1.000,1.000){2}{\rule{0.400pt}{0.241pt}}
\put(843.67,237){\rule{0.400pt}{0.482pt}}
\multiput(843.17,237.00)(1.000,1.000){2}{\rule{0.400pt}{0.241pt}}
\put(844.67,239){\rule{0.400pt}{0.482pt}}
\multiput(844.17,239.00)(1.000,1.000){2}{\rule{0.400pt}{0.241pt}}
\put(845.67,241){\rule{0.400pt}{0.482pt}}
\multiput(845.17,241.00)(1.000,1.000){2}{\rule{0.400pt}{0.241pt}}
\put(835.0,219.0){\rule[-0.200pt]{0.400pt}{0.482pt}}
\put(846.67,245){\rule{0.400pt}{0.482pt}}
\multiput(846.17,245.00)(1.000,1.000){2}{\rule{0.400pt}{0.241pt}}
\put(847.67,247){\rule{0.400pt}{0.482pt}}
\multiput(847.17,247.00)(1.000,1.000){2}{\rule{0.400pt}{0.241pt}}
\put(848.67,249){\rule{0.400pt}{0.482pt}}
\multiput(848.17,249.00)(1.000,1.000){2}{\rule{0.400pt}{0.241pt}}
\put(849.67,251){\rule{0.400pt}{0.482pt}}
\multiput(849.17,251.00)(1.000,1.000){2}{\rule{0.400pt}{0.241pt}}
\put(850.67,253){\rule{0.400pt}{0.482pt}}
\multiput(850.17,253.00)(1.000,1.000){2}{\rule{0.400pt}{0.241pt}}
\put(851.67,255){\rule{0.400pt}{0.723pt}}
\multiput(851.17,255.00)(1.000,1.500){2}{\rule{0.400pt}{0.361pt}}
\put(852.67,258){\rule{0.400pt}{0.482pt}}
\multiput(852.17,258.00)(1.000,1.000){2}{\rule{0.400pt}{0.241pt}}
\put(853.67,260){\rule{0.400pt}{0.482pt}}
\multiput(853.17,260.00)(1.000,1.000){2}{\rule{0.400pt}{0.241pt}}
\put(854.67,262){\rule{0.400pt}{0.482pt}}
\multiput(854.17,262.00)(1.000,1.000){2}{\rule{0.400pt}{0.241pt}}
\put(855.67,264){\rule{0.400pt}{0.482pt}}
\multiput(855.17,264.00)(1.000,1.000){2}{\rule{0.400pt}{0.241pt}}
\put(856.67,266){\rule{0.400pt}{0.723pt}}
\multiput(856.17,266.00)(1.000,1.500){2}{\rule{0.400pt}{0.361pt}}
\put(847.0,243.0){\rule[-0.200pt]{0.400pt}{0.482pt}}
\put(857.67,271){\rule{0.400pt}{0.482pt}}
\multiput(857.17,271.00)(1.000,1.000){2}{\rule{0.400pt}{0.241pt}}
\put(858.67,273){\rule{0.400pt}{0.723pt}}
\multiput(858.17,273.00)(1.000,1.500){2}{\rule{0.400pt}{0.361pt}}
\put(859.67,276){\rule{0.400pt}{0.482pt}}
\multiput(859.17,276.00)(1.000,1.000){2}{\rule{0.400pt}{0.241pt}}
\put(860.67,278){\rule{0.400pt}{0.482pt}}
\multiput(860.17,278.00)(1.000,1.000){2}{\rule{0.400pt}{0.241pt}}
\put(861.67,280){\rule{0.400pt}{0.723pt}}
\multiput(861.17,280.00)(1.000,1.500){2}{\rule{0.400pt}{0.361pt}}
\put(862.67,283){\rule{0.400pt}{0.482pt}}
\multiput(862.17,283.00)(1.000,1.000){2}{\rule{0.400pt}{0.241pt}}
\put(863.67,285){\rule{0.400pt}{0.723pt}}
\multiput(863.17,285.00)(1.000,1.500){2}{\rule{0.400pt}{0.361pt}}
\put(864.67,288){\rule{0.400pt}{0.482pt}}
\multiput(864.17,288.00)(1.000,1.000){2}{\rule{0.400pt}{0.241pt}}
\put(865.67,290){\rule{0.400pt}{0.482pt}}
\multiput(865.17,290.00)(1.000,1.000){2}{\rule{0.400pt}{0.241pt}}
\put(866.67,292){\rule{0.400pt}{0.723pt}}
\multiput(866.17,292.00)(1.000,1.500){2}{\rule{0.400pt}{0.361pt}}
\put(867.67,295){\rule{0.400pt}{0.482pt}}
\multiput(867.17,295.00)(1.000,1.000){2}{\rule{0.400pt}{0.241pt}}
\put(868.67,297){\rule{0.400pt}{0.723pt}}
\multiput(868.17,297.00)(1.000,1.500){2}{\rule{0.400pt}{0.361pt}}
\put(858.0,269.0){\rule[-0.200pt]{0.400pt}{0.482pt}}
\put(869.67,303){\rule{0.400pt}{0.482pt}}
\multiput(869.17,303.00)(1.000,1.000){2}{\rule{0.400pt}{0.241pt}}
\put(870.67,305){\rule{0.400pt}{0.723pt}}
\multiput(870.17,305.00)(1.000,1.500){2}{\rule{0.400pt}{0.361pt}}
\put(871.67,308){\rule{0.400pt}{0.482pt}}
\multiput(871.17,308.00)(1.000,1.000){2}{\rule{0.400pt}{0.241pt}}
\put(872.67,310){\rule{0.400pt}{0.723pt}}
\multiput(872.17,310.00)(1.000,1.500){2}{\rule{0.400pt}{0.361pt}}
\put(873.67,313){\rule{0.400pt}{0.723pt}}
\multiput(873.17,313.00)(1.000,1.500){2}{\rule{0.400pt}{0.361pt}}
\put(874.67,316){\rule{0.400pt}{0.482pt}}
\multiput(874.17,316.00)(1.000,1.000){2}{\rule{0.400pt}{0.241pt}}
\put(875.67,318){\rule{0.400pt}{0.723pt}}
\multiput(875.17,318.00)(1.000,1.500){2}{\rule{0.400pt}{0.361pt}}
\put(876.67,321){\rule{0.400pt}{0.723pt}}
\multiput(876.17,321.00)(1.000,1.500){2}{\rule{0.400pt}{0.361pt}}
\put(877.67,324){\rule{0.400pt}{0.723pt}}
\multiput(877.17,324.00)(1.000,1.500){2}{\rule{0.400pt}{0.361pt}}
\put(878.67,327){\rule{0.400pt}{0.482pt}}
\multiput(878.17,327.00)(1.000,1.000){2}{\rule{0.400pt}{0.241pt}}
\put(879.67,329){\rule{0.400pt}{0.723pt}}
\multiput(879.17,329.00)(1.000,1.500){2}{\rule{0.400pt}{0.361pt}}
\put(870.0,300.0){\rule[-0.200pt]{0.400pt}{0.723pt}}
\put(880.67,335){\rule{0.400pt}{0.723pt}}
\multiput(880.17,335.00)(1.000,1.500){2}{\rule{0.400pt}{0.361pt}}
\put(881.67,338){\rule{0.400pt}{0.482pt}}
\multiput(881.17,338.00)(1.000,1.000){2}{\rule{0.400pt}{0.241pt}}
\put(882.67,340){\rule{0.400pt}{0.723pt}}
\multiput(882.17,340.00)(1.000,1.500){2}{\rule{0.400pt}{0.361pt}}
\put(883.67,343){\rule{0.400pt}{0.723pt}}
\multiput(883.17,343.00)(1.000,1.500){2}{\rule{0.400pt}{0.361pt}}
\put(884.67,346){\rule{0.400pt}{0.723pt}}
\multiput(884.17,346.00)(1.000,1.500){2}{\rule{0.400pt}{0.361pt}}
\put(885.67,349){\rule{0.400pt}{0.723pt}}
\multiput(885.17,349.00)(1.000,1.500){2}{\rule{0.400pt}{0.361pt}}
\put(886.67,352){\rule{0.400pt}{0.723pt}}
\multiput(886.17,352.00)(1.000,1.500){2}{\rule{0.400pt}{0.361pt}}
\put(887.67,355){\rule{0.400pt}{0.723pt}}
\multiput(887.17,355.00)(1.000,1.500){2}{\rule{0.400pt}{0.361pt}}
\put(888.67,358){\rule{0.400pt}{0.723pt}}
\multiput(888.17,358.00)(1.000,1.500){2}{\rule{0.400pt}{0.361pt}}
\put(889.67,361){\rule{0.400pt}{0.723pt}}
\multiput(889.17,361.00)(1.000,1.500){2}{\rule{0.400pt}{0.361pt}}
\put(890.67,364){\rule{0.400pt}{0.723pt}}
\multiput(890.17,364.00)(1.000,1.500){2}{\rule{0.400pt}{0.361pt}}
\put(891.67,367){\rule{0.400pt}{0.723pt}}
\multiput(891.17,367.00)(1.000,1.500){2}{\rule{0.400pt}{0.361pt}}
\put(881.0,332.0){\rule[-0.200pt]{0.400pt}{0.723pt}}
\put(892.67,373){\rule{0.400pt}{0.723pt}}
\multiput(892.17,373.00)(1.000,1.500){2}{\rule{0.400pt}{0.361pt}}
\put(893.67,376){\rule{0.400pt}{0.723pt}}
\multiput(893.17,376.00)(1.000,1.500){2}{\rule{0.400pt}{0.361pt}}
\put(894.67,379){\rule{0.400pt}{0.723pt}}
\multiput(894.17,379.00)(1.000,1.500){2}{\rule{0.400pt}{0.361pt}}
\put(895.67,382){\rule{0.400pt}{0.723pt}}
\multiput(895.17,382.00)(1.000,1.500){2}{\rule{0.400pt}{0.361pt}}
\put(896.67,385){\rule{0.400pt}{0.723pt}}
\multiput(896.17,385.00)(1.000,1.500){2}{\rule{0.400pt}{0.361pt}}
\put(897.67,388){\rule{0.400pt}{0.723pt}}
\multiput(897.17,388.00)(1.000,1.500){2}{\rule{0.400pt}{0.361pt}}
\put(898.67,391){\rule{0.400pt}{0.723pt}}
\multiput(898.17,391.00)(1.000,1.500){2}{\rule{0.400pt}{0.361pt}}
\put(899.67,394){\rule{0.400pt}{0.723pt}}
\multiput(899.17,394.00)(1.000,1.500){2}{\rule{0.400pt}{0.361pt}}
\put(900.67,397){\rule{0.400pt}{0.723pt}}
\multiput(900.17,397.00)(1.000,1.500){2}{\rule{0.400pt}{0.361pt}}
\put(901.67,400){\rule{0.400pt}{0.723pt}}
\multiput(901.17,400.00)(1.000,1.500){2}{\rule{0.400pt}{0.361pt}}
\put(902.67,403){\rule{0.400pt}{0.964pt}}
\multiput(902.17,403.00)(1.000,2.000){2}{\rule{0.400pt}{0.482pt}}
\put(893.0,370.0){\rule[-0.200pt]{0.400pt}{0.723pt}}
\put(903.67,410){\rule{0.400pt}{0.723pt}}
\multiput(903.17,410.00)(1.000,1.500){2}{\rule{0.400pt}{0.361pt}}
\put(904.67,413){\rule{0.400pt}{0.723pt}}
\multiput(904.17,413.00)(1.000,1.500){2}{\rule{0.400pt}{0.361pt}}
\put(905.67,416){\rule{0.400pt}{0.723pt}}
\multiput(905.17,416.00)(1.000,1.500){2}{\rule{0.400pt}{0.361pt}}
\put(906.67,419){\rule{0.400pt}{0.723pt}}
\multiput(906.17,419.00)(1.000,1.500){2}{\rule{0.400pt}{0.361pt}}
\put(907.67,422){\rule{0.400pt}{0.964pt}}
\multiput(907.17,422.00)(1.000,2.000){2}{\rule{0.400pt}{0.482pt}}
\put(908.67,426){\rule{0.400pt}{0.723pt}}
\multiput(908.17,426.00)(1.000,1.500){2}{\rule{0.400pt}{0.361pt}}
\put(909.67,429){\rule{0.400pt}{0.723pt}}
\multiput(909.17,429.00)(1.000,1.500){2}{\rule{0.400pt}{0.361pt}}
\put(910.67,432){\rule{0.400pt}{0.723pt}}
\multiput(910.17,432.00)(1.000,1.500){2}{\rule{0.400pt}{0.361pt}}
\put(911.67,435){\rule{0.400pt}{0.723pt}}
\multiput(911.17,435.00)(1.000,1.500){2}{\rule{0.400pt}{0.361pt}}
\put(912.67,438){\rule{0.400pt}{0.964pt}}
\multiput(912.17,438.00)(1.000,2.000){2}{\rule{0.400pt}{0.482pt}}
\put(913.67,442){\rule{0.400pt}{0.723pt}}
\multiput(913.17,442.00)(1.000,1.500){2}{\rule{0.400pt}{0.361pt}}
\put(914.67,445){\rule{0.400pt}{0.723pt}}
\multiput(914.17,445.00)(1.000,1.500){2}{\rule{0.400pt}{0.361pt}}
\put(904.0,407.0){\rule[-0.200pt]{0.400pt}{0.723pt}}
\put(915.67,451){\rule{0.400pt}{0.723pt}}
\multiput(915.17,451.00)(1.000,1.500){2}{\rule{0.400pt}{0.361pt}}
\put(916.67,454){\rule{0.400pt}{0.964pt}}
\multiput(916.17,454.00)(1.000,2.000){2}{\rule{0.400pt}{0.482pt}}
\put(917.67,458){\rule{0.400pt}{0.723pt}}
\multiput(917.17,458.00)(1.000,1.500){2}{\rule{0.400pt}{0.361pt}}
\put(918.67,461){\rule{0.400pt}{0.723pt}}
\multiput(918.17,461.00)(1.000,1.500){2}{\rule{0.400pt}{0.361pt}}
\put(919.67,464){\rule{0.400pt}{0.723pt}}
\multiput(919.17,464.00)(1.000,1.500){2}{\rule{0.400pt}{0.361pt}}
\put(920.67,467){\rule{0.400pt}{0.964pt}}
\multiput(920.17,467.00)(1.000,2.000){2}{\rule{0.400pt}{0.482pt}}
\put(921.67,471){\rule{0.400pt}{0.723pt}}
\multiput(921.17,471.00)(1.000,1.500){2}{\rule{0.400pt}{0.361pt}}
\put(922.67,474){\rule{0.400pt}{0.723pt}}
\multiput(922.17,474.00)(1.000,1.500){2}{\rule{0.400pt}{0.361pt}}
\put(923.67,477){\rule{0.400pt}{0.723pt}}
\multiput(923.17,477.00)(1.000,1.500){2}{\rule{0.400pt}{0.361pt}}
\put(924.67,480){\rule{0.400pt}{0.964pt}}
\multiput(924.17,480.00)(1.000,2.000){2}{\rule{0.400pt}{0.482pt}}
\put(925.67,484){\rule{0.400pt}{0.723pt}}
\multiput(925.17,484.00)(1.000,1.500){2}{\rule{0.400pt}{0.361pt}}
\put(916.0,448.0){\rule[-0.200pt]{0.400pt}{0.723pt}}
\put(926.67,490){\rule{0.400pt}{0.723pt}}
\multiput(926.17,490.00)(1.000,1.500){2}{\rule{0.400pt}{0.361pt}}
\put(927.67,493){\rule{0.400pt}{0.964pt}}
\multiput(927.17,493.00)(1.000,2.000){2}{\rule{0.400pt}{0.482pt}}
\put(928.67,497){\rule{0.400pt}{0.723pt}}
\multiput(928.17,497.00)(1.000,1.500){2}{\rule{0.400pt}{0.361pt}}
\put(929.67,500){\rule{0.400pt}{0.723pt}}
\multiput(929.17,500.00)(1.000,1.500){2}{\rule{0.400pt}{0.361pt}}
\put(930.67,503){\rule{0.400pt}{0.723pt}}
\multiput(930.17,503.00)(1.000,1.500){2}{\rule{0.400pt}{0.361pt}}
\put(931.67,506){\rule{0.400pt}{0.964pt}}
\multiput(931.17,506.00)(1.000,2.000){2}{\rule{0.400pt}{0.482pt}}
\put(932.67,510){\rule{0.400pt}{0.723pt}}
\multiput(932.17,510.00)(1.000,1.500){2}{\rule{0.400pt}{0.361pt}}
\put(933.67,513){\rule{0.400pt}{0.723pt}}
\multiput(933.17,513.00)(1.000,1.500){2}{\rule{0.400pt}{0.361pt}}
\put(934.67,516){\rule{0.400pt}{0.723pt}}
\multiput(934.17,516.00)(1.000,1.500){2}{\rule{0.400pt}{0.361pt}}
\put(935.67,519){\rule{0.400pt}{0.723pt}}
\multiput(935.17,519.00)(1.000,1.500){2}{\rule{0.400pt}{0.361pt}}
\put(936.67,522){\rule{0.400pt}{0.964pt}}
\multiput(936.17,522.00)(1.000,2.000){2}{\rule{0.400pt}{0.482pt}}
\put(937.67,526){\rule{0.400pt}{0.723pt}}
\multiput(937.17,526.00)(1.000,1.500){2}{\rule{0.400pt}{0.361pt}}
\put(927.0,487.0){\rule[-0.200pt]{0.400pt}{0.723pt}}
\put(938.67,532){\rule{0.400pt}{0.723pt}}
\multiput(938.17,532.00)(1.000,1.500){2}{\rule{0.400pt}{0.361pt}}
\put(939.67,535){\rule{0.400pt}{0.723pt}}
\multiput(939.17,535.00)(1.000,1.500){2}{\rule{0.400pt}{0.361pt}}
\put(940.67,538){\rule{0.400pt}{0.723pt}}
\multiput(940.17,538.00)(1.000,1.500){2}{\rule{0.400pt}{0.361pt}}
\put(941.67,541){\rule{0.400pt}{0.964pt}}
\multiput(941.17,541.00)(1.000,2.000){2}{\rule{0.400pt}{0.482pt}}
\put(942.67,545){\rule{0.400pt}{0.723pt}}
\multiput(942.17,545.00)(1.000,1.500){2}{\rule{0.400pt}{0.361pt}}
\put(943.67,548){\rule{0.400pt}{0.723pt}}
\multiput(943.17,548.00)(1.000,1.500){2}{\rule{0.400pt}{0.361pt}}
\put(944.67,551){\rule{0.400pt}{0.723pt}}
\multiput(944.17,551.00)(1.000,1.500){2}{\rule{0.400pt}{0.361pt}}
\put(945.67,554){\rule{0.400pt}{0.723pt}}
\multiput(945.17,554.00)(1.000,1.500){2}{\rule{0.400pt}{0.361pt}}
\put(946.67,557){\rule{0.400pt}{0.723pt}}
\multiput(946.17,557.00)(1.000,1.500){2}{\rule{0.400pt}{0.361pt}}
\put(947.67,560){\rule{0.400pt}{0.723pt}}
\multiput(947.17,560.00)(1.000,1.500){2}{\rule{0.400pt}{0.361pt}}
\put(948.67,563){\rule{0.400pt}{0.723pt}}
\multiput(948.17,563.00)(1.000,1.500){2}{\rule{0.400pt}{0.361pt}}
\put(939.0,529.0){\rule[-0.200pt]{0.400pt}{0.723pt}}
\put(949.67,570){\rule{0.400pt}{0.723pt}}
\multiput(949.17,570.00)(1.000,1.500){2}{\rule{0.400pt}{0.361pt}}
\put(950.67,573){\rule{0.400pt}{0.723pt}}
\multiput(950.17,573.00)(1.000,1.500){2}{\rule{0.400pt}{0.361pt}}
\put(951.67,576){\rule{0.400pt}{0.723pt}}
\multiput(951.17,576.00)(1.000,1.500){2}{\rule{0.400pt}{0.361pt}}
\put(952.67,579){\rule{0.400pt}{0.723pt}}
\multiput(952.17,579.00)(1.000,1.500){2}{\rule{0.400pt}{0.361pt}}
\put(953.67,582){\rule{0.400pt}{0.723pt}}
\multiput(953.17,582.00)(1.000,1.500){2}{\rule{0.400pt}{0.361pt}}
\put(954.67,585){\rule{0.400pt}{0.723pt}}
\multiput(954.17,585.00)(1.000,1.500){2}{\rule{0.400pt}{0.361pt}}
\put(955.67,588){\rule{0.400pt}{0.723pt}}
\multiput(955.17,588.00)(1.000,1.500){2}{\rule{0.400pt}{0.361pt}}
\put(956.67,591){\rule{0.400pt}{0.723pt}}
\multiput(956.17,591.00)(1.000,1.500){2}{\rule{0.400pt}{0.361pt}}
\put(957.67,594){\rule{0.400pt}{0.723pt}}
\multiput(957.17,594.00)(1.000,1.500){2}{\rule{0.400pt}{0.361pt}}
\put(958.67,597){\rule{0.400pt}{0.482pt}}
\multiput(958.17,597.00)(1.000,1.000){2}{\rule{0.400pt}{0.241pt}}
\put(959.67,599){\rule{0.400pt}{0.723pt}}
\multiput(959.17,599.00)(1.000,1.500){2}{\rule{0.400pt}{0.361pt}}
\put(960.67,602){\rule{0.400pt}{0.723pt}}
\multiput(960.17,602.00)(1.000,1.500){2}{\rule{0.400pt}{0.361pt}}
\put(950.0,566.0){\rule[-0.200pt]{0.400pt}{0.964pt}}
\put(961.67,608){\rule{0.400pt}{0.723pt}}
\multiput(961.17,608.00)(1.000,1.500){2}{\rule{0.400pt}{0.361pt}}
\put(962.67,611){\rule{0.400pt}{0.723pt}}
\multiput(962.17,611.00)(1.000,1.500){2}{\rule{0.400pt}{0.361pt}}
\put(963.67,614){\rule{0.400pt}{0.723pt}}
\multiput(963.17,614.00)(1.000,1.500){2}{\rule{0.400pt}{0.361pt}}
\put(964.67,617){\rule{0.400pt}{0.723pt}}
\multiput(964.17,617.00)(1.000,1.500){2}{\rule{0.400pt}{0.361pt}}
\put(965.67,620){\rule{0.400pt}{0.482pt}}
\multiput(965.17,620.00)(1.000,1.000){2}{\rule{0.400pt}{0.241pt}}
\put(966.67,622){\rule{0.400pt}{0.723pt}}
\multiput(966.17,622.00)(1.000,1.500){2}{\rule{0.400pt}{0.361pt}}
\put(967.67,625){\rule{0.400pt}{0.723pt}}
\multiput(967.17,625.00)(1.000,1.500){2}{\rule{0.400pt}{0.361pt}}
\put(968.67,628){\rule{0.400pt}{0.723pt}}
\multiput(968.17,628.00)(1.000,1.500){2}{\rule{0.400pt}{0.361pt}}
\put(969.67,631){\rule{0.400pt}{0.482pt}}
\multiput(969.17,631.00)(1.000,1.000){2}{\rule{0.400pt}{0.241pt}}
\put(970.67,633){\rule{0.400pt}{0.723pt}}
\multiput(970.17,633.00)(1.000,1.500){2}{\rule{0.400pt}{0.361pt}}
\put(971.67,636){\rule{0.400pt}{0.723pt}}
\multiput(971.17,636.00)(1.000,1.500){2}{\rule{0.400pt}{0.361pt}}
\put(962.0,605.0){\rule[-0.200pt]{0.400pt}{0.723pt}}
\put(972.67,641){\rule{0.400pt}{0.723pt}}
\multiput(972.17,641.00)(1.000,1.500){2}{\rule{0.400pt}{0.361pt}}
\put(973.67,644){\rule{0.400pt}{0.723pt}}
\multiput(973.17,644.00)(1.000,1.500){2}{\rule{0.400pt}{0.361pt}}
\put(974.67,647){\rule{0.400pt}{0.482pt}}
\multiput(974.17,647.00)(1.000,1.000){2}{\rule{0.400pt}{0.241pt}}
\put(975.67,649){\rule{0.400pt}{0.723pt}}
\multiput(975.17,649.00)(1.000,1.500){2}{\rule{0.400pt}{0.361pt}}
\put(976.67,652){\rule{0.400pt}{0.482pt}}
\multiput(976.17,652.00)(1.000,1.000){2}{\rule{0.400pt}{0.241pt}}
\put(977.67,654){\rule{0.400pt}{0.723pt}}
\multiput(977.17,654.00)(1.000,1.500){2}{\rule{0.400pt}{0.361pt}}
\put(978.67,657){\rule{0.400pt}{0.723pt}}
\multiput(978.17,657.00)(1.000,1.500){2}{\rule{0.400pt}{0.361pt}}
\put(979.67,660){\rule{0.400pt}{0.482pt}}
\multiput(979.17,660.00)(1.000,1.000){2}{\rule{0.400pt}{0.241pt}}
\put(980.67,662){\rule{0.400pt}{0.723pt}}
\multiput(980.17,662.00)(1.000,1.500){2}{\rule{0.400pt}{0.361pt}}
\put(981.67,665){\rule{0.400pt}{0.482pt}}
\multiput(981.17,665.00)(1.000,1.000){2}{\rule{0.400pt}{0.241pt}}
\put(982.67,667){\rule{0.400pt}{0.482pt}}
\multiput(982.17,667.00)(1.000,1.000){2}{\rule{0.400pt}{0.241pt}}
\put(983.67,669){\rule{0.400pt}{0.723pt}}
\multiput(983.17,669.00)(1.000,1.500){2}{\rule{0.400pt}{0.361pt}}
\put(973.0,639.0){\rule[-0.200pt]{0.400pt}{0.482pt}}
\put(984.67,674){\rule{0.400pt}{0.723pt}}
\multiput(984.17,674.00)(1.000,1.500){2}{\rule{0.400pt}{0.361pt}}
\put(985.67,677){\rule{0.400pt}{0.482pt}}
\multiput(985.17,677.00)(1.000,1.000){2}{\rule{0.400pt}{0.241pt}}
\put(986.67,679){\rule{0.400pt}{0.482pt}}
\multiput(986.17,679.00)(1.000,1.000){2}{\rule{0.400pt}{0.241pt}}
\put(987.67,681){\rule{0.400pt}{0.723pt}}
\multiput(987.17,681.00)(1.000,1.500){2}{\rule{0.400pt}{0.361pt}}
\put(988.67,684){\rule{0.400pt}{0.482pt}}
\multiput(988.17,684.00)(1.000,1.000){2}{\rule{0.400pt}{0.241pt}}
\put(989.67,686){\rule{0.400pt}{0.482pt}}
\multiput(989.17,686.00)(1.000,1.000){2}{\rule{0.400pt}{0.241pt}}
\put(990.67,688){\rule{0.400pt}{0.482pt}}
\multiput(990.17,688.00)(1.000,1.000){2}{\rule{0.400pt}{0.241pt}}
\put(991.67,690){\rule{0.400pt}{0.723pt}}
\multiput(991.17,690.00)(1.000,1.500){2}{\rule{0.400pt}{0.361pt}}
\put(992.67,693){\rule{0.400pt}{0.482pt}}
\multiput(992.17,693.00)(1.000,1.000){2}{\rule{0.400pt}{0.241pt}}
\put(993.67,695){\rule{0.400pt}{0.482pt}}
\multiput(993.17,695.00)(1.000,1.000){2}{\rule{0.400pt}{0.241pt}}
\put(994.67,697){\rule{0.400pt}{0.482pt}}
\multiput(994.17,697.00)(1.000,1.000){2}{\rule{0.400pt}{0.241pt}}
\put(985.0,672.0){\rule[-0.200pt]{0.400pt}{0.482pt}}
\put(995.67,701){\rule{0.400pt}{0.482pt}}
\multiput(995.17,701.00)(1.000,1.000){2}{\rule{0.400pt}{0.241pt}}
\put(996.67,703){\rule{0.400pt}{0.482pt}}
\multiput(996.17,703.00)(1.000,1.000){2}{\rule{0.400pt}{0.241pt}}
\put(997.67,705){\rule{0.400pt}{0.723pt}}
\multiput(997.17,705.00)(1.000,1.500){2}{\rule{0.400pt}{0.361pt}}
\put(998.67,708){\rule{0.400pt}{0.482pt}}
\multiput(998.17,708.00)(1.000,1.000){2}{\rule{0.400pt}{0.241pt}}
\put(999.67,710){\rule{0.400pt}{0.482pt}}
\multiput(999.17,710.00)(1.000,1.000){2}{\rule{0.400pt}{0.241pt}}
\put(1000.67,712){\rule{0.400pt}{0.482pt}}
\multiput(1000.17,712.00)(1.000,1.000){2}{\rule{0.400pt}{0.241pt}}
\put(1001.67,714){\rule{0.400pt}{0.482pt}}
\multiput(1001.17,714.00)(1.000,1.000){2}{\rule{0.400pt}{0.241pt}}
\put(1003,715.67){\rule{0.241pt}{0.400pt}}
\multiput(1003.00,715.17)(0.500,1.000){2}{\rule{0.120pt}{0.400pt}}
\put(1003.67,717){\rule{0.400pt}{0.482pt}}
\multiput(1003.17,717.00)(1.000,1.000){2}{\rule{0.400pt}{0.241pt}}
\put(1004.67,719){\rule{0.400pt}{0.482pt}}
\multiput(1004.17,719.00)(1.000,1.000){2}{\rule{0.400pt}{0.241pt}}
\put(1005.67,721){\rule{0.400pt}{0.482pt}}
\multiput(1005.17,721.00)(1.000,1.000){2}{\rule{0.400pt}{0.241pt}}
\put(1006.67,723){\rule{0.400pt}{0.482pt}}
\multiput(1006.17,723.00)(1.000,1.000){2}{\rule{0.400pt}{0.241pt}}
\put(996.0,699.0){\rule[-0.200pt]{0.400pt}{0.482pt}}
\put(1007.67,727){\rule{0.400pt}{0.482pt}}
\multiput(1007.17,727.00)(1.000,1.000){2}{\rule{0.400pt}{0.241pt}}
\put(1009,728.67){\rule{0.241pt}{0.400pt}}
\multiput(1009.00,728.17)(0.500,1.000){2}{\rule{0.120pt}{0.400pt}}
\put(1009.67,730){\rule{0.400pt}{0.482pt}}
\multiput(1009.17,730.00)(1.000,1.000){2}{\rule{0.400pt}{0.241pt}}
\put(1010.67,732){\rule{0.400pt}{0.482pt}}
\multiput(1010.17,732.00)(1.000,1.000){2}{\rule{0.400pt}{0.241pt}}
\put(1011.67,734){\rule{0.400pt}{0.482pt}}
\multiput(1011.17,734.00)(1.000,1.000){2}{\rule{0.400pt}{0.241pt}}
\put(1013,735.67){\rule{0.241pt}{0.400pt}}
\multiput(1013.00,735.17)(0.500,1.000){2}{\rule{0.120pt}{0.400pt}}
\put(1013.67,737){\rule{0.400pt}{0.482pt}}
\multiput(1013.17,737.00)(1.000,1.000){2}{\rule{0.400pt}{0.241pt}}
\put(1014.67,739){\rule{0.400pt}{0.482pt}}
\multiput(1014.17,739.00)(1.000,1.000){2}{\rule{0.400pt}{0.241pt}}
\put(1016,740.67){\rule{0.241pt}{0.400pt}}
\multiput(1016.00,740.17)(0.500,1.000){2}{\rule{0.120pt}{0.400pt}}
\put(1016.67,742){\rule{0.400pt}{0.482pt}}
\multiput(1016.17,742.00)(1.000,1.000){2}{\rule{0.400pt}{0.241pt}}
\put(1018,743.67){\rule{0.241pt}{0.400pt}}
\multiput(1018.00,743.17)(0.500,1.000){2}{\rule{0.120pt}{0.400pt}}
\put(1008.0,725.0){\rule[-0.200pt]{0.400pt}{0.482pt}}
\put(1018.67,747){\rule{0.400pt}{0.482pt}}
\multiput(1018.17,747.00)(1.000,1.000){2}{\rule{0.400pt}{0.241pt}}
\put(1020,748.67){\rule{0.241pt}{0.400pt}}
\multiput(1020.00,748.17)(0.500,1.000){2}{\rule{0.120pt}{0.400pt}}
\put(1020.67,750){\rule{0.400pt}{0.482pt}}
\multiput(1020.17,750.00)(1.000,1.000){2}{\rule{0.400pt}{0.241pt}}
\put(1022,751.67){\rule{0.241pt}{0.400pt}}
\multiput(1022.00,751.17)(0.500,1.000){2}{\rule{0.120pt}{0.400pt}}
\put(1023,752.67){\rule{0.241pt}{0.400pt}}
\multiput(1023.00,752.17)(0.500,1.000){2}{\rule{0.120pt}{0.400pt}}
\put(1023.67,754){\rule{0.400pt}{0.482pt}}
\multiput(1023.17,754.00)(1.000,1.000){2}{\rule{0.400pt}{0.241pt}}
\put(1025,755.67){\rule{0.241pt}{0.400pt}}
\multiput(1025.00,755.17)(0.500,1.000){2}{\rule{0.120pt}{0.400pt}}
\put(1025.67,757){\rule{0.400pt}{0.482pt}}
\multiput(1025.17,757.00)(1.000,1.000){2}{\rule{0.400pt}{0.241pt}}
\put(1027,758.67){\rule{0.241pt}{0.400pt}}
\multiput(1027.00,758.17)(0.500,1.000){2}{\rule{0.120pt}{0.400pt}}
\put(1028,759.67){\rule{0.241pt}{0.400pt}}
\multiput(1028.00,759.17)(0.500,1.000){2}{\rule{0.120pt}{0.400pt}}
\put(1028.67,761){\rule{0.400pt}{0.482pt}}
\multiput(1028.17,761.00)(1.000,1.000){2}{\rule{0.400pt}{0.241pt}}
\put(1030,762.67){\rule{0.241pt}{0.400pt}}
\multiput(1030.00,762.17)(0.500,1.000){2}{\rule{0.120pt}{0.400pt}}
\put(1019.0,745.0){\rule[-0.200pt]{0.400pt}{0.482pt}}
\put(1030.67,765){\rule{0.400pt}{0.482pt}}
\multiput(1030.17,765.00)(1.000,1.000){2}{\rule{0.400pt}{0.241pt}}
\put(1032,766.67){\rule{0.241pt}{0.400pt}}
\multiput(1032.00,766.17)(0.500,1.000){2}{\rule{0.120pt}{0.400pt}}
\put(1033,767.67){\rule{0.241pt}{0.400pt}}
\multiput(1033.00,767.17)(0.500,1.000){2}{\rule{0.120pt}{0.400pt}}
\put(1034,768.67){\rule{0.241pt}{0.400pt}}
\multiput(1034.00,768.17)(0.500,1.000){2}{\rule{0.120pt}{0.400pt}}
\put(1035,769.67){\rule{0.241pt}{0.400pt}}
\multiput(1035.00,769.17)(0.500,1.000){2}{\rule{0.120pt}{0.400pt}}
\put(1035.67,771){\rule{0.400pt}{0.482pt}}
\multiput(1035.17,771.00)(1.000,1.000){2}{\rule{0.400pt}{0.241pt}}
\put(1037,772.67){\rule{0.241pt}{0.400pt}}
\multiput(1037.00,772.17)(0.500,1.000){2}{\rule{0.120pt}{0.400pt}}
\put(1038,773.67){\rule{0.241pt}{0.400pt}}
\multiput(1038.00,773.17)(0.500,1.000){2}{\rule{0.120pt}{0.400pt}}
\put(1039,774.67){\rule{0.241pt}{0.400pt}}
\multiput(1039.00,774.17)(0.500,1.000){2}{\rule{0.120pt}{0.400pt}}
\put(1040,775.67){\rule{0.241pt}{0.400pt}}
\multiput(1040.00,775.17)(0.500,1.000){2}{\rule{0.120pt}{0.400pt}}
\put(1041,776.67){\rule{0.241pt}{0.400pt}}
\multiput(1041.00,776.17)(0.500,1.000){2}{\rule{0.120pt}{0.400pt}}
\put(1031.0,764.0){\usebox{\plotpoint}}
\put(1042,778.67){\rule{0.241pt}{0.400pt}}
\multiput(1042.00,778.17)(0.500,1.000){2}{\rule{0.120pt}{0.400pt}}
\put(1043,779.67){\rule{0.241pt}{0.400pt}}
\multiput(1043.00,779.17)(0.500,1.000){2}{\rule{0.120pt}{0.400pt}}
\put(1044,780.67){\rule{0.241pt}{0.400pt}}
\multiput(1044.00,780.17)(0.500,1.000){2}{\rule{0.120pt}{0.400pt}}
\put(1045,781.67){\rule{0.241pt}{0.400pt}}
\multiput(1045.00,781.17)(0.500,1.000){2}{\rule{0.120pt}{0.400pt}}
\put(1046,782.67){\rule{0.241pt}{0.400pt}}
\multiput(1046.00,782.17)(0.500,1.000){2}{\rule{0.120pt}{0.400pt}}
\put(1047,783.67){\rule{0.241pt}{0.400pt}}
\multiput(1047.00,783.17)(0.500,1.000){2}{\rule{0.120pt}{0.400pt}}
\put(1048,784.67){\rule{0.241pt}{0.400pt}}
\multiput(1048.00,784.17)(0.500,1.000){2}{\rule{0.120pt}{0.400pt}}
\put(1049,785.67){\rule{0.241pt}{0.400pt}}
\multiput(1049.00,785.17)(0.500,1.000){2}{\rule{0.120pt}{0.400pt}}
\put(1050,786.67){\rule{0.241pt}{0.400pt}}
\multiput(1050.00,786.17)(0.500,1.000){2}{\rule{0.120pt}{0.400pt}}
\put(1051,787.67){\rule{0.241pt}{0.400pt}}
\multiput(1051.00,787.17)(0.500,1.000){2}{\rule{0.120pt}{0.400pt}}
\put(1042.0,778.0){\usebox{\plotpoint}}
\put(1053,788.67){\rule{0.241pt}{0.400pt}}
\multiput(1053.00,788.17)(0.500,1.000){2}{\rule{0.120pt}{0.400pt}}
\put(1052.0,789.0){\usebox{\plotpoint}}
\put(1054,790.67){\rule{0.241pt}{0.400pt}}
\multiput(1054.00,790.17)(0.500,1.000){2}{\rule{0.120pt}{0.400pt}}
\put(1055,791.67){\rule{0.241pt}{0.400pt}}
\multiput(1055.00,791.17)(0.500,1.000){2}{\rule{0.120pt}{0.400pt}}
\put(1054.0,790.0){\usebox{\plotpoint}}
\put(1057,792.67){\rule{0.241pt}{0.400pt}}
\multiput(1057.00,792.17)(0.500,1.000){2}{\rule{0.120pt}{0.400pt}}
\put(1058,793.67){\rule{0.241pt}{0.400pt}}
\multiput(1058.00,793.17)(0.500,1.000){2}{\rule{0.120pt}{0.400pt}}
\put(1059,794.67){\rule{0.241pt}{0.400pt}}
\multiput(1059.00,794.17)(0.500,1.000){2}{\rule{0.120pt}{0.400pt}}
\put(1056.0,793.0){\usebox{\plotpoint}}
\put(1061,795.67){\rule{0.241pt}{0.400pt}}
\multiput(1061.00,795.17)(0.500,1.000){2}{\rule{0.120pt}{0.400pt}}
\put(1062,796.67){\rule{0.241pt}{0.400pt}}
\multiput(1062.00,796.17)(0.500,1.000){2}{\rule{0.120pt}{0.400pt}}
\put(1060.0,796.0){\usebox{\plotpoint}}
\put(1064,797.67){\rule{0.241pt}{0.400pt}}
\multiput(1064.00,797.17)(0.500,1.000){2}{\rule{0.120pt}{0.400pt}}
\put(1063.0,798.0){\usebox{\plotpoint}}
\put(1065.0,799.0){\usebox{\plotpoint}}
\put(1066,799.67){\rule{0.241pt}{0.400pt}}
\multiput(1066.00,799.17)(0.500,1.000){2}{\rule{0.120pt}{0.400pt}}
\put(1067,800.67){\rule{0.241pt}{0.400pt}}
\multiput(1067.00,800.17)(0.500,1.000){2}{\rule{0.120pt}{0.400pt}}
\put(1065.0,800.0){\usebox{\plotpoint}}
\put(1069,801.67){\rule{0.241pt}{0.400pt}}
\multiput(1069.00,801.17)(0.500,1.000){2}{\rule{0.120pt}{0.400pt}}
\put(1068.0,802.0){\usebox{\plotpoint}}
\put(1071,802.67){\rule{0.241pt}{0.400pt}}
\multiput(1071.00,802.17)(0.500,1.000){2}{\rule{0.120pt}{0.400pt}}
\put(1070.0,803.0){\usebox{\plotpoint}}
\put(1073,803.67){\rule{0.241pt}{0.400pt}}
\multiput(1073.00,803.17)(0.500,1.000){2}{\rule{0.120pt}{0.400pt}}
\put(1072.0,804.0){\usebox{\plotpoint}}
\put(1075,804.67){\rule{0.241pt}{0.400pt}}
\multiput(1075.00,804.17)(0.500,1.000){2}{\rule{0.120pt}{0.400pt}}
\put(1074.0,805.0){\usebox{\plotpoint}}
\put(1076.0,806.0){\usebox{\plotpoint}}
\put(1077.0,806.0){\usebox{\plotpoint}}
\put(1078,806.67){\rule{0.241pt}{0.400pt}}
\multiput(1078.00,806.17)(0.500,1.000){2}{\rule{0.120pt}{0.400pt}}
\put(1077.0,807.0){\usebox{\plotpoint}}
\put(1080,807.67){\rule{0.241pt}{0.400pt}}
\multiput(1080.00,807.17)(0.500,1.000){2}{\rule{0.120pt}{0.400pt}}
\put(1079.0,808.0){\usebox{\plotpoint}}
\put(1082,808.67){\rule{0.241pt}{0.400pt}}
\multiput(1082.00,808.17)(0.500,1.000){2}{\rule{0.120pt}{0.400pt}}
\put(1081.0,809.0){\usebox{\plotpoint}}
\put(1085,809.67){\rule{0.241pt}{0.400pt}}
\multiput(1085.00,809.17)(0.500,1.000){2}{\rule{0.120pt}{0.400pt}}
\put(1083.0,810.0){\rule[-0.200pt]{0.482pt}{0.400pt}}
\put(1087,810.67){\rule{0.241pt}{0.400pt}}
\multiput(1087.00,810.17)(0.500,1.000){2}{\rule{0.120pt}{0.400pt}}
\put(1086.0,811.0){\usebox{\plotpoint}}
\put(1088,812){\usebox{\plotpoint}}
\put(1089,811.67){\rule{0.241pt}{0.400pt}}
\multiput(1089.00,811.17)(0.500,1.000){2}{\rule{0.120pt}{0.400pt}}
\put(1088.0,812.0){\usebox{\plotpoint}}
\put(1092,812.67){\rule{0.241pt}{0.400pt}}
\multiput(1092.00,812.17)(0.500,1.000){2}{\rule{0.120pt}{0.400pt}}
\put(1090.0,813.0){\rule[-0.200pt]{0.482pt}{0.400pt}}
\put(1096,813.67){\rule{0.241pt}{0.400pt}}
\multiput(1096.00,813.17)(0.500,1.000){2}{\rule{0.120pt}{0.400pt}}
\put(1093.0,814.0){\rule[-0.200pt]{0.723pt}{0.400pt}}
\put(1099,814.67){\rule{0.241pt}{0.400pt}}
\multiput(1099.00,814.17)(0.500,1.000){2}{\rule{0.120pt}{0.400pt}}
\put(1097.0,815.0){\rule[-0.200pt]{0.482pt}{0.400pt}}
\put(1100,816){\usebox{\plotpoint}}
\put(1102,815.67){\rule{0.241pt}{0.400pt}}
\multiput(1102.00,815.17)(0.500,1.000){2}{\rule{0.120pt}{0.400pt}}
\put(1100.0,816.0){\rule[-0.200pt]{0.482pt}{0.400pt}}
\put(1106,816.67){\rule{0.241pt}{0.400pt}}
\multiput(1106.00,816.17)(0.500,1.000){2}{\rule{0.120pt}{0.400pt}}
\put(1103.0,817.0){\rule[-0.200pt]{0.723pt}{0.400pt}}
\put(1107.0,818.0){\rule[-0.200pt]{0.964pt}{0.400pt}}
\put(1111.0,818.0){\usebox{\plotpoint}}
\put(1116,818.67){\rule{0.241pt}{0.400pt}}
\multiput(1116.00,818.17)(0.500,1.000){2}{\rule{0.120pt}{0.400pt}}
\put(1111.0,819.0){\rule[-0.200pt]{1.204pt}{0.400pt}}
\put(1117.0,820.0){\rule[-0.200pt]{1.445pt}{0.400pt}}
\put(1123.0,820.0){\usebox{\plotpoint}}
\put(1131,820.67){\rule{0.241pt}{0.400pt}}
\multiput(1131.00,820.17)(0.500,1.000){2}{\rule{0.120pt}{0.400pt}}
\put(1123.0,821.0){\rule[-0.200pt]{1.927pt}{0.400pt}}
\put(1142,821.67){\rule{0.241pt}{0.400pt}}
\multiput(1142.00,821.17)(0.500,1.000){2}{\rule{0.120pt}{0.400pt}}
\put(1132.0,822.0){\rule[-0.200pt]{2.409pt}{0.400pt}}
\put(1161,822.67){\rule{0.241pt}{0.400pt}}
\multiput(1161.00,822.17)(0.500,1.000){2}{\rule{0.120pt}{0.400pt}}
\put(1143.0,823.0){\rule[-0.200pt]{4.336pt}{0.400pt}}
\put(1162.0,824.0){\rule[-0.200pt]{66.729pt}{0.400pt}}
\put(151.0,131.0){\rule[-0.200pt]{0.400pt}{175.375pt}}
\put(151.0,131.0){\rule[-0.200pt]{310.279pt}{0.400pt}}
\put(1439.0,131.0){\rule[-0.200pt]{0.400pt}{175.375pt}}
\put(151.0,859.0){\rule[-0.200pt]{310.279pt}{0.400pt}}
\end{picture}

  \caption{Poder do teste para diferentes médias reais}
  \label{fig:1-power}
\end{figure}

%%% Local Variables:
%%% mode: latex
%%% TeX-master: "../main"
%%% End:
