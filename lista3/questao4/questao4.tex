\newcommand{\TRESicmin}{1.85\xspace}
\newcommand{\TRESicmax}{3.10\xspace}
\newcommand{\TRESicNoveNoveMin}{1.62\xspace}
\newcommand{\TRESicNoveNoveMax}{3.33\xspace}
\newcommand{\TRESnZero}{26\xspace}

\subsection{Teste de hipótese}
\label{questao:4a}
	Para determinar se a proporção populacional de alunos com opinião negativa
	sobre a EAD da TYU é superior à 20\%, definimos as hipotéses:

	\begin{align*} 
		H_0\!:   &\; p = 0.20 \\
		H_1\!:   &\; p > 0.20  \\
		\alpha\!:&\; 0.01
	\end{align*}

	A proporção amostral $\hat{p}$ de alunos (obtida anteriormente na \textit{Lista 2}) com opiniões
	negativas (alunos que se consideram ``Insatisfeitos'' ou ``Muito Insatisfeitos'')
	foi de \QUATROpAmostral, para uma amostra aleatória simples sem reposição de tamanho $n = \QUATROn$.

	\begin{align*} 
		\hat{p}  = \frac{y}{n} = {\frac{\QUATROy}{\QUATROn}} = \QUATROpAmostral
	\end{align*}
	
	Como procura-se evidência de que a proporção de alunos com opiniões negativas seja superior
	a $20\%$ ($0.20$), realiza-se um teste unilateral à direita.
	É possível utilizar a aproximação pela normal, pois:

	\begin{align*}
	  n \times p_0\!  &\; = \QUATROn \cdot 0.20 \\
	  &\; = 40 \geq 5 ,\\
	  & \text{e} \\
	  n \times (1 - p_0)\!	&\;=\; \QUATROn \cdot (1 - 0.20) \\
	  &\;= 160  \geq 5 \\
	\end{align*}

	O cálculo da estatística do teste é feito por:

	\begin{align*}
		z = \frac{y' - n \cdot p_0}{\sqrt{n \cdot p_0 \cdot (1 - p_0)}}
	\end{align*}

	Como $y > n \cdot p_0$, $y' = y + 0.5 = 54.5$. Substituindo os valores, temos:

	\begin{align*}
		z &\; = \frac{\QUATROyLinha - \QUATROn \cdot 0.2}{\sqrt{\QUATROn \cdot 0.2 \cdot (1 - 0.2}}\\
		&\; = \QUATROz
	\end{align*}

	Com o auxílio da tabela de distribuição normal padrão, obtemos a area acima de $z$ e que corresponde 
	ao valor $p$. Para $z = \QUATROz$, $p = \QUATROpValue$. Como $p < \alpha$, rejeita-se $H_0$ ao nível
	de significância de $1\%$ e aceita-se $H_1$. Portanto, conclui-se que há evidência de que a proporção
	populacional com opinião negativa sobre a EAD da TYU seja superior à $20\%$.

\subsection{Poder do Teste}
\label{questao:4b}

	Para obter o tamanho da amostra foi utilizado o software R e a função \r|pwr.p.test|, a qual calcula o poder para testes de proporção de uma amostra.
	A função \r|ES.h|, por sua vez, foi utilizada também para calcular o \textit{Effect Size} para proporção.
	Em vista disso, o seguinte fragmento de código foi executado no ambiente R:

	\inputr{questao4/4b.R}

	Na função \r|pwr.p.test| foram passados como parâmetros os \textit{Effect Sizes} das proporções contidas em \r|probList|;
	o tamanho da amostra (\r|n = 200|); 
	o nível de significância de $1\%$ (\r|sig.level = 0.01|);
	a indicação de que se trata de um teste unilateral à direita (\r|alternative = greater|); 
	e o poder (\r|power|) como NULL, visto que se trata do valor desejado a ser calculado.
	A partir da execução do código R, foi gerada a \autoref{table:4b}, a qual mostra os resultados obtidos do poder do teste calculado para os valores $0.21$, $0.22$,  $0.24$, $0.25$, $0.26$ e $0.27$.

	\begin{table}[ht]
	\centering
	\caption{Poder do teste para diferentes valores de proporção de alunos com opiniões negativas} 
	\label{table:4b}
	\begin{tabular}{rrrrrrr}
		\toprule
		$p$ 	& 0.21 & 0.22 & 0.24 & 0.25 & 0.26 & 0.27 \\ 
		\midrule
		$ES$ 	& \QUATROesVinteUm & \QUATROesVinteDois & \QUATROesVinteQuatro & \QUATROesVinteCinco & \QUATROesVinteSeis & \QUATROesVinteSete \\ 
		$1 - \beta$ & \QUATROpVinteUm & \QUATROpVinteDois & \QUATROpVinteQuatro & \QUATROpVinteCinco & \QUATROpVinteSeis & \QUATROpVinteSete \\ 
	   \bottomrule
	\end{tabular}
	\end{table}

	Como é indicado na \autoref{table:4b}, a probabilidade de um teste $t$ de rejeitar $H_0$ quando a proporção populacional real $p$ é maior que $0.2$ aumenta conforme a proporção real $p$ se distancia de $0.2$.
	Por exemplo, quando $p$ = $0.21$, a hipótese alternativa $H_1$ é verdadeira, no entanto
	o teste tem uma probabilidade de apenas \QUATROpVinteUmP de detectar essa situação a partir de uma
	amostra de $n=\QUATROn$ elementos, rejeitando $H_0$. Porém, caso a proporção real $p$ seja de $0.27$,
	o teste tem \QUATROpVinteSeteP de probabilidade de aceitar a hipótese alternativa $H_1$, para uma amostra de tamanho $n = \QUATROn$.

\subsection{Tamanho mínimo de amostra}
\label{questao:4c}

	O mesmo comando \r|pwr.p.test| do ambiente R, utilizado no \autoref{questao:4b}, pode ser ser usado para calcular o tamanho mínimo da amostra: 

	\inputr{questao4/4c.R}

	Como o objetivo é encontrar o tamanho mínimo da amostra, o parâmetro
	$n$ é passado na função como NULL.
	Como resultado, foram obtidos os dados apresentados pela \autoref{table:4c}. 
	Visto que o tamanho de amostra encontrado pela função não é um inteiro
	(\QUATROtamanhoAmostra), é necessário efetuar um arredondamento para cima,
	obtendo o valor final de \QUATROtamanhoAmostraRounded.

	\begin{table}[ht]
	\centering
	\caption{Poder do Teste para detectar o tamanho da amostra.} 
	\label{table:4c}
	\begin{tabular}{rrrrr}
	  \toprule
		$p$ & $p_0$ & $n$ & $Effect Size$ & $1 - \beta$ \\ 
		\midrule
		$0.23$ & $0.20$ & $\QUATROtamanhoAmostra$ & $\QUATROesAmostra$ & $0.99$ \\ 
		\bottomrule
	\end{tabular}
	\end{table}

	A amostra de \QUATROn elementos, portanto, não é suficiente para detectar com $99\%$ de probabilidade que a proporção	de alunos com opiniões negativas seja igual a $23\%$, considerando $1\%$ de significância.
	Através do cálculo para o tamanho mínimo de amostra realizado no R, 
	obteve-se que este tamanho deve ser de pelo menos \QUATROtamanhoAmostraRounded elementos.
