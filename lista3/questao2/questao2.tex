\newcommand{\TRESicmin}{1.85\xspace}
\newcommand{\TRESicmax}{3.10\xspace}
\newcommand{\TRESicNoveNoveMin}{1.62\xspace}
\newcommand{\TRESicNoveNoveMax}{3.33\xspace}
\newcommand{\TRESnZero}{26\xspace}


\subsection{Teste de Hipótese}

	Para testar a hipótese de que a renda média mensal dos alunos é menor
	que $4,5$ salários mínimos, deve ser aplicado um teste de hipótese com a
	média. Nesse caso, a hipótese nula ($H_0$) é de que a remédia média mensal dos
	alunos é superior ou igual a $4,5$ salários mínimos e a hipótese alternativa
	($H_1$) é de que a renda média mensal dos alunos é inferior a $4,5$ salários
	mínimos. O nível de significância para o teste é de $5\%$.
	%
	\begin{align*}
		H_0: \mu &= \geq 4,5\\
		H_1: \mu &< 4,5   \\
		     \alpha &= 0,05
	\end{align*}

	Como o desvio padrão da população é desconhecido, a estatística do
	teste é expressa pelo valor $t$ da Distribuição T-Student, com grau de
	liberdade $\DOISgrauLiberdade$:
	%
	\begin{align*}
		t &= \frac{(\bar{x} - \mu_0)\cdot\sqrt{n}}{s} \\
		t &= \dfrac{(\DOISmediaAmostra - \DOISmediaPopulacao) \cdot \sqrt{\DOIStamanhoAmostra}}{\DOISdesvioAmostra} \\
		t &\approx 0
	\end{align*}

	Como $t < \alpha$ rejeita-se a hipótese nula. Ou seja, há evidências
	estatísticas de que a média mensal dos alunos é menor que $4,5$ salários
	mínimos com $5\%$ de nível de significância.

\subsection{Poder Teste}

	Para calcular o poder do deste, deve-se empregar a distribuição não
	central T-Student. Portanto, deve-se, primeiramente, calcular o fator de
	centralidade.
	%
	\begin{align*}
		d_{3,50} &= \DOISabicissaA\\
		d_{3,75} &= \DOISabicissaB\\
		d_{3,85} &= \DOISabicissaC\\
		d_{3,95} &= \DOISabicissaD\\
		d_{4,15} &= \DOISabicissaE\\
		d_{4,25} &= \DOISabicissaF\\
		d_{4,35} &= \DOISabicissaG\\
	\end{align*}
	
	Com posse desses dados o poder desses valores, o pode do teste pode ser
	calculado com a função \texttt{pwr.t.test()} do software estatístico R.	Para o
	teste, considerou-se o nível de significância em $1\%$ e o desvio padrão
	amostral como estimativa do desvio padrão populacional.
	%
	\begin{align*}
		1 - \beta_{3,50} &= \DOISpoderA\\
		1 - \beta_{3,75} &= \DOISpoderB\\
		1 - \beta_{3,85} &= \DOISpoderC\\
		1 - \beta_{3,95} &= \DOISpoderD\\
		1 - \beta_{4,15} &= \DOISpoderE\\
		1 - \beta_{4,25} &= \DOISpoderF\\
		1 - \beta_{4,35} &= \DOISpoderG\\
	\end{algin*}
