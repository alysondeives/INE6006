\newcommand{\QUATROpAmostral}{\num{0.2700}\xspace}
\newcommand{\QUATROn}{200\xspace}
\newcommand{\QUATROy}{\num{54.0000}\xspace}
\newcommand{\QUATROyLinha}{\num{54.5000}\xspace}
\newcommand{\QUATROz}{\num{2.5633}\xspace}
\newcommand{\QUATROpValue}{\num{0.0052}\xspace}
\newcommand{\QUATROesVinte}{\num{0.0000}\xspace}
\newcommand{\QUATROesVinteUm}{\num{0.0248}\xspace}
\newcommand{\QUATROesVinteDois}{\num{0.0491}\xspace}
\newcommand{\QUATROesVinteTres}{\num{0.0731}\xspace}
\newcommand{\QUATROesVinteQuatro}{\num{0.0967}\xspace}
\newcommand{\QUATROesVinteCinco}{\num{0.1199}\xspace}
\newcommand{\QUATROesVinteSeis}{\num{0.1428}\xspace}
\newcommand{\QUATROesVinteSete}{\num{0.1655}\xspace}
\newcommand{\QUATROpVinte}{\num{0.0100}\xspace}
\newcommand{\QUATROpVinteUm}{\num{0.0241}\xspace}
\newcommand{\QUATROpVinteDois}{\num{0.0514}\xspace}
\newcommand{\QUATROpVinteTres}{\num{0.0980}\xspace}
\newcommand{\QUATROpVinteQuatro}{\num{0.1687}\xspace}
\newcommand{\QUATROpVinteCinco}{\num{0.2641}\xspace}
\newcommand{\QUATROpVinteSeis}{\num{0.3797}\xspace}
\newcommand{\QUATROpVinteSete}{\num{0.5057}\xspace}
\newcommand{\QUATROesAmostra}{\num{0.0731}\xspace}
\newcommand{\QUATROtamanhoAmostra}{\num{4055.1080}\xspace}
\newcommand{\QUATROtamanhoAmostraRounded}{4056\xspace}


\subsection{Teste de Hipótese}

Para testar a hipótese de que a renda média mensal dos alunos é menor que $4,5$
salários mínimos, deve ser aplicado um teste de hiipótese com a média. Nesse
caso, a hipótese nula ($H_0$) é de que a remédia média mensal dos alunos é
superior ou igual a $4,5$ salários mínimos e a hipótese alternativa ($H_1$) é
de que a renda média mensal dos alunos é inferior a $4,5$ salários mínimos. O
nível de significância para o teste é de $5\%$
%
\begin{align*}
	H_0: \mu &\geq 4,5\\
	H_1: \mu &< 4,5   \\
	     \alpha &= 0,05
\end{align*}

Como o desvio padrão da população é desconhecido, a estatística do teste é
expressa pelo valor $t$ da Distribuição de Student, com grau de liberdade
$\DOISgrauLiberdade$:
%
\begin{align*}
	t &= \frac{(\bar{x} - \mu_0)\cdot\sqrt{n}}{s} \\
	t &= \dfrac{(\DOISmediaAmostra - \DOISmediaPopulacao) \cdot \sqrt{\DOIStamanhoAmostra}}{\DOISdesvioAmostra} \\
	t &\approx 0
\end{align*}

Como $t < \alpha$ rejeita-se a hipótese nula. Ou seja, há evidências
estatísticas de que a média mensal dos alunos é menor que $4,5$ salários
mínimos com $5\%$ de nível de significância.
